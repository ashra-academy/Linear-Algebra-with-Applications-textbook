\section{The Matrix of a Linear Transformation}
\label{sec:9_1}

Let $T : V \to W$ be a linear transformation where $\func{dim }V = n$ and $\func{dim }W = m$. The aim in this section is to describe the action of $T$ as multiplication by an $m \times n$ matrix $A$. The idea is to convert a vector $\vect{v}$ in $V$ into a column in $\RR^n$, multiply that column by $A$ to get a column in $\RR^m$, and convert this column back to get $T(\vect{v})$ in $W$.\index{action!transformations}\index{linear transformations!action of a transformation}\index{linear transformations!matrix of a linear transformation}\index{matrix of a linear transformation}\index{transformations!action}

Converting vectors to columns is a simple matter, but one small change is needed. Up to now the \textit{order} of the vectors in a basis has been of no importance. However, in this section, we shall speak of an \textbf{ordered basis}\index{ordered basis}\index{basis!ordered basis} $\{\vect{b}_{1}, \vect{b}_{2}, \dots, \vect{b}_{n}\}$, which is just a basis where the order in which the vectors are listed is taken into account. Hence $\{\vect{b}_{2}, \vect{b}_{1}, \vect{b}_{3}\}$ is a different \textit{ordered} basis from $\{\vect{b}_{1}, \vect{b}_{2}, \vect{b}_{3}\}$.


If $B = \{\vect{b}_{1}, \vect{b}_{2}, \dots, \vect{b}_{n}\}$ is an ordered basis in a vector space $V$, and if
\begin{equation*}
  \vect{v} = v_1\vect{b}_1 + v_2\vect{b}_2 + \cdots + v_n\vect{b}_n, \quad v_i \in \RR 
\end{equation*}
is a vector in $V$, then the (uniquely determined) numbers $v_{1}, v_{2}, \dots, v_{n}$ are called the \textbf{coordinates}\index{coordinates} of $\vect{v}$ with respect to the basis $B$.


\begin{definition}{Coordinate Vector $C_B(\vect{v})$ of $\vect{v}$ for a basis $B$}{027894}
The \textbf{coordinate vector}\index{coordinate vectors}\index{vectors!coordinate vectors} of $\vect{v}$ with respect to $B$ is defined to be
\begin{equation*}
C_B(\vect{v})= (v_1\vect{b}_1 + v_2\vect{b}_2 + \cdots + v_n\vect{b}_n) = \leftB \begin{array}{c} v_1 \\ v_2 \\ \vdots \\ v_n \end{array} \rightB
\end{equation*}
\end{definition}

\noindent The reason for writing $C_{B}(\vect{v})$ as a column instead of a row will become clear later. Note that $C_{B}(\vect{b}_{i}) = \vect{e}_{i}$ is column $i$ of $I_{n}$.

\begin{example}{}{027904}
The coordinate vector for $\vect{v} = (2, 1, 3)$ with respect to the ordered basis $B = \{(1, 1, 0), (1, 0, 1), (0, 1, 1)\}$ of $\RR^3$ is $C_B(\vect{v}) = \leftB \begin{array}{c} 0 \\ 2 \\ 1 \end{array} \rightB$ because
\begin{equation*}
  \vect{v} = (2, 1, 3) = 0(1, 1, 0) + 2(1, 0, 1) + 1(0, 1, 1)
\end{equation*}
\end{example}

\begin{theorem}{}{027908}
If $V$ has dimension $n$ and $B = \{\vect{b}_{1}, \vect{b}_{2}, \dots, \vect{b}_{n}\}$ is any ordered basis of $V$, the coordinate transformation $C_{B} : V \to \RR^n$ is an isomorphism. In fact, $C_B^{-1} : \RR^n \to V$ is given by\index{coordinate transformation}\index{linear transformations!coordinate transformation}
\begin{equation*}
C_B^{-1} \leftB \begin{array}{c} v_1 \\ v_2 \\ \vdots \\ v_n \end{array} \rightB = v_1\vect{b}_1 + v_2\vect{b}_2 + \cdots + v_n\vect{b}_n \quad \mbox{ for all } \quad \leftB \begin{array}{c} v_1 \\ v_2 \\ \vdots \\ v_n \end{array} \rightB \mbox{ in } \RR^n.
\end{equation*}
\end{theorem}

\begin{proof}
The verification that $C_{B}$ is linear is Exercise~\ref{ex:ex9_1_13}. If $T : \RR^n \to V$ is the map denoted $C_B^{-1}$ in the theorem, one verifies (Exercise~\ref{ex:ex9_1_13}) that $TC_{B} = 1_{V}$ and $C_BT = 1_{\RR^n}$. Note that $C_{B}(\vect{b}_{j})$ is column $j$ of the identity matrix, so $C_{B}$ carries the basis $B$ to the standard basis of $\RR^n$, proving again that it is an isomorphism (Theorem~\ref{thm:022044})
\end{proof}

\begin{wrapfigure}[7]{l}{5cm}
        \vspace*{-1em}
	\centering
	
\begin{tikzpicture}
\node[font=\small,text=black] at (0,0) {$\RR^n$};
\node[font=\small,text=black] at (3,0) {$\RR^m$};
\node[font=\small,text=black] at (0,2) {$V$};
\node[font=\small,text=black] at (3,2) {$W$};
\draw[dkgreenvect,-latex,thick] (0,1.7)--(0,0.3) node[right,font=\small,text=black,midway]{$C_B$};
\draw[dkgreenvect,-latex,thick] (0.2,2)--(2.7,2) node[above,font=\small,text=black,midway]{$T$};
\draw[dkgreenvect,-latex,thick] (3,1.7)--(3,0.3) node[right,font=\small,text=black,midway]{$C_D$};
\draw[dkgreenvect,-latex,thick] (0.3,0)--(2.7,0) node[above,font=\small,text=black,midway]{$T_A$};
\end{tikzpicture}
	%\caption{\label{fig:027931}}
\end{wrapfigure}

Now let $T : V \to W$ be any linear transformation where $\func{dim }V = n$ and $\func{dim }W = m$, and let $B = \{\vect{b}_{1}, \vect{b}_{2}, \dots, \vect{b}_{n}\}$ and $D$ be ordered bases of $V$ and $W$, respectively. Then $C_{B} : V \to \RR^n$ and $C_{D} : W \to \RR^m$ are isomorphisms and we have the situation shown in the diagram where $A$ is an $m \times n$ matrix (to be determined). In fact, the composite
\begin{equation*}
C_DTC_B^{-1} : \RR^n \to \RR^m \mbox{ is a linear transformation}
\end{equation*}
so Theorem~\ref{thm:005789} shows that a unique $m \times n$ matrix $A$ exists such that
\begin{equation*}
C_DTC_B^{-1} = T_A, \quad \mbox{ equivalently } C_DT = T_AC_B
\end{equation*}
$T_{A}$ acts by left multiplication by $A$, so this latter condition is
\begin{equation*}
C_D [T(\vect{v})] = AC_B(\vect{v}) \mbox{ for all } \vect{v} \mbox{ in } V
\end{equation*}
This requirement completely determines $A$. Indeed, the fact that $C_{B}(\vect{b}_{j})$ is column $j$ of the identity matrix gives
\begin{equation*}
\mbox{column } j \mbox{ of } A = AC_B(\vect{b}_j) = C_D[T(\vect{b}_j)]
\end{equation*}
for all $j$. Hence, in terms of its columns,
\begin{equation*}
A = \leftB
\begin{array}{cccc}
 C_D[T(\vect{b}_1)] & C_D[T(\vect{b}_2)] & \cdots & C_D[T(\vect{b}_n)]
\end{array} \rightB
\end{equation*}

\begin{definition}{Matrix $M_{DB}(T)$ of $T : V \to W$ for bases $D$ and $B$}{027950}
This is called the \textbf{matrix of} $T$ \textbf{corresponding to the ordered bases} $B$ \textbf{and} $D$, and we use the following notation:\index{matrix of $T$ corresponding to the ordered bases $B$ and $D$}\index{basis!matrix of $T$ corresponding to the ordered bases $B$ and $D$}\index{basis!ordered basis}\index{ordered basis}\index{linear transformations!matrix of $T$ corresponding to the ordered bases $B$ and $D$}
\begin{equation*}
  M_{DB}(T) = \leftB
\begin{array}{cccc}
 C_D[T(\vect{b}_1)] & C_D[T(\vect{b}_2)] & \cdots & C_D[T(\vect{b}_n)] 
\end{array}\rightB
\end{equation*}
\end{definition}

\noindent This discussion is summarized in the following important theorem.


\begin{theorem}{}{027955}
Let $T : V \to W$ be a linear transformation where $\func{dim }V = n$ and $\func{dim }W = m$, and let $B = \{\vect{b}_{1}, \dots, \vect{b}_{n}\}$ and $D$ be ordered bases of $V$ and $W$, respectively. Then the matrix $M_{DB}(T)$ just given is the unique $m \times n$ matrix $A$ that satisfies
\begin{equation*}
C_DT = T_AC_B
\end{equation*}
Hence the defining property of $M_{DB}(T)$ is
\begin{equation*}
C_D[T(\vect{v})] = M_{DB}(T)C_B(\vect{v}) \mbox{ for all } \vect{v} \mbox{ in } V
\end{equation*}
The matrix $M_{DB}(T)$ is given in terms of its columns by
\begin{equation*}
  M_{DB}(T) = \leftB
\begin{array}{cccc}
 C_D[T(\vect{b}_1)] & C_D[T(\vect{b}_2)] & \cdots & C_D[T(\vect{b}_n)] 
\end{array}\rightB
\end{equation*}
\end{theorem}

\noindent The fact that $T = C_D^{-1}T_AC_B$ means that the action of $T$ on a vector $\vect{v}$ in $V$ can be performed by first taking coordinates (that is, applying $C_{B}$ to $\vect{v}$), then multiplying by $A$ (applying $T_{A}$), and finally converting the resulting $m$-tuple back to a vector in $W$ (applying $C_D^{-1}$).\index{action!transformations}


\begin{example}{}{027973}
Define $T : \vectspace{P}_{2} \to \RR^2$ by $T(a + bx + cx^{2}) = (a + c, b - a - c)$ for all polynomials $a + bx + cx^{2}$. If $B = \{\vect{b}_{1}, \vect{b}_{2}, \vect{b}_{3}\}$ and $D = \{\vect{d}_{1}, \vect{d}_{2}\}$ where
\begin{equation*}
\vect{b}_1 = 1, \vect{b}_2 = x, \vect{b}_3 = x^2 \quad \mbox{ and } \quad \vect{d}_1 = (1, 0), \vect{d}_2 = (0, 1)
\end{equation*}
compute $M_{DB}(T)$ and verify Theorem~\ref{thm:027955}.


\begin{solution}
We have $T(\vect{b}_{1}) = \vect{d}_{1} - \vect{d}_{2}$, $T(\vect{b}_{2}) = \vect{d}_{2}$, and $T(\vect{b}_{3}) = \vect{d}_{1} - \vect{d}_{2}$. Hence
\begin{equation*}
  M_{DB}(T) = \leftB
\begin{array}{ccc}
 C_D[T(\vect{b}_1)] & C_D[T(\vect{b}_2)] & C_D[T(\vect{b}_n)] 
\end{array}\rightB = \leftB \begin{array}{rrr} 1 & 0 & 1 \\ -1 & 1 & -1 \end{array} \rightB
\end{equation*}
If $\vect{v} = a + bx + cx^{2} = a\vect{b}_{1} + b\vect{b}_{2} + c\vect{b}_{3}$, then $T(\vect{v}) = (a + c)\vect{d}_{1} + (b - a - c)\vect{d}_{2}$, so
\begin{equation*}
C_D[T(\vect{v})] = \leftB \begin{array}{c} a + c \\ b - a - c \end{array} \rightB = \leftB \begin{array}{rrr} 1 & 0 & 1 \\ -1 & 1 & -1 \end{array} \rightB \leftB \begin{array}{c} a \\ b \\ c \end{array} \rightB = M_{DB}(T)C_B(\vect{v})
\end{equation*}
as Theorem~\ref{thm:027955} asserts.
\end{solution}
\end{example}

The next example shows how to determine the action of a transformation from its matrix.


\begin{example}{}{028008}
Suppose $T : \vectspace{M}_{22}(\RR) \to \RR^3$ is linear with matrix $M_{DB}(T) = \leftB \begin{array}{rrrr} 1 & -1 & 0 & 0 \\ 0 & 1 & -1 & 0 \\ 0 & 0 & 1 & -1 \end{array} \rightB$ where
\begin{equation*}
B = \left\{ \leftB \begin{array}{cc} 1 & 0 \\ 0 & 0 \end{array} \rightB, \leftB \begin{array}{cc} 0 & 1 \\ 0 & 0 \end{array} \rightB, \leftB \begin{array}{cc} 0 & 0 \\ 1 & 0 \end{array} \rightB, \leftB \begin{array}{cc} 0 & 0 \\ 0 & 1 \end{array} \rightB \right\} \mbox{ and } D = \{(1, 0, 0), (0, 1, 0), (0, 0, 1)\}
\end{equation*}
Compute $T(\vect{v})$ where $\vect{v} = \leftB \begin{array}{cc} a & b \\ c & d \end{array} \rightB$.


\begin{solution}
  The idea is to compute $C_{D}[T(\vect{v})]$ first, and then obtain $T(\vect{v})$. We have
\begin{equation*}
C_D[T(\vect{v})] = M_{DB}(T)C_B(\vect{v}) = \leftB \begin{array}{rrrr} 1 & -1 & 0 & 0 \\ 0 & 1 & -1 & 0 \\ 0 & 0 & 1 & -1 \end{array} \rightB \leftB \begin{array}{c} a \\ b \\ c \\ d \end{array} \rightB = \leftB \begin{array}{c} a - b \\ b - c \\ c - d \end{array} \rightB
\end{equation*}

\begin{align*}
\mbox{Hence } T(\vect{v}) & = (a - b)(1, 0, 0) + (b - c)(0, 1, 0) + (c - d)(0, 0, 1) \\
 & = (a - b, b - c, c - d)
\end{align*}
\end{solution}
\end{example}

\noindent The next two examples will be referred to later.


\begin{example}{}{028025}
Let $A$ be an $m \times n$ matrix, and let $T_{A} : \RR^n \to \RR^m$ be the matrix transformation induced by $A : T_{A}(\vect{x}) = A\vect{x}$ for all columns $\vect{x}$ in $\RR^n$. If $B$ and $D$ are the standard bases of $\RR^n$ and $\RR^m$, respectively (ordered as usual), then
\begin{equation*}
M_{DB}(T_A) = A
\end{equation*}
In other words, the matrix of $T_{A}$ corresponding to the standard bases is $A$ itself.


\begin{solution}
  Write $B = \{\vect{e}_{1}, \dots, \vect{e}_{n}\}$. Because $D$ is the standard basis of $\RR^m$, it is easy to verify that $C_{D}(\vect{y}) = \vect{y}$ for all columns $\vect{y}$ in $\RR^m$. Hence
\begin{equation*}
M_{DB}(T_A) = \leftB
\begin{array}{cccc}
T_A(\vect{e}_1) & T_A(\vect{e}_2) & \cdots & T_A(\vect{e}_n) \end{array} \rightB = \leftB \begin{array}{cccc} A\vect{e}_1 & A\vect{e}_2 & \cdots & A\vect{e}_n \end{array} \rightB = A
\end{equation*}
because $A\vect{e}_{j}$ is the $j$th column of $A$.
\end{solution}
\end{example}

\begin{example}{}{028048}
Let $V$ and $W$ have ordered bases $B$ and $D$, respectively. Let $\func{dim }V = n$.


\begin{enumerate}
\item The identity transformation $1_{V} : V \to V$ has matrix $M_{BB}(1_{V}) = I_{n}$.

\item The zero transformation $0 : V \to W$ has matrix $M_{DB}(0) = 0$.

\end{enumerate}
\end{example}

\noindent The first result in Example~\ref{exa:028048} is false if the two bases of $V$ are not equal. In fact, if $B$ is the standard basis of $\RR^n$, then the basis $D$ of $\RR^n$ can be chosen so that $M_{DB}(1_{\RR^n})$ turns out to be any invertible matrix we wish (Exercise~\ref{ex:ex9_1_14}).


The next two theorems show that composition of linear transformations is compatible with multiplication of the corresponding matrices.\index{composition}\index{linear transformations!composition}

\begin{theorem}{}{028067}
\begin{wrapfigure}{l}{4cm}
        \vspace*{-1em}
	\centering
	
\begin{tikzpicture}
\node[font=\small,text=black] at (0,0) {$V$};
\node[font=\small,text=black] at (1.5,0) {$W$};
\node[font=\small,text=black] at (3,0) {$U$};
\draw[dkgreenvect,-latex,thick] (0.22,0)--(1.28,0) node[above,font=\small,text=black,midway]{$T$};
\draw[dkgreenvect,-latex,thick] (1.68,0)--(2.78,0) node[above,font=\small,text=black,midway]{$S$};
\draw[-latex, dkgreenvect, thick](0.2,-0.1)..controls (1,-0.4) and (1.8,-0.4)..(2.8,-0.1) node[below,midway,font=\small,text=black]{$ST$};
\end{tikzpicture}
	%\caption{\label{fig:028066}}
\end{wrapfigure}	
	
\setlength{\rightskip}{0pt plus 200pt}
Let $V \stackrel{T}{\to} W \stackrel{S}{\to} U$ be linear transformations and let $B$, $D$, and $E$ be finite ordered bases of $V$, $W$, and $U$, respectively. Then
\begin{equation*}
M_{EB}(ST) = M_{ED}(S) \cdot M_{DB}(T)
\end{equation*}
\end{theorem}

\begin{proof}
We use the property in Theorem~\ref{thm:027955} three times. If $\vect{v}$ is in $V$,
\begin{equation*}
M_{ED}(S)M_{DB}(T)C_B(\vect{v}) = M_{ED}(S)C_D[T(\vect{v})] = C_E[ST(\vect{v})] = M_{EB}(ST)C_B(\vect{v})
\end{equation*}
If $B = \{\vect{e}_{1}, \dots, \vect{e}_{n}\}$, then $C_{B}(\vect{e}_{j})$ is column $j$ of $I_{n}$. Hence taking $\vect{v} = \vect{e}_{j}$ shows that $M_{ED}(S)M_{DB}(T)$ and $M_{EB}(ST)$ have equal $j$th columns. The theorem follows.
\end{proof}

\newpage
\begin{theorem}{}{028086}
Let $T : V \to W$ be a linear transformation, where $\func{dim }V = \func{dim }W = n$. The following are equivalent.
\vspace{-1em}
\begin{enumerate}
\item $T$ is an isomorphism.\index{isomorphism}

\item $M_{DB}(T)$ is invertible for all ordered bases $B$ and $D$ of $V$ and $W$.

\item $M_{DB}(T)$ is invertible for some pair of ordered bases $B$ and $D$ of $V$ and $W$.

\end{enumerate}

When this is the case, $[M_{DB}(T)]^{-1} = M_{BD}(T^{-1})$.
\end{theorem}

\begin{proof}
(1) $\Rightarrow$ (2). We have $V \stackrel{T}{\to} W \stackrel{T^{-1}}{\to} V$, so Theorem~\ref{thm:028067} and Example~\ref{exa:028048} give
\begin{equation*}
M_{BD}(T^{-1})M_{DB}(T) = M_{BB}(T^{-1}T) = M_{BB}(1v) = I_n
\end{equation*}
Similarly, $M_{DB}(T)M_{BD}(T^{-1}) = I_{n}$, proving (2) (and the last statement in the theorem).


(2) $\Rightarrow$ (3). This is clear.

\begin{wrapfigure}{l}{4cm}
	\centering
	
\begin{tikzpicture}
\node[font=\small,text=black] at (0,0) {$\RR^n$};
\node[font=\small,text=black] at (1.6,0) {$\RR^n$};
\node[font=\small,text=black] at (3.2,0) {$\RR^n$};
\draw[dkgreenvect,-latex,thick] (0.29,0)--(1.36,0) node[above,font=\small,text=black,midway]{$T_{A^{-1}}$};
\draw[dkgreenvect,-latex,thick] (1.9,0)--(2.95,0) node[above,font=\small,text=black,midway]{$T_A$};
\draw[-latex, dkgreenvect, thick](0.28,-0.1)..controls (1.15,-0.4) and (1.95,-0.4)..(2.96,-0.1) node[below,midway,font=\small,text=black]{$T_{A}T_{A^{-1}}$};
\end{tikzpicture}
	%\caption{\label{fig:028108}}
\end{wrapfigure}
(3) $\Rightarrow$ (1). Suppose that $T_{DB}(T)$ is invertible for some bases $B$ and $D$ and, for convenience, write $A = M_{DB}(T)$. Then we have $C_{D}T = T_{A}C_{B}$ by Theorem~\ref{thm:027955}, so
\begin{equation*}
T = (C_D)^{-1}T_AC_B
\end{equation*}
by Theorem~\ref{thm:027908} where $(C_{D})^{-1}$ and $C_{B}$ are isomorphisms. Hence (1) follows if we can demonstrate that $T_{A} : \RR^n \to \RR^n$ is also an isomorphism. But $A$ is invertible by (3) and one verifies that $T_AT_{A^{-1}} = 1_{\RR^n} = T_{A^{-1}}T_A$. So $T_{A}$ is indeed invertible (and $(T_{A})^{-1} = T_{A^{-1}}$).
\end{proof}

In Section~\ref{sec:7_2} we defined the $\func{rank}$ of a linear transformation $T : V \to W$ by $\func{rank }T = \func{dim}(\func{im }T)$. Moreover, if $A$ is any $m \times n$ matrix and $T_{A} : \RR^n \to \RR^m$ is the matrix transformation, we showed that $\func{rank}(T_{A}) = \func{rank }A$. So it may not be surprising that $\func{rank }T$ equals the $\func{rank}$ of any matrix of $T$.\index{rank!linear transformation}


\begin{theorem}{}{028139}
Let $T : V \to W$ be a linear transformation where $\func{dim }V = n$ and $\func{dim }W = m$. If $B$ and $D$ are any ordered bases of $V$ and $W$, then $\func{rank }T = \func{rank}[M_{DB}(T)]$.\index{basis!choice of basis}\index{choice of basis}
\end{theorem}

\begin{proof}
Write $A = M_{DB}(T)$ for convenience. The column space of $A$ is $U = \{A\vect{x} \mid \vect{x}$ in $\RR^n\}$. This means $\func{rank }A = \func{dim }U$ and so, because $\func{rank }T = \func{dim}(\func{im }T)$, it suffices to find an isomorphism $S : \func{im }T \to U$. Now every vector in $\func{im }T$ has the form $T(\vect{v})$, $\vect{v}$ in $V$. By Theorem~\ref{thm:027955}, $C_{D}[T(\vect{v})] = AC_{B}(\vect{v})$ lies in $U$. So define $S : \func{im }T \to U$ by
\begin{equation*}
S[T(\vect{v})] = C_D[T(\vect{v})] \mbox{ for all vectors } T(\vect{v}) \in \func{im } T
\end{equation*}
The fact that $C_{D}$ is linear and one-to-one implies immediately that $S$ is linear and one-to-one. To see that $S$ is onto, let $A\vect{x}$ be any member of $U$, $\vect{x}$ in $\RR^n$. Then $\vect{x} = C_{B}(\vect{v})$ for some $\vect{v}$ in $V$ because $C_{B}$ is onto. Hence $A\vect{x} = AC_{B}(\vect{v}) = C_{D}[T(\vect{v})] = S[T(\vect{v})]$, so $S$ is onto. This means that $S$ is an isomorphism.
\end{proof}

\begin{example}{}{028158}
Define $T : \vectspace{P}_{2} \to \RR^3$ by $T(a + bx + cx^{2}) = (a - 2b, 3c - 2a, 3c - 4b)$ for $a$, $b$, $c \in \RR$. Compute $\func{rank }T$.


\begin{solution}
  Since $\func{rank }T = \func{rank }[M_{DB}(T)]$ for \textit{any} bases $B \subseteq \vectspace{P}_{2}$ and $D \subseteq \RR^3$, we choose the most convenient ones: $B = \{1, x, x^{2}\}$ and $D = \{(1, 0, 0), (0, 1, 0), (0, 0, 1)\}$. Then $M_{DB}(T) = \leftB \begin{array}{ccc} C_{D}[T(1)] & C_{D}[T(x)] & C_{D}[T(x^{2})] \end{array} \rightB = A$ where
\begin{equation*}
A = \leftB \begin{array}{rrr} 1 & -2 & 0 \\ -2 & 0 & 3 \\ 0 & -4 & 3 \end{array} \rightB. \quad \mbox{Since } A \to \leftB \begin{array}{rrr} 1 & -2 & 0 \\ 0 & -4 & 3 \\ 0 & -4 & 3 \end{array} \rightB \to \leftB \begin{array}{rrr} 1 & -2 & 0 \\ 0 & 1 & -\frac{3}{4} \\ 0 & 0 & 0 \end{array} \rightB
\end{equation*}
we have $\func{rank }A = 2$. Hence $\func{rank }T = 2$ as well.
\end{solution}
\end{example}

We conclude with an example showing that the matrix of a linear transformation can be made very simple by a careful choice of the two bases.


\begin{example}{}{028178}
Let $T : V \to W$ be a linear transformation where $\func{dim }V = n$ and $\func{dim }W = m$. Choose an ordered basis $B = \{\vect{b}_{1}, \dots, \vect{b}_{r}, \vect{b}_{r+1}, \dots, \vect{b}_{n}\}$ of $V$ in which $\{\vect{b}_{r+1}, \dots, \vect{b}_{n}\}$ is a basis of $\func{ker }T$, possibly empty. Then $\{T(\vect{b}_{1}), \dots, T(\vect{b}_{r})\}$ is a basis of $\func{im }T$ by Theorem~\ref{thm:021572}, so extend it to an ordered basis $D = \{T(\vect{b}_{1}), \dots, T(\vect{b}_{r}), \vect{f}_{r+1}, \dots, \vect{f}_{m}\}$ of $W$. Because $T(\vect{b}_{r+1}) = \cdots = T(\vect{b}_{n}) = \vect{0}$, we have
\begin{equation*}
M_{DB}(T) = \leftB \begin{array}{cccccc} C_D[T(\vect{b}_1)] & \cdots & C_D[T(\vect{b}_r)] & C_D[T(\vect{b}_{r+1})] & \cdots & C_D[T(\vect{b}_n)] \end{array} \rightB = \leftB \begin{array}{cc} I_r & 0 \\ 0 & 0 \end{array} \rightB
\end{equation*}
Incidentally, this shows that $\func{rank }T = r$ by Theorem~\ref{thm:028139}.
\end{example}

\section*{Exercises for \ref{sec:9_1}}

\begin{Filesave}{solutions}
\solsection{Section~\ref{sec:9_1}}
\end{Filesave}

\begin{multicols}{1}
\begin{ex}
In each case, find the coordinates of $\vect{v}$ with respect to the basis $B$ of the vector space $V$.

\begin{enumerate}[label={\alph*.}]
\item $V = \vectspace{P}_2$, $\vect{v} = 2x^2 + x - 1$, $B = \{x + 1, x^2, 3\}$

\item $V = \vectspace{P}_2$, $\vect{v} = ax^2 + bx + c$, $B = \{x^2, x + 1, x + 2\}$

\item $V = \RR^3$, $\vect{v} = (1, -1, 2)$, \\ $B = \{(1, -1, 0), (1, 1, 1), (0, 1, 1)\}$

\item $V = \RR^3$, $\vect{v} = (a, b, c)$, \\ $B = \{(1, -1, 2), (1, 1, -1), (0, 0, 1)\}$

\item $V = \vectspace{M}_{22}$, $\vect{v} = \leftB \begin{array}{rr} 1 & 2 \\ -1 & 0 \end{array} \rightB$, \\ $B = \left\{ \leftB \begin{array}{rr} 1 & 1 \\ 0 & 0 \end{array} \rightB, \leftB \begin{array}{rr} 1 & 0 \\ 1 & 0 \end{array} \rightB, \leftB \begin{array}{rr} 0 & 0 \\ 1 & 1 \end{array} \rightB, \leftB \begin{array}{rr} 1 & 0 \\ 0 & 1 \end{array} \rightB \right\}$


\end{enumerate}
\begin{sol}
\begin{enumerate}[label={\alph*.}]
\setcounter{enumi}{1}
\item $\leftB \begin{array}{c} a \\ 2b - c \\ c - b \end{array} \rightB$


\setcounter{enumi}{3}
\item  $\frac{1}{2}\leftB \begin{array}{c} a - b \\ a + b \\ -a + 3b + 2c \end{array} \rightB$


\end{enumerate}
\end{sol}
\end{ex}

\begin{ex}
Suppose $T : \vectspace{P}_{2} \to \RR^2$ is a linear transformation. If $B = \{ 1, x, x^{2}\}$ and $D = \{(1, 1), (0, 1)\}$, find the action of $T$ given:


\begin{enumerate}[label={\alph*.}]
\item $M_{DB}(T) = \leftB \begin{array}{rrr} 1 & 2 & -1 \\ -1 & 0 & 1 \end{array} \rightB$


\item $M_{DB}(T) = \leftB \begin{array}{rrr} 2 & 1 & 3 \\ -1 & 0 & -2 \end{array} \rightB$


\end{enumerate}
\begin{sol}
\begin{enumerate}[label={\alph*.}]
\setcounter{enumi}{1}
\item Let $\vect{v} = a + bx + cx^2$. Then $C_D[T(\vect{v})] = M_{DB}(T)C_B(\vect{v}) = \leftB \begin{array}{rrr} 2 & 1 & 3 \\ -1 & 0 & -2 \end{array} \rightB\leftB \begin{array}{c} a \\ b \\ c \end{array} \rightB = \leftB \begin{array}{c} 2a + b + 3c \\ -a - 2c \end{array} \rightB$

Hence
\begin{align*}
T(\vect{v}) & = (2a + b + 3c)(1, 1) + (-a - 2c)(0, 1) \\
& = (2a + b + 3c, a + b + c).
\end{align*}

\end{enumerate}
\end{sol}
\end{ex}

\begin{ex}
In each case, find the matrix of the linear transformation $T : V \to W$ corresponding to the bases $B$ and $D$ of $V$ and $W$, respectively.


\begin{enumerate}[label={\alph*.}]
\item $T : \vectspace{M}_{22} \to \RR$, $T(A) = \func{tr } A$; \\
\hspace*{-1em}$B= \left\{ \leftB \begin{array}{rr} 1 & 0 \\ 0 & 0 \end{array} \rightB, \leftB \begin{array}{rr} 0 & 1 \\ 0 & 0 \end{array} \rightB, \leftB \begin{array}{rr} 0 & 0 \\ 1 & 0 \end{array} \rightB, \leftB \begin{array}{rr} 0 & 0 \\ 0 & 1 \end{array} \rightB \right\}$,
$D = \{1\}$


\item $T : \vectspace{M}_{22} \to \vectspace{M}_{22}$, $T(A) = A^T$; 
\\ \hspace*{-1em}$B = D \\ = \left\{ \leftB \begin{array}{rr} 1 & 0 \\ 0 & 0 \end{array} \rightB, \leftB \begin{array}{rr} 0 & 1 \\ 0 & 0 \end{array} \rightB, \leftB \begin{array}{rr} 0 & 0 \\ 1 & 0 \end{array} \rightB, \leftB \begin{array}{rr} 0 & 0 \\ 0 & 1 \end{array} \rightB \right\}$


\item $T : \vectspace{P}_2 \to \vectspace{P}_3$, $T[p(x)] = xp(x)$; 
$B = \{1, x, x^2\}$ and $D = \{1, x, x^2, x^3\}$

\item $T : \vectspace{P}_2 \to \vectspace{P}_2$, $T[p(x)] = p(x + 1)$; 
\\ $B = D = \{1, x, x^2\}$


\end{enumerate}
\begin{sol}
\begin{enumerate}[label={\alph*.}]
\setcounter{enumi}{1}
\item $\leftB \begin{array}{cccc} 1 & 0 & 0 & 0 \\ 0 & 0 & 1 & 0 \\ 0 & 1 & 0 & 0 \\ 0 & 0 & 0 & 1 \end{array} \rightB$


\setcounter{enumi}{3}
\item $\leftB \begin{array}{ccc} 1 & 1 & 1 \\ 0 & 1 & 2 \\ 0 & 0 & 1 \end{array} \rightB$


\end{enumerate}
\end{sol}
\end{ex}

\begin{ex}
In each case, find the matrix of \\ $T : V \to W$ corresponding to the bases $B$ and $D$, respectively, and use it to compute $C_{D}[T(\vect{v})]$, and hence $T(\vect{v})$.


\begin{enumerate}[label={\alph*.}]
\item $T : \RR^3 \to \RR^4$, $T(x, y, z) = (x + z, 2z, y - z, x + 2y)$; $B$ and $D$ standard; $\vect{v} = (1, -1, 3)$


\item $T : \RR^2 \to \RR^4$, $T(x, y) = (2x - y, 3x + 2y, 4y, x)$; $B = \{(1, 1), (1, 0)\}$, $D$ standard; $\vect{v} = (a, b)$


\item $T : \vectspace{P}_2 \to \RR^2$, $T(a + bx + cx^2) = (a + c, 2b)$; \\ $B = \{1, x, x^2\}$, $D = \{(1, 0), (1, -1)\}$; \\ $\vect{v} = a + bx + cx^2$


\item $T : \vectspace{P}_2 \to \RR^2$, $T(a + bx + cx^2) = (a + b, c)$; \\ $B = \{1, x, x^2\}$, $D = \{(1, -1), (1, 1)\}$; \\ $\vect{v} = a + bx + cx^2$


\item $T : \vectspace{M}_{22} \to \RR$, $T\leftB \begin{array}{cc} a & b \\ c & d \end{array} \rightB = a + b + c + d$; 
\\ \hspace*{-1em}$B = \left\{ \leftB \begin{array}{rr} 1 & 0 \\ 0 & 0 \end{array} \rightB, \leftB \begin{array}{rr} 0 & 1 \\ 0 & 0 \end{array} \rightB, \leftB \begin{array}{rr} 0 & 0 \\ 1 & 0 \end{array} \rightB, \leftB \begin{array}{rr} 0 & 0 \\ 0 & 1 \end{array} \rightB \right\}$, \\
$D = \{1\}$; $\vect{v} = \leftB \begin{array}{cc} a & b \\ c & d \end{array} \rightB$


\item $T : \vectspace{M}_{22} \to \vectspace{M}_{22}$, \\ $T\leftB \begin{array}{cc} a & b \\ c & d \end{array} \rightB = \leftB \begin{array}{cc} a & b + c \\ b + c & d \end{array} \rightB$; \\
$B = D = {}$ \\ $\left\{ \leftB \begin{array}{rr} 1 & 0 \\ 0 & 0 \end{array} \rightB, \leftB \begin{array}{rr} 0 & 1 \\ 0 & 0 \end{array} \rightB, \leftB \begin{array}{rr} 0 & 0 \\ 1 & 0 \end{array} \rightB, \leftB \begin{array}{rr} 0 & 0 \\ 0 & 1 \end{array} \rightB \right\}$; 
$\vect{v} = \leftB \begin{array}{cc} a & b \\ c & d \end{array} \rightB$


\end{enumerate}
\begin{sol}
\begin{enumerate}[label={\alph*.}]
\setcounter{enumi}{1}
\item $\leftB \begin{array}{cc} 1 & 2 \\ 5 & 3 \\ 4 & 0 \\ 1 & 1 \end{array} \rightB$; \\ $C_D[T(a, b)] = \leftB \begin{array}{cc} 1 & 2 \\ 5 & 3 \\ 4 & 0 \\ 1 & 1 \end{array} \rightB\leftB \begin{array}{cc} b \\ a - b \end{array} \rightB = \leftB \begin{array}{c} 2a - b \\ 3a + 2b \\ 4b \\ a \end{array} \rightB$


\setcounter{enumi}{3}
\item $\frac{1}{2}\leftB \begin{array}{rrr} 1 & 1 & -1 \\ 1 & 1 & 1 \end{array} \rightB$; $C_D[T(a + bx + cx^2)] = \frac{1}{2}\leftB \begin{array}{rrr} 1 & 1 & -1 \\ 1 & 1 & 1 \end{array} \rightB\leftB \begin{array}{c} a \\ b \\ c \end{array} \rightB = \frac{1}{2}\leftB \begin{array}{c} a + b - c \\ a + b + c \end{array} \rightB$


\setcounter{enumi}{5}
\item $\leftB \begin{array}{cccc} 1 & 0 & 0 & 0 \\ 0 & 1 & 1 & 0 \\ 0 & 1 & 1 & 0 \\ 0 & 0 & 0 & 1 \end{array} \rightB$; $C_D\left(T\leftB \begin{array}{cc} a & b \\ c & d \end{array} \rightB\right) =  \leftB \begin{array}{cccc} 1 & 0 & 0 & 0 \\ 0 & 1 & 1 & 0 \\ 0 & 1 & 1 & 0 \\ 0 & 0 & 0 & 1 \end{array} \rightB\leftB \begin{array}{c} a \\ b \\ c \\ d \end{array} \rightB = \leftB \begin{array}{c} a \\ b + c \\ b + c \\ d \end{array} \rightB$


\end{enumerate}
\end{sol}
\end{ex}

\begin{ex}
In each case, verify Theorem~\ref{thm:028067}. Use the standard basis in $\RR^n$ and $\{1, x, x^{2}\}$ in $\vectspace{P}_{2}$.


\begin{enumerate}[label={\alph*.}]
\item $\RR^3 \stackrel{T}{\to} \RR^2 \stackrel{S}{\to} \RR^4$; $T(a, b, c) = (a + b, b - c)$, $S(a, b) = (a, b - 2a, 3b, a + b)$


\item $\RR^3 \stackrel{T}{\to} \RR^4 \stackrel{S}{\to} \RR^2$; \\ $T(a, b, c) = (a + b, c + b, a + c, b - a)$, \\ $S(a, b, c, d) = (a + b, c - d)$


\item $\vectspace{P}_2 \stackrel{T}{\to} \RR^3 \stackrel{S}{\to} \vectspace{P}_2$; $T(a + bx + cx^2) = (a, b - c, c - a)$, $S(a, b, c) = b + cx + (a - c)x^2$


\item $\RR^3 \stackrel{T}{\to} \vectspace{P}_2 \stackrel{S}{\to} \RR^2$; \\ $T(a, b, c) = (a - b) + (c - a)x + bx^2$, \\ $S(a + bx + cx^2) = (a - b, c)$


\end{enumerate}
\begin{sol}
\begin{enumerate}[label={\alph*.}]
\setcounter{enumi}{1}
\item $M_{ED}(S)M_{DB}(T) = {}$ \\ $\leftB \begin{array}{rrrr} 1 & 1 & 0 & 0 \\ 0 & 0 & 1 & -1 \end{array} \rightB\leftB \begin{array}{rrr} 1 & 1 & 0 \\ 0 & 1 & 1 \\ 1 & 0 & 1 \\ -1 & 1 & 0 \end{array} \rightB = {}$ \\ $\leftB \begin{array}{rrr} 1 & 2 & 1 \\ 2 & -1 & 1 \end{array} \rightB = M_{EB}(ST)$


\setcounter{enumi}{3}
\item $M_{ED}(S)M_{DB}(T) = {}$ \\ $\leftB \begin{array}{rrr} 1 & -1 & 0 \\ 0 & 0 & 1 \end{array} \rightB\leftB \begin{array}{rrr} 1 & -1 & 0 \\ -1 & 0 & 1 \\ 0 & 1 & 0 \end{array} \rightB = {}$ \\ $\leftB \begin{array}{rrr} 2 & -1 & -1 \\ 0 & 1 & 0 \end{array} \rightB = M_{EB}(ST)$


\end{enumerate}
\end{sol}
\end{ex}

\begin{ex}
Verify Theorem~\ref{thm:028067} for \\ $\vectspace{M}_{22} \stackrel{T}{\to} \vectspace{M}_{22} \stackrel{S}{\to} \vectspace{P}_2$ where $T(A) = A^{T}$ and \\ $S \leftB \begin{array}{cc} a & b \\ c & d \end{array} \rightB = b + (a + d)x + cx^2$. Use the bases \\ $B = D = \left\{ \leftB \begin{array}{rr} 1 & 0 \\ 0 & 0 \end{array} \rightB, \leftB \begin{array}{rr} 0 & 1 \\ 0 & 0 \end{array} \rightB, \leftB \begin{array}{rr} 0 & 0 \\ 1 & 0 \end{array} \rightB, \leftB \begin{array}{rr} 0 & 0 \\ 0 & 1 \end{array} \rightB \right\}$ \\ and $E = \{1, x, x^{2}\}$.
\end{ex}

\begin{ex}
In each case, find $T^{-1}$ and verify that $[M_{DB}(T)]^{-1} = M_{BD}(T^{-1})$.


\begin{enumerate}[label={\alph*.}]
\item $T : \RR^2 \to \RR^2$, $T(a, b) = (a + 2b, 2a + 5b)$; \\$B = D = $ standard


\item $T : \RR^3 \to \RR^3$, $T(a, b, c) = (b + c, a + c, a + b)$; $B = D = $ standard


\item $T : \vectspace{P}_2 \to \RR^3$, $T(a + bx + cx^2) = (a - c, b, 2a - c)$; $B = \{1, x, x^2\}$, $D = $ standard


\item $T : \vectspace{P}_2 \to \RR^3$, \\$T(a + bx + cx^2) = (a + b + c, b + c, c)$; \\$B = \{1, x, x^2\}$, $D = $ standard


\end{enumerate}
\begin{sol}
\begin{enumerate}[label={\alph*.}]
\setcounter{enumi}{1}
\item \hspace{1em} \\
\hspace*{-1em}$T^{-1}(a, b, c) = \frac{1}{2}(b + c - a, a + c - b, a + b - c)$; \\
$M_{DB}(T) = \leftB \begin{array}{ccc} 0 & 1 & 1 \\ 1 & 0 & 1 \\ 1 & 1 & 0 \end{array} \rightB$; \\ $M_{BD}(T^{-1}) = \frac{1}{2}\leftB \begin{array}{rrr} -1 & 1 & 1 \\ 1 & -1 & 1 \\ 1 & 1 & -1 \end{array} \rightB$


\setcounter{enumi}{3}
\item $T^{-1}(a, b, c) = (a - b) + (b - c)x + cx^2$; \\
$M_{DB}(T) = \leftB \begin{array}{ccc} 1 & 1 & 1 \\ 0 & 1 & 1 \\ 0 & 0 & 1 \end{array} \rightB$; \\ $M_{BD}(T^{-1}) = \leftB \begin{array}{rrr} 1 & -1 & 0 \\ 0 & 1 & -1 \\ 0 & 0 & 1 \end{array} \rightB$


\end{enumerate}
\end{sol}
\end{ex}

\begin{ex}
In each case, show that $M_{DB}(T)$ is invertible and use the fact that $M_{BD}(T^{-1}) = [M_{BD}(T)]^{-1}$ to determine the action of $T^{-1}$.


\begin{enumerate}[label={\alph*.}]
\item $T : \vectspace{P}_2 \to \RR^3$, $T(a + bx + cx^2) = (a + c, c, b - c)$; $B = \{1, x, x^2\}$, $D = $ standard


\item $T : \vectspace{M}_{22} \to \RR^4$, \\ $T\leftB \begin{array}{cc} a & b \\ c & d \end{array} \rightB = (a + b + c, b + c, c, d)$; \\
\hspace*{-2em}$B = \left\{ \leftB \begin{array}{rr} 1 & 0 \\ 0 & 0 \end{array} \rightB, \leftB \begin{array}{rr} 0 & 1 \\ 0 & 0 \end{array} \rightB, \leftB \begin{array}{rr} 0 & 0 \\ 1 & 0 \end{array} \rightB, \leftB \begin{array}{rr} 0 & 0 \\ 0 & 1 \end{array} \rightB \right\}$, $D = $ standard


\end{enumerate}
\begin{sol}
\begin{enumerate}[label={\alph*.}]
\setcounter{enumi}{1}
\item $M_{DB}(T^{-1}) = [M_{BD}(T)]^{-1} =  \leftB \begin{array}{rrrr} 1 & 1 & 1 & 0 \\ 0 & 1 & 1 & 0 \\ 0 & 0 & 1 & 0 \\ 0 & 0 & 0 & 1 \end{array} \rightB^{-1} = \leftB \begin{array}{rrrr} 1 & -1 & 0 & 0 \\ 0 & 1 & -1 & 0 \\ 0 & 0 & 1 & 0 \\ 0 & 0 & 0 & 1 \end{array} \rightB$.

Hence $C_B[T^{-1}(a, b, c, d)] = {}$ \\ $M_{BD}(T^{-1})C_D(a, b, c, d) = {}$ \\ $\leftB \begin{array}{rrrr} 1 & -1 & 0 & 0 \\ 0 & 1 & -1 & 0 \\ 0 & 0 & 1 & 0 \\ 0 & 0 & 0 & 1 \end{array} \rightB\leftB \begin{array}{c} a \\ b \\ c \\ d \end{array} \rightB = \leftB \begin{array}{c} a - b \\ b - c \\ c \\ d \end{array} \rightB$, so 
$T^{-1}(a, b, c, d) = \leftB \begin{array}{cc} a -b & b - c \\ c & d \end{array} \rightB$.


\end{enumerate}
\end{sol}
\end{ex}

\begin{ex}
Let $D : \vectspace{P}_{3} \to \vectspace{P}_{2}$ be the differentiation map given by $D[p(x)] = p^\prime(x)$. Find the matrix of $D$ corresponding to the bases $B = \{1, x, x^{2}, x^{3}\}$ and \\$E = \{1, x, x^{2}\}$, and use it to compute \\$D(a + bx + cx^{2} + dx^{3})$.
\end{ex}

\begin{ex}
Use Theorem~\ref{thm:028086} to show that \\$T : V \to V$ is not an isomorphism if $\func{ker }T \neq 0$ (assume $\func{dim }V = n$). [\textit{Hint}: Choose any ordered basis $B$ containing a vector in $\func{ker }T$.]
\end{ex}

\begin{ex}
Let $T : V \to \RR$ be a linear transformation, and let $D = \{1\}$ be the basis of $\RR$. Given any ordered basis $B = \{\vect{e}_{1}, \dots, \vect{e}_{n}\}$ of $V$, show that \\$M_{DB}(T) = [T(\vect{e}_{1}) \cdots T(\vect{e}_{n})]$.
\end{ex}

\begin{ex}
Let $T : V \to W$ be an isomorphism, let $B = \{\vect{e}_{1}, \dots, \vect{e}_{n}\}$ be an ordered basis of $V$, and let $D = \{T(\vect{e}_{1}), \dots, T(\vect{e}_{n})\}$. Show that $M_{DB}(T) = I_{n}$---the $n\times n$ identity matrix.

\begin{sol}
Have $C_{D}[T(\vect{e}_{j})]$ = column $j$ of $I_{n}$. Hence $M_{DB}(T) = \leftB \begin{array}{cccc} C_{D}[T(\vect{e}_{1})] & C_{D}[T(\vect{e}_{2})] & \cdots & C_{D}[T(\vect{e}_{n})] \end{array} \rightB = I_{n}$.
\end{sol}
\end{ex}

\begin{ex}\label{ex:ex9_1_13}
Complete the proof of Theorem~\ref{thm:027908}.
\end{ex}

\begin{ex}\label{ex:ex9_1_14}
Let $U$ be any invertible $n \times n$ matrix, and let $D = \{\vect{f}_{1}, \vect{f}_{2}, \dots, \vect{f}_{n}\}$ where $\vect{f}_{j}$ is column $j$ of $U$. Show that $M_{BD}(1_{\RR^n}) = U$ when $B$ is the standard basis of $\RR^n$.
\end{ex}

\begin{ex}
Let $B$ be an ordered basis of the $n$-dimensional space $V$ and let $C_{B} : V \to \RR^n$ be the coordinate transformation. If $D$ is the standard basis of $\RR^n$, show that $M_{DB}(C_{B}) = I_{n}$.
\end{ex}

\begin{ex}
Let $T : \vectspace{P}_{2} \to \RR^3$ be defined by \\ $T(p) = (p(0), p(1), p(2))$ for all $p$ in $\vectspace{P}_{2}$. Let \\ $B = \{1, x, x^{2}\}$ and $D = \{(1, 0, 0), (0, 1, 0), (0, 0, 1)\}$.

\begin{enumerate}[label={\alph*.}]
\item Show that $M_{DB}(T) = \leftB \begin{array}{ccc} 1 & 0 & 0 \\ 1 & 1 & 1 \\ 1 & 2 & 4 \end{array} \rightB$ and conclude that $T$ is an isomorphism.

\item Generalize to $T : \vectspace{P}_{n} \to \RR^{n+1}$ where \\ $T(p) = (p(a_{0}), p(a_{1}), \dots, p(a_{n}))$ and $a_{0}, a_{1}, \dots, a_{n}$ are distinct real numbers. \newline [\textit{Hint}: Theorem~\ref{thm:008552}.]

\end{enumerate}
\begin{sol}
\begin{enumerate}[label={\alph*.}]
\setcounter{enumi}{1}
\item If $D$ is the standard basis of $\RR^{n+1}$ and $B = \{1, x, x^{2}, \dots, x^{n}\}$, then $M_{DB}(T) = {}$ \\
$\leftB \begin{array}{cccc} C_D[T(1)] & C_D[T(x)] & \cdots & C_D[T(x^n)] \end{array} \rightB =  \leftB \begin{array}{ccccc} 1 & a_0 & a_0^2 & \cdots & a_0^n \\ 1 & a_1 & a_1^2 & \cdots & a_1^n \\ 1 & a_2 & a_2^2 & \cdots & a_2^n \\ \vdots & \vdots & \vdots & & \vdots \\ 1 & a_n & a_n^2 & \cdots & a_n^n \\ \end{array} \rightB$.

This matrix has nonzero determinant by Theorem~\ref{thm:008552} (since the $a_{i}$ are distinct), so $T$ is an isomorphism.
\end{enumerate}
\end{sol}
\end{ex}

\begin{ex}
Let $T : \vectspace{P}_{n} \to \vectspace{P}_{n}$ be defined by $T[p(x)] = p(x) + xp^\prime(x)$, where $p^\prime(x)$ denotes the derivative. Show that $T$ is an isomorphism by finding $M_{BB}(T)$ when $B = \{1, x, x^{2}, \dots, x^{n}\}$.
\end{ex}

\begin{ex}
If $k$ is any number, define \\$T_{k} : \vectspace{M}_{22} \to \vectspace{M}_{22}$ by $T_{k}(A) = A + kA^{T}$.

\begin{enumerate}[label={\alph*.}]
\item If $B = $ \\ \hspace*{-2em}$\left\{ \leftB \begin{array}{rr} 1 & 0 \\ 0 & 0 \end{array} \rightB, \leftB \begin{array}{rr} 0 & 0 \\ 0 & 1 \end{array} \rightB, \leftB \begin{array}{rr} 0 & 1 \\ 1 & 0 \end{array} \rightB, \leftB \begin{array}{rr} 0 & 1 \\ -1 & 0 \end{array} \rightB \right\}$ find $M_{BB}(T_k)$, and conclude that $T_k$ is invertible if $k \neq 1$ and $k \neq -1$.

\item Repeat for $T_{k} : \vectspace{M}_{33} \to \vectspace{M}_{33}$. Can you generalize?

\end{enumerate}
\end{ex}

The remaining exercises require the following definitions. If $V$ and $W$ are vector spaces, the set of all linear transformations from $V$ to $W$ will be denoted by 
\begin{equation*}
\vectspace{L}(V, W) = \{T \mid T : V \to W \mbox{ is a linear transformation } \}
\end{equation*}
Given $S$ and $T$ in $\vectspace{L}(V, W)$ and $a$ in $\RR$, define $S + T : V \to W$ and $aT : V \to W$ by
\begin{align*}
(S + T)(\vect{v}) & = S(\vect{v}) + T(\vect{v}) & \mbox{ for all } \vect{v} \mbox{ in } V \\
(aT)(\vect{v}) & = aT(\vect{v}) & \mbox{ for all } \vect{v} \mbox{ in } V
\end{align*}

\begin{ex}\label{ex:ex9_1_19}
Show that $\vectspace{L}(V, W)$ is a vector space.
\end{ex}

\begin{ex}\label{ex:ex9_1_20}
Show that the following properties hold provided that the transformations link together in such a way that all the operations are defined.


\begin{enumerate}[label={\alph*.}]
\item $R(ST) = (RS)T$

\item $1_{W}T = T = T1_{V}$

\item $R(S + T) = RS + RT$

\item $(S + T)R = SR + TR$

\item $(aS)T = a(ST) = S(aT)$

\end{enumerate}
\begin{sol}
\begin{enumerate}[label={\alph*.}]
\setcounter{enumi}{3}
\item $[(S + T)R](\vect{v}) = (S + T)(R(\vect{v})) = S[(R(\vect{v}))] + T[(R(\vect{v}))] = SR(\vect{v}) + TR(\vect{v}) = [SR + TR](\vect{v})$ holds for all $\vect{v}$ in $V$. Hence $(S + T)R = SR + TR$.

\end{enumerate}
\end{sol}
\end{ex}

\begin{ex}
Given $S$ and $T$ in $\vectspace{L}(V, W)$, show that:


\begin{enumerate}[label={\alph*.}]
\item $\func{ker }S \cap \func{ker }T \subseteq \func{ker}(S + T)$

\item $\func{im}(S + T) \subseteq \func{im }S + \func{im }T$

\end{enumerate}
\begin{sol}
\begin{enumerate}[label={\alph*.}]
\setcounter{enumi}{1}
\item  If $\vect{w}$ lies in $\func{im}(S + T)$, then $\vect{w} = (S + T)(\vect{v})$ for some $\vect{v}$ in $V$. But then $\vect{w} = S(\vect{v}) + T(\vect{v})$, so $\vect{w}$ lies in $\func{im }S + \func{im}T$.

\end{enumerate}
\end{sol}
\end{ex}

\begin{ex}
Let $V$ and $W$ be vector spaces. If $X$ is a subset of $V$, define 
\begin{equation*}
X^{0} = \{T \mbox{ in } \vectspace{L}(V, W) \mid T(\vect{v}) = 0 \mbox{ for all } \vect{v} \mbox{ in } X\}
\end{equation*}


\begin{enumerate}[label={\alph*.}]
\item Show that $X^{0}$ is a subspace of $\vectspace{L}(V, W)$.

\item If $X \subseteq X_{1}$, show that $X_1^0 \subseteq X^0$.

\item If $U$ and $U_{1}$ are subspaces of $V$, show that \\$(U + U_1)^0 = U^0 \cap U_1^0$.

\end{enumerate}
\begin{sol}
\begin{enumerate}[label={\alph*.}]
\setcounter{enumi}{1}
\item If $X \subseteq X_{1}$, let $T$ lie in $X_1^0$. Then $T(\vect{v}) = \vect{0}$ for all $\vect{v}$ in $X_{1}$, whence $T(\vect{v}) = \vect{0}$ for all $\vect{v}$ in $X$. Thus $T$ is in $X^{0}$ and we have shown that $X_1^0 \subseteq X^{0}$.

\end{enumerate}
\end{sol}
\end{ex}

\begin{ex}
Define $R : \vectspace{M}_{mn} \to \vectspace{L}(\RR^n, \RR^m)$ by $R(A) = T_{A}$ for each $m \times n$ matrix $A$, where $T_{A} : \RR^n \to \RR^m$ is given by $T_{A}(\vect{x}) = A\vect{x}$ for all $\vect{x}$ in $\RR^n$. Show that $R$ is an isomorphism.
\end{ex}

\begin{ex}
Let $V$ be any vector space (we do not assume it is finite dimensional). Given $\vect{v}$ in $V$, define $S_{\vect{v}} : \RR \to V$ by $S_{\vect{v}}(r) = r\vect{v}$ for all $r$ in $\RR$.


\begin{enumerate}[label={\alph*.}]
\item Show that $S_{\vect{v}}$ lies in $\vectspace{L}(\RR, V)$ for each $\vect{v}$ in $V$.

\item Show that the map $R : V \to \vectspace{L}(\RR, V)$ given by $R(\vect{v}) = S_{\vect{v}}$ is an isomorphism. [\textit{Hint}: To show that $R$ is onto, if $T$ lies in $\vectspace{L}(\RR, V)$, show that $T = S_{\vect{v}}$ where $\vect{v} = T(1)$.]

\end{enumerate}
\begin{sol}
\begin{enumerate}[label={\alph*.}]
\setcounter{enumi}{1}
\item $R$ is linear means $S_{\vect{v}+\vect{w}} = S_{\vect{v}} + S_{\vect{w}}$ and $S_{a\vect{v}} = aS_{\vect{v}}$. These are proved as follows: $S_{\vect{v}+\vect{w}}(r) = r(\vect{v} + \vect{w}) = r\vect{v} + r\vect{w} = S\vect{v}(r) + S\vect{w}(r) = (S\vect{v} + S\vect{w})(r)$, and $S_{a\vect{v}}(r) = r(a\vect{v}) = a(r\vect{v}) = (aS_{\vect{v}})(r)$ for all $r$ in $\RR$. To show $R$ is one-to-one, let $R(\vect{v}) = \vect{0}$. This means $S_{\vect{v}} = 0$ so $0 = S_{\vect{v}}(r) = r\vect{v}$ for all $r$. Hence $\vect{v} = \vect{0}$ (take $r = 1$). Finally, to show $R$ is onto, let $T$ lie in $\vectspace{L}(\RR, V)$. We must find $\vect{v}$ such that $R(\vect{v}) = T$, that is $S_{\vect{v}} = T$. In fact, $\vect{v} = T(1)$ works since then $T(r) = T(r \dotprod 1) = rT(1) = r\vect{v} = S_{\vect{v}}(r)$ holds for all $r$, so $T = S_{\vect{v}}$.

\end{enumerate}
\end{sol}
\end{ex}

\begin{ex}
Let $V$ be a vector space with ordered basis $B = \{\vect{b}_{1}, \vect{b}_{2}, \dots, \vect{b}_{n}\}$. For each $i = 1, 2, \dots, m$, define $S_{i} : \RR \to V$ by $S_{i}(r) = r\vect{b}_{i}$ for all $r$ in $\RR$.


\begin{enumerate}[label={\alph*.}]
\item Show that each $S_{i}$ lies in $\vectspace{L}(\RR, V)$ and $S_{i}(1) = \vect{b}_{i}$.

\item Given $T$ in $\vectspace{L}(\RR, V)$, let \\ $T(1) = a_{1}\vect{b}_{1} + a_{2}\vect{b}_{2} + \cdots + a_{n}\vect{b}_{n}$, $a_{i}$ in $\RR$. Show that $T = a_{1}S_{1} + a_{2}S_{2} + \cdots + a_{n}S_{n}$.

\item Show that $\{S_{1}, S_{2}, \dots, S_{n}\}$ is a basis of $\vectspace{L}(\RR, V)$.

\end{enumerate}
\begin{sol}
\begin{enumerate}[label={\alph*.}]
\setcounter{enumi}{1}
\item Given $T : \RR \to V$, let $T(1) = a_{1}\vect{b}_{1} + \cdots + a_{n}\vect{b}_{n}$, $a_{i}$ in $\RR$. For all $r$ in $\RR$, we have $(a_{1}S_{1} + \cdots + a_{n}S_{n})(r) = a_{1}S_{1}(r) + \cdots + a_{n}S_{n}(r) = (a_{1}r\vect{b}_{1} + \cdots + a_{n}r\vect{b}_{n}) = rT(1) = T(r)$. This shows that $a_{1}S_{1} + \cdots + a_{n}S_{n} = T$.

\end{enumerate}
\end{sol}
\end{ex}

\begin{ex} \label{ex:9_1_26}
Let $\func{dim }V = n$, $\func{dim }W = m$, and let $B$ and $D$ be ordered bases of $V$ and $W$, respectively. Show that $M_{DB} : \vectspace{L}(V, W) \to \vectspace{M}_{mn}$ is an isomorphism of vector spaces. [\textit{Hint}: Let $B = \{\vect{b}_{1}, \dots, \vect{b}_{n}\}$ and $D = \{\vect{d}_{1}, \dots, \vect{d}_{m}\}$. Given $A = \leftB a_{ij} \rightB$ in $\vectspace{M}_{mn}$, show that $A = M_{DB}(T)$ where $T : V \to W$ is defined by \\ $T(\vect{b}_{j}) = a_{1j}\vect{d}_{1} + a_{2j}\vect{d}_{2} + \cdots + a_{mj}\vect{d}_{m}$ for each $j$.]
\end{ex}

\begin{ex}
If $V$ is a vector space, the space $V^{*} = \vectspace{L}(V, \RR)$ is called the \textbf{dual}\index{dual} of $V$. Given a basis $B = \{\vect{b}_{1}, \vect{b}_{2}, \dots, \vect{b}_{n}\}$ of $V$, let $E_{i} : V \to \RR$ for each $i = 1, 2, \dots, n$ be the linear transformation satisfying
\begin{equation*}
E_i(\vect{b}_j) = \left\{ \begin{array}{ll} 0 & \mbox{ if } i \neq j \\ 1 & \mbox{ if } i = j \end{array} \right.
\end{equation*}
(each $E_{i}$ exists by Theorem~\ref{thm:020916}). Prove the following:


\begin{enumerate}[label={\alph*.}]
\item $E_{i}(r_{1}\vect{b}_{1} + \cdots + r_{n}\vect{b}_{n}) = r_{i}$ for each $i = 1, 2, \dots, n$

\item $\vect{v} = E_{1}(\vect{v})\vect{b}_{1} + E_{2}(\vect{v})\vect{b}_{2} + \cdots + E_{n}(\vect{v})\vect{b}_{n}$ for all $\vect{v}$ in $V$

\item $T = T(\vect{b}_{1})E_{1} + T(\vect{b}_{2})E_{2} + \cdots + T(\vect{b}_{n})E_{n}$ for all $T$ in $V^{*}$

\item $\{E_{1}, E_{2}, \dots, E_{n}\}$ is a basis of $V^{*}$ (called the \textbf{dual basis}\index{dual basis}\index{basis!dual basis}\index{dual} of $B$).


Given $\vect{v}$ in $V$, define $\vect{v}^{*} : V \to \RR$ by \\ $\vect{v}^{*}(\vect{w}) = E_{1}(\vect{v})E_{1}(\vect{w}) + E_{2}(\vect{v})E_{2}(\vect{w}) + \cdots + E_{n}(\vect{v})E_{n}(\vect{w})$ for all $\vect{w}$ in $V$. Show that:

\item $\vect{v}^{*} : V \to \RR$ is linear, so $\vect{v}^{*}$ lies in $V^{*}$.

\item $\vect{b}_i^{*} = E_{i}$ for each $i = 1, 2, \dots, n$.

\item The map $R : V \to V^{*}$ with $R(\vect{v}) = \vect{v}^{*}$ is an isomorphism. [\textit{Hint}: Show that $R$ is linear and one-to-one and use Theorem~\ref{thm:022192}. Alternatively, show that $R^{-1}(T) = T(\vect{b}_{1})\vect{b}_{1} + \cdots + T(\vect{b}_{n})\vect{b}_{n}$.]

\end{enumerate}
\begin{sol}
\begin{enumerate}[label={\alph*.}]
\setcounter{enumi}{1}
\item Write $\vect{v} = v_{1}\vect{b}_{1} + \cdots + v_{n}\vect{b}_{n}$, $v_{j}$ in $\RR$. Apply $E_{i}$ to get $E_{i}(\vect{v}) = v_{1}E_{i}(\vect{b}_{1}) + \cdots + v_{n}E_{i}(\vect{b}_{n}) = v_{i}$ by the definition of the $E_{i}$.

\end{enumerate}
\end{sol}
\end{ex}
\end{multicols}

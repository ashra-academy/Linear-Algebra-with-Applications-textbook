\noindent In Section \ref{sec:5_3} we introduced the dot product in $\RR^n$ and extended the basic geometric notions of length and distance. A set $\{ \vect{f}_1, \vect{f}_2, \dots, \vect{f}_m\}$ of nonzero vectors in $\RR^n$ was called an \textbf{orthogonal set}\index{orthogonal sets} if $\vect{f}_i \dotprod \vect{f}_j =0$ for all $i \neq j$, and it was proved that every orthogonal set is independent. In particular, it was observed that the expansion of a vector as a linear combination of orthogonal basis vectors is easy to obtain because formulas exist for the coefficients. Hence the orthogonal bases are the ``nice'' bases, and much of this chapter is devoted to extending results about bases to orthogonal bases. This leads to some very powerful methods and theorems. Our first task is to show that every subspace of $\RR^n$ \textit{has} an orthogonal basis. 

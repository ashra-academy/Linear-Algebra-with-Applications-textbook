\noindent If $V$ and $W$ are vector spaces, a function $T : V \to W$ is a rule that assigns to each vector $\vect{v}$ in $V$ a uniquely determined vector $T(\vect{v})$ in $W$. As mentioned in Section \ref{sec:2_2}, two functions $S : V \to W$ and $T : V \to W$ are equal if $S(\vect{v}) = T(\vect{v})$ for every $\vect{v}$ in $V$. A function $T : V \to W$ is called a \textit{linear transformation} if $T(\vect{v} + \vect{v}_1) = T(\vect{v}) + T(\vect{v}_1)$ for all $\vect{v}$, $\vect{v}_1$ in $V$ and $T(r\vect{v}) = rT(\vect{v})$ for all $\vect{v}$ in $V$ and all scalars $r$. $T(\vect{v})$ is called the \textit{image} of $\vect{v}$ under $T$. We have already studied linear transformation $T : \RR^n \to \RR^m$ and shown (in Section \ref{sec:2_6}) that they are all given by multiplication by a uniquely determined $m \times n$ matrix $A$; that is $T(\vect{x}) = A\vect{x}$ for all $\vect{x}$ in $\RR^n$. In the case of linear operators $\RR^2 \to \RR^2$, this yields an important way to describe geometric functions such as rotations about the origin and reflections in a line through the origin.

In the present chapter we will describe linear transformations in general, introduce the \textit{kernel} and \textit{image} of a linear transformation, and prove a useful result (called the \textit{dimension theorem}\index{dimension theorem}) that relates the dimensions of the kernel and image, and unifies and extends several earlier results. Finally we study the notion of \textit{isomorphic} vector spaces, that is, spaces that are identical except for notation, and relate this to composition of transformations that was introduced in Section \ref{sec:2_3}.

\section{Subspaces and Spanning Sets}
\label{sec:6_2}

\noindent Chapter \ref{chap:5} is essentially about the subspaces of $\RR^n$. We now extend this notion.

\begin{definition}{Subspaces of a Vector Space}{018059}
If $V$ is a vector space, a nonempty subset $U \subseteq V$ is called a \textbf{subspace}\index{subspaces!defined} of $V$ if $U$ is itself a vector space using the addition and scalar multiplication of $V$.\index{subspaces!vector spaces}\index{vector spaces!subspaces}
\end{definition}

\noindent Subspaces of $\RR^n$ (as defined in Section~\ref{sec:5_1}) are subspaces in the present sense by Example~\ref{exa:017680}. Moreover, the defining properties for a subspace of $\RR^n$ actually \textit{characterize} subspaces in general.

\newpage
\begin{theorem}{Subspace Test}{018065}%
A subset $U$ of a vector space is a subspace of $V$ if and only if it satisfies the following three conditions:\index{subspace test}\index{subspaces!subspace test}

\begin{enumerate}
\item $\vect{0}$ lies in $U$ where $\vect{0}$ is the zero vector of $V$.

\item If $\vect{u}_{1}$ and $\vect{u}_{2}$ are in $U$, then $\vect{u}_{1} + \vect{u}_{2}$ is also in $U$.

\item If $\vect{u}$ is in $U$, then $a\vect{u}$ is also in $U$ for each scalar $a$.

\end{enumerate}
\end{theorem}

\begin{proof}
If $U$ is a subspace of $V$, then (2) and (3) hold by axioms A1 and S1 respectively, applied to the vector space $U$. Since $U$ is nonempty (it is a vector space), choose $\vect{u}$ in $U$. Then (1) holds because $\vect{0} = 0\vect{u}$ is in $U$ by (3) and Theorem~\ref{thm:017797}.

Conversely, if (1), (2), and (3) hold, then axioms A1 and S1 hold because of (2) and (3), and axioms A2, A3, S2, S3, S4, and S5 hold in $U$ because they hold in $V$. Axiom A4 holds because the zero vector $\vect{0}$ of $V$ is actually in $U$ by (1), and so serves as the zero of $U$. Finally, given $\vect{u}$ in $U$, then its negative $-\vect{u}$ in $V$ is again in $U$ by (3) because $-\vect{u} = (-1)\vect{u}$ (again using Theorem~\ref{thm:017797}). Hence $-\vect{u}$ serves as the negative of $\vect{u}$ in $U$.
\end{proof}

\noindent Note that the proof of Theorem~\ref{thm:018065} shows that if $U$ is a subspace of $V$, then $U$ and $V$ share the same zero vector, and that the negative of a vector in the space $U$ is the same as its negative in $V$.

\begin{example}{}{018086}
If $V$ is any vector space, show that $\{\vect{0}\}$ and $V$ are subspaces of $V$.

\begin{solution}
$U = V$ clearly satisfies the conditions of the subspace test. As to $U = \{\vect{0}\}$, it satisfies the conditions because $\vect{0} + \vect{0} = \vect{0}$ and $a\vect{0} = \vect{0}$ for all $a$ in $\RR$.
\end{solution}
\end{example}

The vector space $\{\vect{0}\}$ is called the \textbf{zero subspace}\index{zero subspace}\index{subspaces!zero subspace} of $V$.

\begin{example}{}{018093}
Let $\vect{v}$ be a vector in a vector space $V$. Show that the set
\begin{equation*}
\RR\vect{v} = \{a\vect{v} \mid a \mbox{ in } \RR \}
\end{equation*}
of all scalar multiples of $\vect{v}$ is a subspace of $V$.

\begin{solution}
Because $\vect{0} = 0\vect{v}$, it is clear that $\vect{0}$ lies in $\RR\vect{v}$. Given two vectors $a\vect{v}$ and $a_{1}\vect{v}$ in $\RR\vect{v}$, their sum $a\vect{v} + a_{1}\vect{v} = (a + a_{1})\vect{v}$ is also a scalar multiple of $\vect{v}$ and so lies in $\RR\vect{v}$. Hence $\RR\vect{v}$ is closed under addition. Finally, given $a\vect{v}$, $r(a\vect{v}) = (ra)\vect{v}$ lies in $\RR\vect{v}$ for all $r \in \RR$, so $\RR\vect{v}$ is closed under scalar multiplication. Hence the subspace test applies.
\end{solution}
\end{example}

\noindent In particular, given $\vect{d} \neq \vect{0}$ in $\RR^{3}$, $\RR\vect{d}$ is the line through the origin with direction vector $\vect{d}$.

The space $\RR\vect{v}$ in Example~\ref{exa:018093} is described by giving the \textit{form} of each vector in $\RR\vect{v}$. The next example describes a subset $U$ of the space $\vectspace{M}_{nn}$ by giving a \textit{condition} that each matrix of $U$ must satisfy.

\begin{example}{}{018107}
Let $A$ be a fixed matrix in $\vectspace{M}_{nn}$. Show that $U = \{X\mbox{ in } \vectspace{M}_{nn} \mid AX = XA\}$ is a subspace of $\vectspace{M}_{nn}$.

\begin{solution}
If $0$ is the $n \times n$ zero matrix, then $A0 = 0A$, so $0$ satisfies the condition for membership in $U$. Next suppose that $X$ and $X_{1}$ lie in $U$ so that $AX = XA$ and $AX_{1} = X_{1}A$. Then
\begin{align*}
A(X + X_1) 	&= AX + AX_1 = XA + X_1A + (X + X_1)A \\
A(aX) 		&= a(AX) = a(XA) = (aX)A
\end{align*}
for all $a$ in $\RR$, so both $X + X_{1}$ and $aX$ lie in $U$. Hence $U$ is a subspace of $\vectspace{M}_{nn}$.
\end{solution}
\end{example}

Suppose $p(x)$ is a polynomial and $a$ is a number. Then the number $p(a)$ obtained by replacing $x$ by $a$ in the expression for $p(x)$ is called the \textbf{evaluation}\index{evaluation}\index{polynomials!evaluation} of $p(x)$ at $a$. For example, if $p(x) = 5 - 6x + 2x^{2}$, then the evaluation of $p(x)$ at $a = 2$ is $p(2) = 5 - 12 + 8 = 1$. If $p(a) = 0$, the number $a$ is called a \textbf{root}\index{root!of polynomials} of $p(x)$.

\begin{example}{}{018124}
Consider the set $U$ of all polynomials in $\vectspace{P}$ that have $3$ as a root:
\begin{equation*}
U = \{p(x) \in \vectspace{P} \mid p(3) = 0 \}
\end{equation*}
Show that $U$ is a subspace of $\vectspace{P}$.

\begin{solution}
  Clearly, the zero polynomial lies in $U$. Now let $p(x)$ and $q(x)$ lie in $U$ so $p(3) = 0$ and $q(3) = 0$. We have $(p + q)(x) = p(x) + q(x)$ for all $x$, so $(p + q)(3) = p(3) + q(3) = 0 + 0 = 0$, and $U$ is closed under addition. The verification that $U$ is closed under scalar multiplication is similar.
\end{solution}
\end{example}

Recall that the space $\vectspace{P}_{n}$ consists of all polynomials of the form
\begin{equation*}
a_0 + a_1x + a_2x^2 + \dots + a_nx^n
\end{equation*}
where $a_{0}, a_{1}, a_{2}, \dots, a_{n}$ are real numbers, and so is closed under the addition and scalar multiplication in $\vectspace{P}$. Moreover, the zero polynomial is included in $\vectspace{P}_{n}$. Thus the subspace test gives Example~\ref{exa:018140}. 

\begin{example}{}{018140}
$\vectspace{P}_{n}$ is a subspace of $\vectspace{P}$ for each $n \geq 0$.
\end{example}

The next example involves the notion of the derivative $f^\prime$ of a function $f$. (If the reader is not familiar with calculus, this example may be omitted.) A function $f$ defined on the interval $[a, b]$ is called \textbf{differentiable}\index{differentiable function}\index{function!differentiable function} if the derivative $f^\prime(r)$ exists at every $r$ in $[a, b]$.

\begin{example}{}{018145}
Show that the subset $\vectspace{D}[a, b]$ of all \textbf{differentiable functions} on $[a, b]$ is a subspace of the vector space $\vectspace{F}[a, b]$ of all functions on $[a, b]$.

\begin{solution}
The derivative of any constant function is the constant function $0$; in particular, $0$ itself is differentiable and so lies in $\vectspace{D}[a, b]$. If $f$ and $g$ both lie in $\vect{D}[a, b]$ (so that $f^\prime$ and $g^\prime$ exist), then it is a theorem of calculus that $f + g$ and $rf$ are both differentiable for any $r \in \RR$. In fact, $(f + g)^\prime = f^\prime + g^\prime$ and $(rf)^\prime = rf^\prime$, so both lie in $\vectspace{D}[a, b]$. This shows that $\vectspace{D}[a, b]$ is a subspace of $\vectspace{F}[a, b]$.
\end{solution}
\end{example}

\subsection*{Linear Combinations and Spanning Sets}
\begin{definition}{Linear Combinations and Spanning}{018153}
Let $\{\vect{v}_{1}, \vect{v}_{2}, \dots, \vect{v}_{n}\}$ be a set of vectors in a vector space $V$. As in $\RR^n$, a vector $\vect{v}$ is called a \textbf{linear  combination}\index{linear combinations!vectors} of the vectors $\vect{v}_{1}, \vect{v}_{2}, \dots, \vect{v}_{n}$  if it can be expressed in the form

\begin{equation*}
\vect{v} = a_1\vect{v}_1 + a_2\vect{v}_2 + \dots + a_n\vect{v}_n
\end{equation*}
where $a_{1}, a_{2}, \dots, a_{n}$ are scalars, called the \textbf{coefficients}\index{coefficients!linear combination}\index{coefficients!of vectors} of $\vect{v}_{1}, \vect{v}_{2}, \dots, \vect{v}_{n}$. The set of all linear combinations of these vectors is called their \textbf{span}\index{span}, and is denoted by
\begin{equation*}
\func{span}\{\vect{v}_1, \vect{v}_2, \dots, \vect{v}_n\} = \{ a_1\vect{v}_1 + a_2\vect{v}_2 + \dots + a_n\vect{v}_n \mid a_i \mbox{ in } \RR \}
\end{equation*}
\end{definition}

\noindent If it happens that $V = \func{span}\{\vect{v}_{1}, \vect{v}_{2}, \dots, \vect{v}_{n}\}$, these vectors are called a \textbf{spanning set}\index{linear combinations!spanning sets}\index{spanning sets}\index{vector spaces!spanning sets} for $V$. For example, the span of two vectors $\vect{v}$ and $\vect{w}$ is the set
\begin{equation*}
\func{span}\{\vect{v}, \vect{w} \} = \{s\vect{v} + t\vect{w} \mid s \mbox{ and } t \mbox{ in } \RR \}
\end{equation*}
of all sums of scalar multiples of these vectors.

\begin{example}{}{018176}
Consider the vectors $p_{1} = 1 + x + 4x^{2}$ and $p_{2} = 1 + 5x + x^{2}$ in $\vectspace{P}_{2}$. Determine whether $p_{1}$ and $p_{2}$ lie in $\func{span}\{1 + 2x - x^{2}, 3 + 5x + 2x^{2}\}$.

\begin{solution}
For $p_{1}$, we want to determine if $s$ and $t$ exist such that
\begin{equation*}
p_1 = s(1 + 2x - x^2) + t(3 + 5x + 2x^2)
\end{equation*}
Equating coefficients of powers of $x$ (where $x^{0} = 1$) gives
\begin{equation*}
1 = s + 3t,\quad 1 = 2s + 5t, \quad \mbox{ and } \quad 4 = -s + 2t
\end{equation*}
These equations have the solution $s = -2$ and $t = 1$, so $p_{1}$ is indeed in $\func{span}\{1 + 2x - x^{2}, 3 + 5x + 2x^{2}\}$.

Turning to $p_{2} = 1 + 5x + x^{2}$, we are looking for $s$ and $t$ such that 
\begin{equation*}
p_{2} = s(1 + 2x - x^{2}) + t(3 + 5x + 2x^{2})
\end{equation*}
 Again equating coefficients of powers of $x$ gives equations $1 = s + 3t$, $5 = 2s + 5t$, and $1 = -s + 2t$. But in this case there is no solution, so $p_{2}$ is \textit{not} in $\func{span}\{1 + 2x - x^{2}, 3 + 5x + 2x^{2}\}$.
\end{solution}
\end{example}

We saw in Example~\ref{exa:013694} that $\RR^m = \func{span}\{\vect{e}_{1}, \vect{e}_{2}, \dots, \vect{e}_{m}\}$ where the vectors $\vect{e}_{1}, \vect{e}_{2}, \dots, \vect{e}_{m}$ are the columns of the $m \times m$ identity matrix. Of course $\RR^m = \vectspace{M}_{m1}$ is the set of all $m \times 1$ matrices, and there is an analogous spanning set for each space $\vectspace{M}_{mn}$. For example, each $2 \times 2$ matrix has the form
\begin{equation*}
\leftB \begin{array}{rr}
a & b \\
c & d
\end{array} \rightB
= a \leftB \begin{array}{rr}
1 & 0 \\
0 & 0
\end{array} \rightB
+ b \leftB \begin{array}{rr}
0 & 1 \\
0 & 0
\end{array} \rightB
+ c \leftB \begin{array}{rr}
0 & 0 \\
1 & 0
\end{array} \rightB
+ d \leftB \begin{array}{rr}
0 & 0 \\
0 & 1
\end{array} \rightB
\end{equation*}
so
\begin{equation*}
\vectspace{M}_{22} = \func{span}
\left \{
\leftB \begin{array}{rr}
1 & 0 \\
0 & 0
\end{array} \rightB, 
\leftB \begin{array}{rr}
0 & 1 \\
0 & 0
\end{array} \rightB,\
\leftB \begin{array}{rr}
0 & 0 \\
1 & 0
\end{array} \rightB,\
\leftB \begin{array}{rr}
0 & 0 \\
0 & 1
\end{array} \rightB
\right \}
\end{equation*}
Similarly, we obtain

\begin{example}{}{018224}
$\vectspace{M}_{mn}$ is the span of the set of all $m \times n$ matrices with exactly one entry equal to $1$, and all other entries zero.
\end{example}

The fact that every polynomial in $\vectspace{P}_{n}$ has the form $a_{0} + a_{1}x + a_{2}x^{2} + \dots+ a_{n}x^{n}$ where each $a_{i}$ is in $\RR$ shows that

\begin{example}{}{018237}
$\vectspace{P}_{n} = \func{span}\{1, x, x^{2}, \dots, x^{n}\}$.
\end{example}

\noindent In Example~\ref{exa:018093} we saw that $\func{span}\{\vect{v}\} = \{a\vect{v} \mid a \mbox{ in } \RR\} = \RR\vect{v}$ is a subspace for any vector $\vect{v}$ in a vector space $V$. More generally, the span of \textit{any} set of vectors is a subspace. In fact, the proof of Theorem~\ref{thm:013606} goes through to prove:

\begin{theorem}{}{018244}
Let $U = \func{span}\{\vect{v}_{1}, \vect{v}_{2}, \dots, \vect{v}_{n}\}$ in a vector space $V$. Then:

\begin{enumerate}
\item $U$ is a subspace of $V$ containing each of $\vect{v}_{1}, \vect{v}_{2}, \dots, \vect{v}_{n}$.

\item $U$ is the ``smallest'' subspace containing these vectors in the sense that any subspace that contains each of $\vect{v}_{1}, \vect{v}_{2}, \dots, \vect{v}_{n}$ must contain $U$.

\end{enumerate}
\end{theorem}

Here is how condition 2 in Theorem~\ref{thm:018244} is used. Given vectors $\vect{v}_1, \dots, \vect{v}_{k}$ in a vector space $V$ and a subspace $U \subseteq V$, then:
\begin{equation*}
 \func{span}\{\vect{v}_{1}, \dots, \vect{v}_{n}\} \subseteq U \Leftrightarrow \mbox{ each } \vect{v}_i \in U
\end{equation*}
The following examples illustrate this. 

\begin{example}{}{018262}
Show that $\vectspace{P}_{3} = \func{span}\{x^{2} + x^{3}, x, 2x^{2} + 1, 3\}$.

\begin{solution}
Write $U = \func{span}\{x^{2} + x^{3}, x, 2x^{2} + 1, 3\}$. Then $U \subseteq \vectspace{P}_{3}$, and we use the fact that $\vectspace{P}_{3} = \func{span}\{1, x, x^{2}, x^{3}\}$ to show that $\vectspace{P}_{3} \subseteq U$. In fact, $x$ and $1 = \frac{1}{3} \cdot 3$ clearly lie in $U$. But then successively,
\begin{equation*}
x^2 = \frac{1}{2}[(2x^2 + 1) - 1] \quad \mbox{ and } \quad x^3 = (x^2 + x^3) - x^2
\end{equation*}
also lie in $U$. Hence $\vectspace{P}_{3} \subseteq U$ by Theorem~\ref{thm:018244}.
\end{solution}
\end{example}

\begin{example}{}{018282}
Let $\vect{u}$ and $\vect{v}$ be two vectors in a vector space $V$. Show that
\begin{equation*}
\func{span}\{\vect{u}, \vect{v}\} = \func{span}\{\vect{u} + 2\vect{v}, \vect{u} - \vect{v} \}
\end{equation*}
\begin{solution}
We have $\func{span}\{\vect{u} + 2\vect{v}, \vect{u} - \vect{v}\} \subseteq$ $\func{span}\{\vect{u}, \vect{v}\}$ by Theorem~\ref{thm:018244} because both $\vect{u} + 2\vect{v}$ and $\vect{u} - \vect{v}$ lie in $\func{span}\{\vect{u}, \vect{v}\}$. On the other hand,
\begin{equation*}
\vect{u} = \frac{1}{3}(\vect{u} + 2\vect{v}) + \frac{2}{3}(\vect{u} - \vect{v}) \quad \mbox{ and } \quad \vect{v} = \frac{1}{3}(\vect{u} + 2\vect{v}) - \frac{1}{3}(\vect{u} - \vect{v})
\end{equation*}
so $\func{span}\{\vect{u}, \vect{v}\} \subseteq$ $\func{span}\{\vect{u} + 2\vect{v}, \vect{u} - \vect{v}\}$, again by Theorem~\ref{thm:018244}.
\end{solution}
\end{example}

\section*{Exercises for \ref{sec:6_2}}

\begin{Filesave}{solutions}
\solsection{Section~\ref{sec:6_2}}
\end{Filesave}

\begin{multicols}{2}
\begin{ex}
Which of the following are subspaces of $\vectspace{P}_{3}$? Support your answer.

\begin{enumerate}[label={\alph*.}]
\item $U = \{f(x) \mid f(x) \in \vectspace{P}_{3}, f(2) = 1\}$

\item $U = \{xg(x) \mid g(x) \in \vectspace{P}_{2}\}$

\item $U = \{xg(x) \mid g(x) \in \vectspace{P}_{3}\}$

\item $U = \{xg(x) + (1 - x)h(x) \mid g(x) \mbox{ and } h(x) \in \vectspace{P}_{2}\}$

\item $U =$ The set of all polynomials in $\vectspace{P}_{3}$ with constant term $0$

\item $U = \{f(x) \mid f(x) \in \vectspace{P}_{3}, \func{deg} f(x) = 3\}$

\end{enumerate}
\begin{sol}
\begin{enumerate}[label={\alph*.}]
\setcounter{enumi}{1}
\item  Yes

\setcounter{enumi}{3}
\item  Yes

\setcounter{enumi}{5}
\item  No; not closed under addition or scalar multiplication, and $0$ is not in the set.

\end{enumerate}
\end{sol}
\end{ex}


\columnbreak
\begin{ex}
Which of the following are subspaces of $\vectspace{M}_{22}$? Support your answer.

\begin{enumerate}[label={\alph*.}]
\item $U = \left\{
\leftB \begin{array}{rr}
a & b \\
0 & c
\end{array} \rightB
\middle|\ a, b, \mbox{ and } c \mbox{ in } \RR \right\}$

\item $U = \left\{
\leftB \begin{array}{rr}
a & b \\
c & d
\end{array} \rightB
\middle|\ a + b = c + d;\ a, b, c, d \mbox{ in } \RR \right\}$

\item $U = \{A \mid A \in \vectspace{M}_{22}, A = A^{T}\}$

\item $U = \{A \mid A \in \vectspace{M}_{22}, AB = 0\}$, $B$ a fixed $2 \times 2$ matrix

\item $U = \{A \mid A \in \vectspace{M}_{22}, A^{2} = A\}$

\item $U = \{A \mid A \in \vectspace{M}_{22}, A \mbox{ is not invertible}\}$

\item $U = \{A \mid A \in \vectspace{M}_{22}, BAC = CAB\}$, $B$ and $C$ fixed $2 \times 2$ matrices

\end{enumerate}
\begin{sol}
\begin{enumerate}[label={\alph*.}]
\setcounter{enumi}{1}
\item  Yes.

\setcounter{enumi}{3}
\item  Yes.

\setcounter{enumi}{5}
\item  No; not closed under addition.

\end{enumerate}
\end{sol}
\end{ex}

\begin{ex}
Which of the following are subspaces of $\vectspace{F}[0, 1]$? Support your answer.

\begin{enumerate}[label={\alph*.}]
\item $U = \{f \mid f(0) = 0\}$

\item $U = \{f \mid f(0) = 1\}$

\item $U = \{f \mid f(0) = f(1)\}$

\item $U = \{f \mid f(x) \geq 0 \mbox{ for all } x \mbox{ in } [0, 1]\}$

\item $U = \{f \mid f(x) = f(y) \mbox{ for all } x \mbox{ and } y \mbox{ in } [0, 1]\}$

\item $U = \{f \mid f(x + y) = f(x) + f(y) \mbox{ for all } \\ x \mbox{ and } y \mbox{ in } [0, 1]\}$

\item $U = \{f \mid f \mbox{ is integrable and } \int_{0}^{1} f(x)dx = 0\}$
\end{enumerate}
\begin{sol}
\begin{enumerate}[label={\alph*.}]
\setcounter{enumi}{1}
\item  No; not closed under addition.

\setcounter{enumi}{3}
\item  No; not closed under scalar multiplication.

\setcounter{enumi}{5}
\item  Yes.

\end{enumerate}
\end{sol}
\end{ex}

\begin{ex}
Let $A$ be an $m \times n$ matrix. For which columns $\vect{b}$ in $\RR^m$ is $U = \{\vect{x} \mid \vect{x} \in \RR^n, A\vect{x} = \vect{b}\}$ a subspace of $\RR^n$? Support your answer.
\end{ex}

\begin{ex}
Let $\vect{x}$ be a vector in $\RR^n$ (written as a column), and define $U = \{A\vect{x} \mid A \in \vectspace{M}_{mn}\}$.

\begin{enumerate}[label={\alph*.}]
\item Show that $U$ is a subspace of $\RR^m$.

\item Show that $U = \RR^m$ if $\vect{x} \neq \vect{0}$.

\end{enumerate}
\begin{sol}
\begin{enumerate}[label={\alph*.}]
\setcounter{enumi}{1}
\item  If entry $k$ of $\vect{x}$ is $x_{k} \neq 0$, and if $\vect{y}$ is in $\RR^n$, then $\vect{y} = A\vect{x}$ where the column of $A$ is $x_{k}^{-1}\vect{y}$, and the other columns are zero.

\end{enumerate}
\end{sol}
\end{ex}

\begin{ex}
Write each of the following as a linear combination of $x + 1$, $x^{2} + x$, and $x^{2} + 2$.

\begin{exenumerate}
\exitem $x^{2} + 3x + 2$
\exitem $2x^{2} - 3x + 1$
\exitem $x^{2} + 1$
\exitem $x$
\end{exenumerate}
\begin{sol}
\begin{enumerate}[label={\alph*.}]
\setcounter{enumi}{1}
\item  $-3(x + 1) + 0(x^{2} + x) + 2(x^{2} + 2)$

\setcounter{enumi}{3}
\item  $\frac{2}{3}(x + 1) + \frac{1}{3}(x^{2} + x) - \frac{1}{3}(x^{2} + 2)$

\end{enumerate}
\end{sol}
\end{ex}

\begin{ex}
Determine whether $\vect{v}$ lies in $\func{span}\{\vect{u}, \vect{w}\}$ in each case.

\begin{enumerate}[label={\alph*.}]
\item $\vect{v} = 3x^{2} - 2x - 1$; $\vect{u} = x^{2} + 1$, $\vect{w} = x + 2$

\item $\vect{v} = x$; $\vect{u} = x^{2} + 1$, $\vect{w} = x + 2$

\item $\vect{v} =
\leftB \begin{array}{rr}
1 & 3 \\
-1 & 1
\end{array} \rightB$;  $\vect{u} = 
\leftB \begin{array}{rr}
1 & -1 \\
2 & 1
\end{array} \rightB$, $\vect{w} = 
\leftB \begin{array}{rr}
2 & 1 \\
1 & 0
\end{array} \rightB$

\item $\vect{v} =
\leftB \begin{array}{rr}
1 & -4 \\
5 & 3
\end{array} \rightB$; $\vect{u} = 
\leftB \begin{array}{rr}
1 & -1 \\
2 & 1
\end{array} \rightB$, $\vect{w} = 
\leftB \begin{array}{rr}
2 & 1 \\
1 & 0
\end{array} \rightB$

\end{enumerate}
\begin{sol}
\begin{enumerate}[label={\alph*.}]
\setcounter{enumi}{1}
\item  No.

\setcounter{enumi}{3}
\item  Yes; $\vect{v} = 3\vect{u} - \vect{w}$.

\end{enumerate}
\end{sol}
\end{ex}

\begin{ex}
Which of the following functions lie in $\func{span}\{\cos^{2}x, \sin^{2} x\}$? (Work in $\vectspace{F}[0, \pi]$.)

\begin{exenumerate}
\exitem $\cos 2x$
\exitem $1$
\exitem $x^{2}$
\exitem $1 + x^{2}$
\end{exenumerate}
\begin{sol}
\begin{enumerate}[label={\alph*.}]
\setcounter{enumi}{1}
\item  Yes; $1 = \cos^{2} x + \sin^{2} x$

\setcounter{enumi}{3}
\item  No. If $1 + x^{2} = a \cos^{2} x + b \sin^{2} x$, then taking $x = 0$ and $x = \pi$ gives $a = 1$ and $a = 1 + \pi^{2}$.

\end{enumerate}
\end{sol}
\end{ex}

\begin{ex}
\begin{enumerate}[label={\alph*.}]
\item Show that $\RR^3$ is spanned by \\$\{(1, 0, 1), (1, 1, 0), (0, 1, 1)\}$.

\item Show that $\vectspace{P}_{2}$ is spanned by $\{1 + 2x^{2}, 3x, 1 + x\}$.

\item Show that $\vectspace{M}_{22}$ is spanned by \\
\hspace*{-1em}$
\left\{
\leftB \begin{array}{rr}
1 & 0 \\
0 &  0
\end{array} \rightB
, 
\leftB \begin{array}{rr}
1 & 0 \\
0 & 1
\end{array} \rightB
, 
\leftB \begin{array}{rr}
0 & 1 \\
1 & 0
\end{array} \rightB
, 
\leftB \begin{array}{rrr}
1 & 1 \\
0 & 1
\end{array} \rightB
\right\}
$.
\end{enumerate}
\begin{sol}
\begin{enumerate}[label={\alph*.}]
\setcounter{enumi}{1}
\item  Because $\vectspace{P}_{2} = \func{span}\{1, x, x^{2}\}$, it suffices to show that $\{1, x, x^{2}\} \subseteq$
 $\func{span}\{1 + 2x^{2}, 3x, 1 + x\}$. But $x = \frac{1}{3}(3x); 1 = (1 + x) - x$ and $x^{2} = \frac{1}{2}[(1 + 2x^{2}) - 1]$.

\end{enumerate}
\end{sol}
\end{ex}

\begin{ex}
If $X$ and $Y$ are two sets of vectors in a vector space $V$, and if $X \subseteq Y$, show that \newline $\func{span} X \subseteq \func{span} Y$.
\end{ex}

\begin{ex}
Let $\vect{u}$, $\vect{v}$, and $\vect{w}$ denote vectors in a vector space $V$. Show that:

\begin{enumerate}[label={\alph*.}]
\item $\func{span}\{\vect{u}, \vect{v}, \vect{w}\} = \func{span}\{\vect{u} + \vect{v}, \vect{u} + \vect{w}, \vect{v} + \vect{w}\}$

\item $\func{span}\{\vect{u}, \vect{v}, \vect{w}\} = \func{span}\{\vect{u} - \vect{v}, \vect{u} + \vect{w}, \vect{w}\}$

\end{enumerate}
\begin{sol}
\begin{enumerate}[label={\alph*.}]
\setcounter{enumi}{1}
\item  $\vect{u} = (\vect{u} + \vect{w}) - \vect{w}$, $\vect{v} = -(\vect{u} - \vect{v}) + (\vect{u} + \vect{w}) - \vect{w}$, and $\vect{w} = \vect{w}$

\end{enumerate}
\end{sol}
\end{ex}

\begin{ex}
Show that 
\begin{equation*}
\func{span}\{\vect{v}_{1}, \vect{v}_{2}, \dots, \vect{v}_{n}, \vect{0}\} = \func{span}\{\vect{v}_{1}, \vect{v}_{2}, \dots, \vect{v}_{n}\}
\end{equation*}
holds for any set of vectors $\{\vect{v}_{1}, \vect{v}_{2}, \dots, \vect{v}_{n}\}$.
\end{ex}

\begin{ex}
If $X$ and $Y$ are nonempty subsets of a vector space $V$ such that $\func{span} X = \func{span} Y = V$, must there be a vector common to both $X$ and $Y$? Justify your answer.
\end{ex}

\begin{ex}
Is it possible that $\{(1, 2, 0), (1, 1, 1)\}$ can span the subspace $U = \{(a, b, 0) \mid a \mbox{ and } b \mbox{ in } \RR\}$?

\begin{sol}
No. 
\end{sol}
\end{ex}

\begin{ex}
Describe $\func{span}\{\vect{0}\}$.
\end{ex}

\begin{ex}
Let $\vect{v}$ denote any vector in a vector space $V$. Show that $\func{span}\{\vect{v}\} = \func{span}\{a\vect{v}\}$ for any $a \neq 0$.
\end{ex}

\begin{ex}
Determine all subspaces of $\RR\vect{v}$ where $\vect{v} \neq \vect{0}$ in some vector space $V$.

\begin{sol}
\begin{enumerate}[label={\alph*.}]
\setcounter{enumi}{1}
\item  Yes.

\end{enumerate}
\end{sol}
\end{ex}

\begin{ex}
Suppose $V = \func{span}\{\vect{v}_{1}, \vect{v}_{2}, \dots, \vect{v}_{n}\}$. If $\vect{u} = a_{1}\vect{v}_{1} + a_{2}\vect{v}_{2} + \dots+ a_{n}\vect{v}_{n}$ where the $a_{i}$ are in $\RR$ and $a_{1} \neq 0$, show that $V = \func{span}\{\vect{u}, \vect{v}_{2}, \dots, \vect{v}_{n}\}$.

\begin{sol}
$\vect{v}_1 = \frac{1}{a_1}\vect{u} - \frac{a_2}{a_1}\vect{v}_2 - \dots - \frac{a_n}{a_1}\vect{v}_n$, so $V \subseteq \func{span}\{\vect{u}, \vect{v}_2, \dots, \vect{v}_n \}$
\end{sol}
\end{ex}

\begin{ex}
If $\vectspace{M}_{nn} = \func{span}\{A_1, A_2, \dots, A_k\}$, show that $\vectspace{M}_{nn} = \func{span}\{A_1^T, A_2^T, \dots, A_k^T\}$.
\end{ex}

\begin{ex}
If $\vectspace{P}_{n} = \func{span}\{p_{1}(x), p_{2}(x), \dots, p_{k}(x)\}$ and $a$ is in $\RR$, show that $p_{i}(a) \neq 0$ for some $i$.
\end{ex}

\begin{ex}
Let $U$ be a subspace of a vector space $V$.

\begin{enumerate}[label={\alph*.}]
\item If $a\vect{u}$ is in $U$ where $a \neq 0$, show that $\vect{u}$ is in $U$.

\item If $\vect{u}$ and $\vect{u} + \vect{v}$ are in $U$, show that $\vect{v}$ is in $U$.

\end{enumerate}
\begin{sol}
\begin{enumerate}[label={\alph*.}]
\setcounter{enumi}{1}
\item  $\vect{v} = (\vect{u} + \vect{v}) - \vect{u}$ is in $U$.

\end{enumerate}
\end{sol}
\end{ex}

\begin{ex}
Let $U$ be a nonempty subset of a vector space $V$. Show that $U$ is a subspace of $V$ if and only if $\vect{u}_{1} + a\vect{u}_{2}$ lies in $U$ for all $\vect{u}_{1}$ and $\vect{u}_{2}$ in $U$ and all $a$ in $\RR$.

\begin{sol}
Given the condition and $\vect{u} \in U$, $\vect{0} = \vect{u} + (-1)\vect{u} \in U$. The converse holds by the subspace test.
\end{sol}
\end{ex}

\begin{ex}
Let $U = \{p(x) \mbox{ in } \vectspace{P} \mid p(3) = 0\}$ be the set in
Example~\ref{exa:018124}. Use the factor theorem (see
Section~\ref{sec:6_5}) to show that $U$ consists of multiples of $x - 3$; that is, show that $U = \{(x - 3)q(x) \mid q(x) \in \vectspace{P}\}$. Use this to show that $U$ is a subspace of $\vectspace{P}$.
\end{ex}

\begin{ex}
Let $A_{1}, A_{2}, \dots, A_{m}$ denote $n \times n$ matrices. If $\vect{0} \neq \vect{y} \in \RR^n$ and $A_{1}\vect{y} = A_{2}\vect{y} = \dots = A_{m}\vect{y} = \vect{0}$, show that $\{A_{1}, A_{2}, \dots, A_{m}\}$ cannot $\func{span} \vectspace{M}_{nn}$.
\end{ex}

\begin{ex}
Let $\{\vect{v}_{1}, \vect{v}_{2}, \dots, \vect{v}_{n}\}$ and $\{\vect{u}_{1}, \vect{u}_{2}, \dots, \vect{u}_{n}\}$ be sets of vectors in a vector space, and let
\begin{equation*}
X = 
\leftB \begin{array}{c}
\vect{v}_1 \\
\vdots \\
\vect{v}_n
\end{array} \rightB
\ Y =
\leftB \begin{array}{c}
\vect{u}_1 \\
\vdots \\
\vect{u}_n
\end{array} \rightB
\end{equation*}
as in Exercise~\ref{ex:6_1_18}.

\begin{enumerate}[label={\alph*.}]
\item Show that $\func{span}\{\vect{v}_{1}, \dots, \vect{v}_{n}\} \subseteq \func{span}\{\vect{u}_{1}, \dots, \vect{u}_{n}\}$ if and only if $AY = X$ for some $n \times n$ matrix $A$.

\item If $X = AY$ where $A$ is invertible, show that $\func{span}\{\vect{v}_{1}, \dots, \vect{v}_{n}\} = \func{span}\{\vect{u}_{1}, \dots, \vect{u}_{n}\}$.

\end{enumerate}
\end{ex}

\begin{ex}
If $U$ and $W$ are subspaces of a vector space $V$, let 
$U \cup W = \{\vect{v} \mid \vect{v}$ is in $U$ or $\vect{v}$ is in $W\}$. Show that $U \cup W$ is a subspace if and only if $U \subseteq W$ or $W \subseteq U$.
\end{ex}

\begin{ex}
Show that $\vectspace{P}$ cannot be spanned by a finite set of polynomials.
\end{ex}
\end{multicols}

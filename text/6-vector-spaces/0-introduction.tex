\noindent In this chapter we introduce vector spaces in full generality. The reader will notice some similarity with the discussion of the space $\RR^n$ in Chapter \ref{chap:5}. In fact much of the present material has been developed in that context, and there is some repetition. However, Chapter \ref{chap:6} deals with the notion of an \textit{abstract}\index{vector spaces!abstract}\index{abstract vector space} vector space, a concept that will be new to most readers. It turns out that there are many systems in which a natural addition and scalar multiplication are defined and satisfy the usual rules familiar from $\RR^n$. The study of abstract vector spaces is a way to deal with all these examples simultaneously. The new aspect is that we are dealing with an abstract system in which \textit{all we know} about the vectors is that they are objects that can be added and multiplied by a scalar and satisfy rules familiar from $\RR^n$.\index{vector spaces!introduction of concept}

The novel thing is the \textit{abstraction}. Getting used to this new conceptual level is facilitated by the work done in Chapter \ref{chap:5}: First, the vector manipulations are familiar, giving the reader more time to become accustomed to the abstract setting; and, second, the mental images developed in the concrete setting of $\RR^n$ serve as an aid to doing many of the exercises in Chapter \ref{chap:6}.

The concept of a vector space was first introduced in 1844 by the German mathematician Hermann Grassmann (1809-1877), but his work did not receive the attention it deserved. It was not until 1888 that the Italian mathematician Guiseppe Peano (1858-1932) clarified Grassmann's work in his book \textit{Calcolo Geometrico} and gave the vector space axioms in their present form. Vector spaces became established with the work of the Polish mathematician Stephan Banach (1892-1945), and the idea was finally accepted in 1918 when Hermann Weyl (1885-1955) used it in his widely read book \textit{Raum-Zeit-Materie} (``Space-Time-Matter''), an introduction to the general theory of relativity. \index{Banach, Stephan}\index{\textit{Calcolo Geometrico} (Peano)}\index{general theory of relativity}\index{Grassmann, Hermann}\index{Peano, Guiseppe}\index{\textit{Raum-Zeit-Materie} (``Space-Time-Matter'')(Weyl)}\index{Weyl, Hermann}

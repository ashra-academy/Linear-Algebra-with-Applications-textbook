\chapter*{Preface}
\label{chap:preface}

This textbook is an introduction to the
 ideas and techniques of linear algebra for first- or second-year 
students with a working knowledge of high school algebra. The contents 
have enough flexibility to present a traditional introduction to the 
subject, or to allow for a more applied course. Chapters~\ref{chap:1}--\ref{chap:4} contain a one-semester course for beginners whereas Chapters~\ref{chap:5}--\ref{chap:9}
 contain a second semester course (see the Suggested Course Outlines 
below). The text is primarily about real linear algebra with complex 
numbers being mentioned when appropriate (reviewed in Appendix \ref{chap:appacomplexnumbers}).
 Overall, the aim of the text is to achieve a balance among 
computational skills, theory, and applications of linear algebra. 
Calculus is not a prerequisite; places where it is mentioned may be 
omitted.


As a rule, students of linear algebra 
learn by studying examples and solving problems. Accordingly, the book 
contains a variety of exercises (over 1200, many with multiple parts), 
ordered as to their difficulty. In addition, more than 375 solved 
examples are included in the text, many of which are computational in 
nature. The examples are also used to motivate (and illustrate) concepts
 and theorems, carrying the student from concrete to abstract. While the
 treatment is rigorous, proofs are presented at a level appropriate to 
the student and may be omitted with no loss of continuity. As a result, 
the book can be used to give a course that emphasizes computation and 
examples, or to give a more theoretical treatment (some longer proofs 
are deferred to the end of the Section).


Linear Algebra has application to the 
natural sciences, engineering, management, and the social sciences as 
well as mathematics. Consequently, 18 optional ``applications'' sections 
are included in the text introducing topics as diverse as electrical 
networks, economic models, Markov chains, linear recurrences, systems of
 differential equations, and linear codes over finite fields. 
Additionally some applications (for example linear dynamical systems, 
and directed graphs) are introduced in context. The applications 
sections appear at the end of the relevant chapters to encourage 
students to browse.


\section*{SUGGESTED COURSE OUTLINES}

This text includes the basis for a two-semester course in linear algebra.


\begin{itemize}
\item Chapters~\ref{chap:1}--\ref{chap:4}
 provide a standard one-semester course of 35 lectures, including linear
 equations, matrix algebra, determinants, diagonalization, and geometric
 vectors, with applications as time permits. At Calgary, we cover Sections \ref{sec:1_1}--\ref{sec:1_3}, \ref{sec:2_1}--\ref{sec:2_6}, \ref{sec:3_1}--\ref{sec:3_3}, and \ref{sec:4_1}--\ref{sec:4_4} and the course is taken by all science and engineering students in 
their first semester. Prerequisites include a working knowledge of high 
school algebra (algebraic manipulations and some familiarity with 
polynomials); calculus is not required.

\item Chapters~\ref{chap:5}--\ref{chap:9} contain a second semester course including $\RR^n$,
 abstract vector spaces, linear transformations (and their matrices), 
orthogonality, complex matrices (up to the spectral theorem) and 
applications. There is more material here than can be covered in one 
semester, and at Calgary we cover Sections \ref{sec:5_1}--\ref{sec:5_5}, \ref{sec:6_1}--\ref{sec:6_4}, \ref{sec:7_1}--\ref{sec:7_3}, \ref{sec:8_1}--\ref{sec:8_6}, and \ref{sec:9_1}--\ref{sec:9_3} with a couple of applications as time permits.

\item Chapter~\ref{chap:5} is a ``bridging'' chapter that introduces concepts like spanning, independence, and basis in the concrete setting of $\RR^n$, before venturing into the abstract in Chapter~\ref{chap:6}. The duplication is balanced by the value of reviewing these notions, and it enables the student to focus in Chapter~\ref{chap:6} on the new idea of an abstract system. Moreover, Chapter~\ref{chap:5}
 completes the discussion of rank and diagonalization from earlier 
chapters, and includes a brief introduction to orthogonality in $\RR^n$, which creates the possibility of a one-semester, matrix-oriented course covering Chapter~\ref{chap:1}--\ref{chap:5} for students not wanting to study the abstract theory.

\end{itemize}

\section*{CHAPTER DEPENDENCIES}

The following chart suggests how the 
material introduced in each chapter draws on concepts covered in certain
 earlier chapters. A solid arrow means that ready assimilation of ideas 
and techniques presented in the later chapter depends on familiarity 
with the earlier chapter. A broken arrow indicates that some reference 
to the earlier chapter is made but the chapter need not be covered.

\begin{figure}[H]
\centering
\tikzstyle{block} = [rectangle, draw=ltgreenvect!50, fill=ltgreenvect!50, text centered, rounded corners, font=\scriptsize]
\tikzstyle{arrow}=[draw,-latex,thick,dkbluevect]
\tikzstyle{dsharrow}=[draw, dashed, -latex,thick,dkbluevect]

\begin{tikzpicture}[scale=0.4]
\node[block](1) {Chapter 1: Systems of Linear Equations};
\node[block, below of=1](2) {Chapter 2: Matrix Algebra};
\node[below of=2](null){};
\node[block, left of=null, node distance=3cm] (3) {Chapter 3: Determinants and Diagonalization};
\node[block, right of=null, node distance=3cm] (4) {Chapter 4: Vector Geometry};
\node[block, below of=null] (5) {Chapter 5: The Vector Space $\RR^n$};
\node[block, below of=5] (6) {Chapter 6: Vector Spaces};
\node[below of=6](null2){};
\node[block, left of=null2, node distance=3cm] (7) {Chapter 7: Linear Transformations};
\node[block, right of=null2,node distance=3cm] (8) {Chapter 8: Orthogonality};
\node[block, below of=null2] (9) {Chapter 9: Change of Basis};
\node[below of=9] (null3) {};
\node[block, left of=null3, node distance=3cm] (10) {Chapter 10: Inner Product Spaces};
\node[block, right of=null3, node distance=3cm] (11) {Chapter 11: Canonical Forms};


\path[arrow] (1)--(2);
\path[arrow] (2)--(3);
\path[arrow] (2)--(4);
\path[dsharrow] (3)--(4);
\path[arrow] (3)--(5);
\path[arrow] (4)--(5);
\path[arrow] (5)--(6);
\path[arrow] (6)--(7);
\path[arrow] (6)--(8);
\path[dsharrow] (7)--(8);
\path[arrow] (7)--(9);
\path[arrow] (8)--(9);
\path[arrow] (9)--(10);
\path[arrow] (9)--(11);


\end{tikzpicture}

\end{figure}


%% \section*{NEW IN THE SEVENTH EDITION}

%% \begin{itemize}
%% \item \textbf{Vector notation}.
%%  Based on feedback from reviewers and current users, all vectors are 
%% denoted by boldface letters (used only in abstract spaces in earlier 
%% editions). Thus \textit{x} becomes \textbf{x} in $\RR$\textsuperscript{2} and $\RR$\textsuperscript{3} (Chapter~\ref{chap:4}), and in $\RR$\textit{\textsuperscript{n}} the column \textit{X} becomes \textbf{x}. Furthermore, the notation [\textit{x}\textsubscript{1} \textit{x}\textsubscript{2} \dots  \textit{x\textsubscript{n}}]\textit{\textsuperscript{T}} for vectors in $\RR$\textit{\textsuperscript{n}} has been eliminated; instead we write vectors as \textit{n}-tuples (\textit{x}\textsubscript{1}, \textit{x}\textsubscript{2}, \dots , \textit{x\textsubscript{n}}) or as columns $\leftB \begin{array}{c}
%% x_1 \\
%% x_2 \\
%% \vdots \\
%% x_n 
%% \end{array}\rightB$.
%%  The result is a uniform notation for vectors throughout the text.

%% \item \textbf{Definitions.}
%%  Important ideas and concepts are identified in their given context for 
%% student's understanding. These are highlighted in the text when they are
%%  first discussed, identified in the left margin, and listed on the 
%% inside back cover for reference.

%% \item \textbf{Exposition}.
%%  Several new margin diagrams have been included to clarify concepts, and
%%  the exposition has been improved to simplify and streamline discussion 
%% and proofs.

%% \end{itemize}

%% \section*{OTHER CHANGES}

%% \begin{itemize}
%% \item Several new examples and exercises have been added.

%% \item The motivation for the matrix inversion algorithm has been rewritten in Section~\ref{sec:2_4}.

%% \item For geometric vectors in $\RR$\textsuperscript{2}, addition (parallelogram law) and scalar multiplication now appear earlier (Section~\ref{sec:2_2}). The discussion of reflections in Section~\ref{sec:2_6} has been simplified, and projections are now included.

%% \item The example in Section~\ref{sec:3_3}, which illustrates that \textbf{x} in $\RR$\textsuperscript{2} is an eigenvector of \textit{A} if, and only if, the line $\RR$\textit{\textsubscript{x}} is \textit{A}-invariant, has been completely rewritten.

%% \item The first part of Section~\ref{sec:4_1} on vector geometry in $\RR$\textsuperscript{2} and $\RR$\textsuperscript{3} has also been rewritten and shortened.

%% \item In Section~\ref{sec:6_4} there are three improvements: Theorem \ref{thm:019430} now shows that an independent set can be extended to a basis by adding vectors from \textit{any prescribed basis}; the proof that a spanning set can be cut down to a basis has been simplified (in Theorem \ref{thm:019593}); and in Theorem \ref{thm:019633}, the argument that independence is equivalent to spanning for a set \textit{S} $\subseteq$ \textit{V} with |\textit{S}| = dim \textit{V} has been streamlined and a new example added.

%% \item In Section~\ref{sec:8_1}, the definition of projections has been clarified, as has the discussion of the nature of quadratic forms in $\RR$\textsuperscript{2}.

%% \end{itemize}
\vspace*{-2em}
\section*{HIGHLIGHTS OF THE TEXT}

\begin{itemize}
\item \textbf{Two-stage definition of matrix multiplication.} First, in Section~\ref{sec:2_2}
 matrix-vector products are introduced naturally by viewing the left 
side of a system of linear equations as a product. Second, matrix-matrix
 products are defined in Section~\ref{sec:2_3} by taking the columns of a product $AB$ to be $A$ times the corresponding columns of $B$.
 This is motivated by viewing the matrix product as composition of maps 
(see next item). This works well pedagogically and the usual dot-product
 definition follows easily. As a bonus, the proof of associativity of 
matrix multiplication now takes four lines.

\item \textbf{Matrices as transformations.} Matrix-column multiplications are viewed (in Section~\ref{sec:2_2}) as transformations $\RR^n \to  \RR^m$. These maps are then used to describe simple geometric reflections and rotations in $\RR^2$ as well as systems of linear equations.

\item \textbf{Early linear transformations.}
 It has been said that vector spaces exist so that linear 
transformations can act on them---consequently these maps are a recurring 
theme in the text. Motivated by the matrix transformations introduced 
earlier, linear transformations $\RR^n \to  \RR^m$ are defined in Section~\ref{sec:2_6},
 their standard matrices are derived, and they are then used to describe
 rotations, reflections, projections, and other operators on $\RR^2$.

\item \textbf{Early diagonalization.}
 As requested by engineers and scientists, this important technique is 
presented in the first term using only determinants and matrix inverses 
(before defining independence and dimension). Applications to population
 growth and linear recurrences are given.

\item \textbf{Early dynamical systems.} These are introduced in Chapter~\ref{chap:3},
 and lead (via diagonalization) to applications like the possible 
extinction of species. Beginning students in science and engineering can
 relate to this because they can see (often for the first time) the 
relevance of the subject to the real world.

\item \textbf{Bridging chapter.} Chapter~\ref{chap:5} lets students deal with tough concepts (like independence, spanning, and basis) in the concrete setting of $\RR^n$ before having to cope with abstract vector spaces in Chapter~\ref{chap:6}.

\item \textbf{Examples.}
 The text contains over 375 worked examples, which present the main 
techniques of the subject, illustrate the central ideas, and are keyed 
to the exercises in each section.

\item \textbf{Exercises.}
 The text contains a variety of exercises (nearly 1175, many with 
multiple parts), starting with computational problems and gradually 
progressing to more theoretical exercises. Select solutions are available at the end of the book or in the Student Solution Manual. There is a complete Solution Manual is 
available for instructors.

\item \textbf{Applications}.
 There are optional applications at the end of most chapters (see the 
list below). While some are presented in the course of the text, most 
appear at the end of the relevant chapter to encourage students to 
browse.

\item \textbf{Appendices.} Because complex numbers are needed in the text, they are described in Appendix \ref{chap:appacomplexnumbers}, which includes the polar form and roots of unity. Methods of proofs are discussed in Appendix \ref{chap:appbproofs}, followed by mathematical induction in Appendix \ref{chap:appcinduction}. A brief discussion of polynomials is included in Appendix \ref{chap:appdpolynomials}. All these topics are presented at the high-school level.

\item \textbf{Self-Study.} This text is self-contained and therefore is suitable for self-study.

\item \textbf{Rigour.}
 Proofs are presented as clearly as possible (some at the end of the 
section), but they are optional and the instructor can choose how much 
he or she wants to prove. However the proofs are there, so this text is 
more rigorous than most. Linear algebra provides one of the better 
venues where students begin to think logically and argue concisely. To 
this end, there are exercises that ask the student to ``show'' some simple
 implication, and others that ask her or him to either prove a given 
statement or give a counterexample. I personally present a few proofs in
 the first semester course and more in the second (see the Suggested 
Course Outlines).

\item \textbf{Major Theorems.}
 Several major results are presented in the book. Examples: Uniqueness 
of the reduced row-echelon form; the cofactor expansion for 
determinants; the Cayley-Hamilton theorem; the Jordan canonical form; 
Schur's theorem on block triangular form; the principal axes and 
spectral theorems; and others. Proofs are included because the stronger 
students should at least be aware of what is involved.

\end{itemize}

\section*{CHAPTER SUMMARIES}


\subsection*{Chapter 1: Systems of Linear Equations.}


A standard treatment of gaussian 
elimination is given. The rank of a matrix is introduced via the 
row-echelon form, and solutions to a homogeneous system are presented as 
linear combinations of basic solutions. Applications to network flows, 
electrical networks, and chemical reactions are provided.



\subsection*{Chapter 2: Matrix Algebra.}


After a traditional look at matrix addition, scalar multiplication, and transposition in Section~\ref{sec:2_1}, matrix-vector multiplication is introduced in Section~\ref{sec:2_2} by viewing the left side of a system of linear equations as the product $A\vect{x}$ of the coefficient matrix $A$ with the column $\vect{x}$ of variables. The usual dot-product definition of a matrix-vector multiplication follows. Section~\ref{sec:2_2} ends by viewing an $m \times n$ matrix $A$ as a transformation $\RR^n \to \RR^m$. This is illustrated for $\RR^2 \to \RR^2$ by describing reflection in the $x$ axis, rotation of $\RR^2$ through $\frac{\pi}{2}$, shears, and so on.


In Section~\ref{sec:2_3}, the product of matrices $A$ and $B$ is defined by $AB = \leftB \begin{array}{cccc} A\vect{b}_1 & A\vect{b}_2 & \cdots & A\vect{b}_n \end{array}\rightB$, where the $\vect{b}_i$ are the columns of $B$. A routine computation shows that this is the matrix of the transformation $B$ followed by $A$.
 This observation is used frequently throughout the book, and leads to 
simple, conceptual proofs of the basic axioms of matrix algebra. Note 
that linearity is not required---all that is needed is some basic 
properties of matrix-vector multiplication developed in Section~\ref{sec:2_2}.
 Thus the usual arcane definition of matrix multiplication is split into
 two well motivated parts, each an important aspect of matrix algebra. 
Of course, this has the pedagogical advantage that the conceptual power 
of geometry can be invoked to illuminate and clarify algebraic 
techniques and definitions.


In Section \ref{sec:2_4} and \ref{sec:2_5}
 matrix inverses are characterized, their geometrical meaning is 
explored, and block multiplication is introduced, emphasizing those 
cases needed later in the book. Elementary matrices are discussed, and 
the Smith normal form is derived. Then in Section~\ref{sec:2_6}, linear transformations $\RR^n \to \RR^m$ are defined and shown to be matrix transformations. The matrices of reflections, rotations, and projections in the plane are determined. Finally, matrix multiplication is related to directed graphs, matrix 
LU-factorization is introduced, and applications to economic models and 
Markov chains are presented.



\subsection*{Chapter 3: Determinants and Diagonalization.}


The cofactor expansion is stated 
(proved by induction later) and used to define determinants inductively 
and to deduce the basic rules. The product and adjugate theorems are 
proved. Then the diagonalization algorithm is presented (motivated by an
 example about the possible extinction of a species of birds). As 
requested by our Engineering Faculty, this is done earlier than in most 
texts because it requires only determinants and matrix inverses, 
avoiding any need for subspaces, independence and dimension. 
Eigenvectors of a $2 \times 2$ matrix $A$ are described geometrically (using the $A$-invariance
 of lines through the origin). Diagonalization is then used to study 
discrete linear dynamical systems and to discuss applications to linear 
recurrences and systems of differential equations. A brief discussion of
 Google PageRank is included.



\subsection*{Chapter 4: Vector Geometry.}


Vectors are presented intrinsically in 
terms of length and direction, and are related to matrices via 
coordinates. Then vector operations are defined using matrices and shown
 to be the same as the corresponding intrinsic definitions. Next, dot 
products and projections are introduced to solve problems about lines 
and planes. This leads to the cross product. Then matrix transformations
 are introduced in $\RR^3$,
 matrices of projections and reflections are derived, and areas and 
volumes are computed using determinants. The chapter closes with an 
application to computer graphics.



\subsection*{Chapter 5: The Vector Space $\RR^n$.}


Subspaces, spanning, independence, and dimensions are introduced in the context of $\RR^n$
 in the first two sections. Orthogonal bases are introduced and used to 
derive the expansion theorem. The basic properties of rank are presented
 and used to justify the definition given in Section~\ref{sec:1_2}.
 Then, after a rigorous study of diagonalization, best approximation and
 least squares are discussed. The chapter closes with an application to 
correlation and variance.

This is a ``bridging'' chapter, easing the transition to abstract spaces. Concern about duplication with Chapter~\ref{chap:6}
 is mitigated by the fact that this is the most difficult part of the 
course and many students welcome a repeat discussion of concepts like 
independence and spanning, albeit in the abstract setting. In a 
different direction, Chapter~\ref{chap:1}--\ref{chap:5} could serve as a solid introduction to linear algebra for students not requiring abstract theory.



\subsection*{Chapter 6: Vector Spaces.}


Building on the work on $\RR^n$ in Chapter~\ref{chap:5},
 the basic theory of abstract finite dimensional vector spaces is 
developed emphasizing new examples like matrices, polynomials and 
functions. This is the first acquaintance most students have had with an
 abstract system, so not having to deal with spanning, independence and 
dimension in the general context eases the transition to abstract 
thinking. Applications to polynomials and to differential equations are 
included.



\subsection*{Chapter 7: Linear Transformations.}


General linear transformations are 
introduced, motivated by many examples from geometry, matrix theory, and
 calculus. Then kernels and images are defined, the dimension theorem is
 proved, and isomorphisms are discussed. The chapter ends with an 
application to linear recurrences. A proof is included that the order of
 a differential equation (with constant coefficients) equals the 
dimension of the space of solutions.



\subsection*{Chapter 8: Orthogonality.}


The study of orthogonality in $\RR^n$, begun in Chapter~\ref{chap:5},
 is continued. Orthogonal complements and projections are defined and 
used to study orthogonal diagonalization. This leads to the principal 
axes theorem, the Cholesky factorization of a positive definite matrix, QR-factorization, and to a discussion of the singular value decomposition, the polar form, and the pseudoinverse. The theory is extended to $\mathbb{C}^n$ in Section~\ref{sec:8_6}
 where hermitian and unitary matrices are discussed, culminating in 
Schur's theorem and the spectral theorem. A short proof of the 
Cayley-Hamilton theorem is also presented. In Section~\ref{sec:8_7} the field $\mathbb{Z}_p$ of integers modulo $p$ is constructed informally for any prime $p$,
 and codes are discussed over any finite field. The chapter concludes 
with applications to quadratic forms, constrained optimization, and 
statistical principal component analysis.



\subsection*{Chapter 9: Change of Basis.}


The matrix of general linear 
transformation is defined and studied. In the case of an operator, the 
relationship between basis changes and similarity is revealed. This is 
illustrated by computing the matrix of a rotation about a line through 
the origin in $\RR^3$.
 Finally, invariant subspaces and direct sums are introduced, related to
 similarity, and (as an example) used to show that every involution is 
similar to a diagonal matrix with diagonal entries $\pm 1$.



\subsection*{Chapter 10: Inner Product Spaces.}


General inner products are introduced 
and distance, norms, and the Cauchy-Schwarz inequality are discussed. 
The Gram-Schmidt algorithm is presented, projections are defined and the
 approximation theorem is proved (with an application to Fourier 
approximation). Finally, isometries are characterized, and distance 
preserving operators are shown to be composites of a translations and 
isometries.


\vspace*{-0.5em}
\subsection*{Chapter 11: Canonical Forms.}


The work in Chapter~\ref{chap:9}
 is continued. Invariant subspaces and direct sums are used to derive 
the block triangular form. That, in turn, is used to give a compact 
proof of the Jordan canonical form. Of course the level is higher.



\subsection*{Appendices}


In Appendix \ref{chap:appacomplexnumbers}, complex arithmetic is developed far enough to find $n$th roots. In Appendix \ref{chap:appbproofs}, methods of proof are discussed, while Appendix \ref{chap:appcinduction} presents mathematical induction. Finally, Appendix \ref{chap:appdpolynomials} describes the properties of polynomials in elementary terms.




\section*{LIST OF APPLICATIONS}

\begin{itemize}
\item Network Flow (Section~\ref{sec:1_4})

\item Electrical Networks (Section~\ref{sec:1_5})

\item Chemical Reactions (Section~\ref{sec:1_6})

\item Directed Graphs (in Section~\ref{sec:2_3})

\item Input-Output Economic Models (Section~\ref{sec:2_8})

\item Markov Chains (Section~\ref{sec:2_9})

\item Polynomial Interpolation (in Section~\ref{sec:3_2})

\item Population Growth (Examples \ref{exa:008923} and \ref{exa:009389}, Section~\ref{sec:3_3})

\item Google PageRank (in Section~\ref{sec:3_3})

\item Linear Recurrences (Section~\ref{sec:3_4}; see also Section~\ref{sec:7_5})

\item Systems of Differential Equations (Section~\ref{sec:3_5})

\item Computer Graphics (Section~\ref{sec:4_5})

\item Least Squares Approximation (in Section~\ref{sec:5_6})

\item Correlation and Variance (Section~\ref{sec:5_7})

\item Polynomials (Section~\ref{sec:6_5})

\item Differential Equations (Section~\ref{sec:6_6})

\item Linear Recurrences (Section~\ref{sec:7_5})

\item Error Correcting Codes (Section~\ref{sec:8_7})

\item Quadratic Forms (Section~\ref{sec:8_8})

\item Constrained Optimization (Section~\ref{sec:8_9})

\item Statistical Principal Component Analysis (Section~\ref{sec:8_10})

\item Fourier Approximation (Section~\ref{sec:10_5})

\end{itemize}

\section*{ACKNOWLEDGMENTS}

Many colleagues have contributed to the development of this text over many years of publication, and I specially thank the following instructors for their reviews of the $7^{th}$ edition:


Robert Andre\\
\hspace*{3em}\textit{University of Waterloo}

Dietrich Burbulla\\
\hspace*{3em}\textit{University of Toronto}

Dzung M. Ha\\
\hspace*{3em}\textit{Ryerson University}

Mark Solomonovich\\
\hspace*{3em}\textit{Grant MacEwan}

Fred Szabo\\
\hspace*{3em}\textit{Concordia University}

Edward Wang\\
\hspace*{3em}\textit{Wilfred Laurier}

Petr Zizler\\
\hspace*{3em}\textit{Mount Royal University}

It is also a pleasure to recognize the contributions of several people. Discussions with Thi Dinh and Jean Springer have been invaluable and many of their suggestions have been incorporated. Thanks are also due to Kristine Bauer and Clifton Cunningham for several conversations about the new way to look at matrix multiplication. I also wish to extend my thanks to Joanne Canape for being there when I had technical questions. Thanks also go to Jason Nicholson for his help in various aspects of the book, particularly the Solutions Manual. Finally, I want to thank my wife Kathleen, without whose understanding and cooperation, this book would not exist.

As we undertake this new publishing model with the text as an open educational resource, I would also like to thank my previous publisher. The team who supported my text greatly contributed to its success. 

Now that the text has an open license, we have a much more fluid and powerful mechanism to incorporate comments and suggestions. The editorial group at Lyryx invites instructors and students to contribute to the text, and also offers to provide adaptations of the material for specific courses. Moreover the LaTeX source files are available to anyone wishing to do the adaptation and editorial work themselves!

W. Keith Nicholson \\
\hspace*{3em}\textit{University of Calgary}

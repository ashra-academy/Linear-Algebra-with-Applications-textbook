\chapter*{Foreward}
\label{chap:foreward}

Mathematics education at the beginning university level is closely tied to the traditional publishers. In my opinion, it gives them too much control of both cost and content. The main goal of most publishers is profit, and the result has been a sales-driven business model as opposed to a pedagogical one. This results in frequent new ``editions'' of textbooks motivated largely to reduce the sale of used books rather than to update content quality. It also introduces copyright restrictions which stifle the creation and use of new pedagogical methods and materials. The overall result is high cost textbooks which may not meet the evolving educational needs of instructors and students.

To be fair, publishers do try to produce material that reflects new trends. But their goal is to sell books and not necessarily to create tools for student success in mathematics education. Sadly, this has led to a model where the primary choice for adapting to (or initiating) curriculum change is to find a different commercial textbook. My editor once said that the text that is adopted is often everyone's third choice.

Of course instructors can produce their own lecture notes, and have done so for years,
but this remains an onerous task. The publishing industry arose from the need to provide
authors with copy-editing, editorial, and marketing services, as well as extensive reviews of prospective customers to ascertain market trends and content updates. These are necessary skills and services that the industry continues to offer. 

Authors of open educational resources (OER) including (but not limited to) textbooks
and lecture notes, cannot afford this on their own. But they do have two great advantages: The cost to students is significantly lower, and open licenses return content control to instructors. Through editable file formats and open licenses, OER can be developed, maintained, reviewed, edited, and improved by a variety of contributors. Instructors can now respond to curriculum change by revising and reordering material to create content that meets the needs of their students. While editorial and quality control remain daunting tasks, great strides have been made in addressing the issues of accessibility, affordability
and adaptability of the material.

For the above reasons I have decided to release my text under an open license, even though
it was published for many years through a traditional publisher.

Supporting students and instructors in a typical classroom requires much more than a
textbook. Thus, while anyone is welcome to use and adapt my text at no cost, I also
decided to work closely with Lyryx Learning. With colleagues at the University of Calgary,
I helped create Lyryx almost 20 years ago. The original idea was to develop quality online
assessment (with feedback) well beyond the multiple-choice style then available. Now Lyryx
also works to provide and sustain open textbooks; working with authors, contributors, and
reviewers to ensure instructors need not sacrifice quality and rigour when switching to an
open text.

I believe this is the right direction for mathematical publishing going forward, and look
forward to being a part of how this new approach develops.

\medskip

\noindent W. Keith Nicholson, Author \\
\noindent University of Calgary



\section{Projections and Planes}
\label{sec:4_2}

\begin{wrapfigure}[9]{l}{5cm} 
\centering
\begin{tikzpicture}[scale=0.9]
\coordinate (ptRight) at (4, 0.5);
\coordinate (ptBottom) at (2, -1);
\filldraw[color=dkbluevect, fill=ltbluevect, thick] (0, 0)--(2, 1.5)--(ptRight)--(ptBottom)--cycle;
\coordinate (ptP) at (1.5, 1.75);
\coordinate (ptQ) at (2, 0.25);

%line under the plane
\path[name path=lineRightEdge] (ptRight)--(ptBottom);
\path[name path=linePQ] (ptP)-- ++(1, -3);
\path[dkbluevect, thick, name intersections={of=lineRightEdge and linePQ, by=ptBelow}];
\draw[dkbluevect, thick] (ptBelow)-- +(0.25, -0.75);

%lines and right angles near Q
\draw[dkbluevect, thick] (ptQ)-- +(-0.5, 0.25); %left of Q, m = -0.5
\draw[dkbluevect, thick] (ptQ)-- +(0.5, 0.375); %right of Q, m = 0.75
\draw[dkbluevect, thick] (ptQ)--(ptP)-- +(-0.1, 0.3); %line PQ, m = -3
\draw[dkbluevect, thick] (ptQ)++(-0.179, 0.0895) -- ++(-0.063, 0.189) -- ++(0.179, -0.0895) -- ++(0.16, 0.12) -- ++(0.063, -0.189) ; %x^2 + (0.5x)^2 = 0.2^2 => x = 0.179

\fill[dkbluevect] (ptP) circle (2pt);
\fill[dkbluevect] (ptQ) circle (2pt);

\node[right] at (ptP) {\small $P$};
\node[right] at (ptQ) {\small $Q$};
\end{tikzpicture}

\caption{\label{fig:011780}}
\end{wrapfigure}

Any
 student of geometry soon realizes that the notion of perpendicular 
lines is fundamental. As an illustration, suppose a point $P$ and a plane are given and it is desired to find the point $Q$ that lies in the plane and is closest to $P$, as shown in Figure~\ref{fig:011780}. Clearly, what is required is to find the line through $P$ that is perpendicular to the plane and then to obtain $Q$ as the point of intersection of this line with the plane. Finding the line \textit{perpendicular}\index{line!perpendicular lines}\index{perpendicular lines} to the plane requires a way to determine when two vectors are perpendicular. This can be done using the idea of the dot product of two  vectors.\index{vector geometry!line perpendicular to plane}


\subsection*{The Dot Product and Angles}

\begin{definition}{Dot Product in $\RR^3$}{011783}
Given vectors
$\vect{v} = \leftB
\begin{array}{c}
x_{1} \\
y_{1}\\
z_{1} 
\end{array} \rightB$
 and 
$\vect{w} = \leftB
\begin{array}{c}
x_{2} \\
y_{2}\\
z_{2} 
\end{array} \rightB$, their \textbf{dot product}\index{product!dot product}\index{dot product!defined} $\vect{v} \dotprod \vect{w}$ is a number defined
\begin{equation*}
\vect{v} \dotprod \vect{w} = x_{1}x_{2} + y_{1}y_{2} + z_{1}z_{2} = \vect{v}^T\vect{w}
\end{equation*}
\end{definition}

\noindent Because $\vect{v} \dotprod \vect{w}$ is a number, it is sometimes called the \textbf{scalar product}\index{scalar product!defined}\index{product!scalar product}\index{dot product!of two vectors} of $\vect{v}$ and $\vect{w}$.\footnote{Similarly, if 
	$\vect{v} = \leftB
	\begin{array}{c}
	x_{1} \\
	y_{1}
	\end{array} \rightB$
 and 
 $\vect{w} = \leftB
 \begin{array}{c}
 x_{2} \\
 y_{2} 
 \end{array} \rightB$
 in $\RR^2$, then $\vect{v} \dotprod \vect{w} = x_{1}x_{2} + y_{1}y_{2}$.}


\begin{example}{}{011790}
If 
$\vect{v} = \leftB
\begin{array}{r}
2 \\
-1 \\
3 
\end{array} \rightB$
 and
$\vect{w} = \leftB
\begin{array}{r}
1 \\
4\\
-1 
\end{array} \rightB$, then $\vect{v} \dotprod \vect{w} = 2 \cdot 1 + (-1) \cdot 4 + 3 \cdot (-1) = -5$.
\end{example}

The next theorem lists several basic properties of the dot product.


\begin{theorem}{}{011796}
Let $\vect{u}$, $\vect{v}$, and $\vect{w}$ denote vectors in $\RR^3$ (or $\RR^2$).\index{dot product!basic properties}


\begin{enumerate}
\item $\vect{v} \dotprod \vect{w}$ is a real number.

\item  $\vect{v} \dotprod \vect{w} = \vect{w} \dotprod \vect{v}$.

\item  $\vect{v} \dotprod \vect{0} = 0 = \vect{0} \dotprod \vect{v}$.

\item  $\vect{v} \dotprod \vect{v} = \vectlength\vect{v}\vectlength^{2}$.

\item $(k\vect{v}) \dotprod \vect{w} = k(\vect{w} \dotprod \vect{v}) = \vect{v} \dotprod (k\vect{w})$ for all scalars $k$.

\item $\vect{u} \dotprod (\vect{v} \pm \vect{w}) = \vect{u} \dotprod \vect{v} \pm \vect{u} \dotprod \vect{w}$

\end{enumerate}
\end{theorem}

\begin{proof}
(1), (2), and (3) are easily verified, and (4) comes from Theorem~\ref{thm:010965}. The rest are properties of matrix arithmetic (because $\vect{w} \dotprod \vect{v} = \vect{v}^{T}\vect{w}$), and are left to the reader.
\end{proof}

The properties in Theorem~\ref{thm:011796} enable us to do calculations like
\begin{equation*}
3\vect{u} \dotprod (2\vect{v} - 3\vect{w} + 4\vect{z}) = 6(\vect{u} \dotprod \vect{v}) - 9(\vect{u} \dotprod \vect{w}) + 12(\vect{u} \dotprod \vect{z})
\end{equation*}
and such computations will be used without comment below. Here is an example.


\begin{example}{}{011821}
Verify that $\vectlength\vect{v} -3\vect{w}\vectlength^{2} = 1$ when $\vectlength\vect{v}\vectlength = 2$, $\vectlength\vect{w}\vectlength = 1$, and $\vect{v} \dotprod \vect{w} = 2$.


\begin{solution}
  We apply Theorem~\ref{thm:011796} several times:
\begin{align*}
\vectlength \vect{v} - 3\vect{w} \vectlength^2 &= (\vect{v} - 3\vect{w}) \dotprod (\vect{v} - 3\vect{w}) \\
		&= \vect{v} \dotprod (\vect{v} - 3\vect{w}) - 3\vect{w} \dotprod (\vect{v} - 3\vect{w}) \\
		&= \vect{v} \dotprod \vect{v} - 3(\vect{v} \dotprod \vect{w}) - 3(\vect{w} \dotprod \vect{v}) + 9(\vect{w} \dotprod \vect{w}) \\
		&=\vectlength \vect{v} \vectlength^2 - 6(\vect{v} \dotprod \vect{w}) + 9\vectlength \vect{w} \vectlength^2 \\
		&= 4 - 12 + 9 = 1 
\end{align*}
\end{solution}
\end{example}

There is an intrinsic description of the dot product of two nonzero vectors in $\RR^3$. To understand it we require the following result from trigonometry.


\begin{theorem*}{Law of Cosines}{011831}
If a triangle has sides $a$, $b$, and $c$, and if $\theta$ is the interior angle opposite $c$ then\index{cosine}\index{law of cosines}
\begin{equation*}
c^2 = a^2 + b^2 -2ab \cos\theta
\end{equation*}
\end{theorem*}

\begin{wrapfigure}[7]{l}{5cm} 
\centering
\begin{tikzpicture}[scale=0.8]
\coordinate (ptAB) at (0, 0);
\coordinate (ptAC) at (1.5, 2);
\coordinate (ptBC) at (4, 0);
\coordinate (ptBP) at (1.5, 0);

\draw[dkgreenvect, thick] (ptAB)--(ptAC) node[left, text=black, midway]{\small $a$};
\draw[dkgreenvect, thick] (ptAC)--(ptBC) node[above, text=black, midway]{\small $c$};
\draw[dkgreenvect, thick] (ptBC)--(ptAB) node[below, text=black, midway]{\small $b$};
\draw[dkgreenvect, thick] (ptAC)--(ptBP) node[right, text=black, midway]{\small $p$};
\draw[dkbluevect, thick] (1.5, 0.2)--(1.7, 0.2)--(1.7, 0);

\draw[dkbluevect, thick] (0.3, 0) arc [start angle=0, end angle=56.3, radius=0.3] node[right, text=black] {\small $\theta$};
\node[above] at (1, -0.1){\small $q$};
\node[above] at (2.5, -0.1){\small $b - q$};
\end{tikzpicture}

\caption{\label{fig:011840}}
\end{wrapfigure}

\begin{proof} We prove it when is $\theta$ acute, that is $0 \leq \theta < \frac{\pi}{2}$; the obtuse case is similar. In Figure~\ref{fig:011840} we have $p = a \sin \theta$ and $q = a \cos \theta$. Hence Pythagoras' theorem gives
\begin{align*}
c^2 = p^2 + (b - q)^2 &= a^2\sin^2\theta + (b - a\cos\theta)^2 \\
					  &= a^2(\sin^2\theta + \cos^2\theta) +b^2 - 2ab\cos\theta
\end{align*}
The law of cosines follows because $\sin^{2} \theta + \cos^{2} \theta = 1$ for any angle $\theta$.
\end{proof}

\newpage
\begin{wrapfigure}[13]{l}{5cm} 
\centering
\begin{tikzpicture}[scale=0.9]
\draw[dkgreenvect, -latex, thick] (0, 0)--(110:2cm) node[left, text=black, midway]{\small $\vect{v}$};
\draw[dkgreenvect, -latex, thick] (0, 0)--(3, 0) node[below, text=black, midway]{\small $\vect{w}$};

\draw[dkbluevect, thick] (0.5, 0) arc [start angle=0, end angle=110, radius=0.5] node[above, text=black, midway] {\small $\theta$};
\node[right] at (0, 1.5) {\small $\theta$ obtuse};

%bottom image
\draw[dkgreenvect, -latex, thick] (0, -3)-- +(60:2cm) node[left, text=black, midway]{\small $\vect{v}$};
\draw[dkgreenvect, -latex, thick] (0, -3)--(3, -3) node[below, text=black, midway]{\small $\vect{w}$};

\draw[dkbluevect, thick] (0.5, -3) arc [start angle=0, end angle=60, radius=0.5] node[right, text=black, pos=0.6] {\small $\theta$};
\node[right] at (1, -2) {\small $\theta$ acute};
\end{tikzpicture}

\caption{\label{fig:011848}}
\end{wrapfigure}

\noindent Note that the law of cosines reduces to Pythagoras' theorem if $\theta$ is a right angle (because $\cos\frac{\pi}{2} = 0$).\index{Pythagoras' theorem}

Now let $\vect{v}$ and $\vect{w}$ be nonzero vectors positioned with a common tail as in Figure~\ref{fig:011848}. Then they determine a unique angle $\theta$ in the range
\begin{equation*}
0 \leq \theta \leq \pi
\end{equation*}

This angle $\theta$ will be called the \textbf{angle between}\index{vector geometry!angle between two vectors}\index{angles!angle between two vectors} $\vect{v}$ and $\vect{w}$. Figure~\ref{fig:011848} illustrates when $\theta$ is acute (less than $\frac{\pi}{2}$) and obtuse (greater than $\frac{\pi}{2}$). Clearly $\vect{v}$ and $\vect{w}$ are parallel if $\theta$ is either $0$ or $\pi$. Note that we do not define the angle between $\vect{v}$ and $\vect{w}$ if one of these vectors is $\vect{0}$.


The next result gives an easy way to compute the angle between two nonzero vectors using the dot product.


\begin{theorem}{}{011851}
Let $\vect{v}$ and $\vect{w}$ be nonzero vectors. If $\theta$ is the angle between $\vect{v}$ and $\vect{w}$, then
\begin{equation*}
\vect{v} \dotprod \vect{w} = \vectlength \vect{v} \vectlength \vectlength \vect{w} \vectlength \cos\theta
\end{equation*}
\end{theorem}

\begin{wrapfigure}[4]{l}{5cm}
\vspace*{-2em} 
\centering
\begin{tikzpicture}[scale=0.8]
\draw[dkgreenvect, -latex, thick] (0, 0)-- +(45:2.828cm) node[left, text=black, midway]{\small $\vect{v}$}; % 2 / cos 45 = 2.828
\draw[dkgreenvect, -latex, thick] (0, 0)--(4, 0) node[below, text=black, midway]{\small $\vect{w}$};
\draw[dkgreenvect, -latex, thick] (4, 0)-- +(135:2.828cm) node[right, text=black, midway]{\small $\vect{v} - \vect{w}$};

\draw[dkbluevect, thick] (0.5, 0) arc [start angle=0, end angle=45, radius=0.5] node[right, text=black, pos=0.8] {\small $\theta$};
\end{tikzpicture}
\caption{\label{fig:011860}}
\end{wrapfigure}

\begin{proof} We calculate $\vectlength\vect{v} - \vect{w}\vectlength^{2}$ in two ways. First apply the law of cosines to the triangle in Figure~\ref{fig:011860} to obtain:
\begin{equation*}
\vectlength \vect{v} - \vect{w} \vectlength^2 = \vectlength \vect{v} \vectlength^2 + \vectlength \vect{w} \vectlength^2 - 2\vectlength \vect{v} \vectlength \vectlength \vect{w} \vectlength \cos\theta
\end{equation*}

\vspace{1em}
\noindent On the other hand, we use Theorem~\ref{thm:011796}:
\begin{align*}
\vectlength \vect{v} - \vect{w} \vectlength^2 &= (\vect{v} - \vect{w}) \dotprod (\vect{v} - \vect{w}) \\
		&= \vect{v} \dotprod \vect{v} - \vect{v} \dotprod \vect{w} - \vect{w} \dotprod \vect{v} + \vect{w} \dotprod \vect{w} \\
		&= \vectlength \vect{v} \vectlength^2 - 2(\vect{v} \dotprod \vect{w}) + \vectlength \vect{w} \vectlength^2
\end{align*}
Comparing these we see that $-2 \vectlength\vect{v}\vectlength\vectlength\vect{w}\vectlength \cos \theta = -2(\vect{v} \dotprod \vect{w})$, and the result follows.
\end{proof}

If $\vect{v}$ and $\vect{w}$ are nonzero vectors, Theorem~\ref{thm:011851} gives an intrinsic description of $\vect{v} \dotprod \vect{w}$ because $\vectlength\vect{v}\vectlength$, $\vectlength\vect{w}\vectlength$, and the angle $\theta$ between $\vect{v}$ and $\vect{w}$ do not depend on the choice of coordinate system. Moreover, since $\vectlength\vect{v}\vectlength$ and $\vectlength\vect{w}\vectlength$ are nonzero ($\vect{v}$ and $\vect{w}$ are nonzero vectors), it gives a formula for the cosine of the angle $\theta$:
\begin{equation}\label{eq:costheta}
\cos\theta = \frac{\vect{v} \dotprod \vect{w}}{\vectlength \vect{v} \vectlength \vectlength \vect{w} \vectlength}
\end{equation}
Since $0 \leq \theta \leq \pi$, this can be used to find $\theta$.


\begin{example}{}{011867}
Compute the angle between 
$\vect{u} = \leftB
\begin{array}{r}
-1 \\
1 \\
2
\end{array}
\rightB$ 
and 
$\vect{v} = \leftB
\begin{array}{r}
2 \\
1 \\
-1
\end{array}
\rightB$.


\begin{wrapfigure}[7]{l}{5cm}
\vspace*{-2em}
\centering
\begin{tikzpicture}[scale=0.85]
\begin{axis}[disabledatascaling, 
width=5cm, 
height=5cm, 
xlabel={$x$}, 
ylabel={$y$}, 
axis lines=middle, 
xtick=\empty, 
ytick=\empty, 
xticklabels=\empty, 
yticklabels=\empty, 
every axis x label/.style={
	at={(ticklabel* cs:1.05)},
	anchor=west,
},
every axis y label/.style={
	at={(ticklabel* cs:1.05)},
	anchor=south,
},
domain=-5:5, 
samples=100, 
xmin=-1.5, 
xmax=1.5, 
ymin=-1.5, 
ymax=1.5]

\draw[dkgreenvect, thick] (0, 0) circle (1);
\draw[dkbluevect, -latex, thick] (0.5, 0) arc [start angle=0, end angle=120, radius=0.5] node[right, text=black, midway] {\scriptsize $\frac{2\pi}{3}$};
\draw[dkbluevect, thick] (0, 0)--(120:1) node (pt) {};

\draw[dkbluevect, dashed] (pt)--(pt |- 0, 0) node (ptXCoord) {};
\fill[dkbluevect] (pt) circle (2pt) node[above=0.1, text=black] {\tiny $\left(\frac{-1}{2}, \frac{\sqrt{3}}{2}\right)$};
\fill[dkbluevect] (ptXCoord) circle (2pt) node[below, text=black] {\scriptsize $\frac{-1}{2}$};
\node[below right] at (0, 0) {\scriptsize $O$};
\end{axis}
\end{tikzpicture}
%\captionof{figure}{\label{fig:011878}}
\end{wrapfigure}

\setlength{\rightskip}{0pt plus 200pt}
\begin{solution}
 Compute $\cos\theta = \frac{\vect{v} \dotprod \vect{w}}{\vectlength \vect{v} \vectlength \vectlength \vect{w} \vectlength} = \frac{-2 + 1 -2}{\sqrt{6}\sqrt{6}} = -\frac{1}{2}$. Now recall that $\cos\theta$ and $\sin \theta$ are defined so that ($\cos \theta$, $\sin \theta$) is the point on the unit circle determined by the angle $\theta$ (drawn counterclockwise, starting from the positive $x$ axis). In the present case, we know that $\cos \theta = -\frac{1}{2}$ and that $0 \leq \theta \leq \pi$. Because $\cos\frac{\pi}{3} = \frac{1}{2}$, it follows that $\theta = \frac{2\pi}{3}$ (see the diagram).
\vspace{2em}
\end{solution}
\end{example}

If $\vect{v}$ and $\vect{w}$ are nonzero, equation (\ref{eq:costheta}) shows that $\cos \theta$ has the same sign as $\vect{v} \dotprod \vect{w}$, so
\begin{equation*}
\begin{array}{lll}
\vect{v} \dotprod \vect{w} > 0 & \mbox{if and only if } & \theta \mbox{ is acute } (0 \leq \theta < \frac{\pi}{2}) \\
\vect{v} \dotprod \vect{w} < 0 & \mbox{if and only if } & \theta \mbox{ is obtuse } (\frac{\pi}{2} < \theta \leq 0) \\
\vect{v} \dotprod \vect{w} = 0 & \mbox{if and only if } & \theta = \frac{\pi}{2}
\end{array}
\end{equation*}
In this last case, the (nonzero) vectors are perpendicular. The following terminology is used in linear algebra:


\begin{definition}{Orthogonal Vectors in $\RR^3$}{011882}
Two vectors $\vect{v}$ and $\vect{w}$ are said to be \textbf{orthogonal}\index{orthogonal vectors}\index{vectors!orthogonal vectors} if $\vect{v} = \vect{0}$ or $\vect{w} = \vect{0}$ or the angle between them is $\frac{\pi}{2}$.
\end{definition}

\noindent Since $\vect{v} \dotprod \vect{w} = 0$ if either $\vect{v} = \vect{0}$ or $\vect{w} = \vect{0}$, we have the following theorem:


\begin{theorem}{}{011886}
Two vectors $\vect{v}$ and $\vect{w}$ are orthogonal if and only if $\vect{v} \dotprod \vect{w} = 0$.
\end{theorem}

\begin{example}{}{011889}
Show that the points $P(3, -1, 1)$, $Q(4, 1, 4)$, and $R(6, 0, 4)$ are the vertices of a right triangle.


\begin{solution}
  The vectors along the sides of the triangle are
\begin{equation*}
\longvect{PQ} = \leftB
\begin{array}{r}
1 \\
2 \\
3
\end{array}
\rightB,\
\longvect{PR} = \leftB
\begin{array}{r}
3 \\
1 \\
3
\end{array}
\rightB, \mbox{ and }
\longvect{QR} = \leftB
\begin{array}{r}
2 \\
-1 \\
0
\end{array}
\rightB
\end{equation*}
Evidently $\longvect{PQ} \cdot \longvect{QR} = 2 - 2 + 0 = 0$, so $\longvect{PQ}$ and $\longvect{QR}$ are orthogonal vectors. This means sides $PQ$ and $QR$ are perpendicular---that is, the angle at $Q$ is a right angle.
\end{solution}
\end{example}

Example~\ref{exa:011899} demonstrates how the dot product can be used to verify geometrical theorems involving perpendicular lines.


\begin{example}{}{011899}
A parallelogram with sides of equal length is called a \textbf{rhombus}\index{parallelogram!rhombus}\index{rhombus}. Show that the diagonals of a rhombus are perpendicular.

\begin{wrapfigure}[6]{l}{5cm}
\centering
\begin{tikzpicture}
[scale=1.3]
\draw[dkgreenvect, -latex, thick] (0, 0)--(2, 0) node[below, text=black, midway] {\small $\vect{v}$};
\draw[dkgreenvect, -latex, thick] (0, 0)-- +(60:2cm) node[left, text=black, midway] {\small $\vect{u}$};
\draw[dkgreenvect, -latex, thick] (2, 0)-- +(120:2cm) node[right, text=black, pos=0.8] {\small $\vect{u} - \vect{v}$};
\draw[dkbluevect, thick] (2, 0)-- ++(60:2cm) node (sum){} -- ++(-2, 0);
\draw[dkgreenvect, -latex, thick] (0, 0)--(sum.center) node[right=0.1, text=black, pos=0.55] {\small $\vect{u} + \vect{v}$};
\end{tikzpicture}
%\captionof{figure}{\label{fig:011906}}
\end{wrapfigure}

\setlength{\rightskip}{0pt plus 200pt} 
\begin{solution}  Let $\vect{u}$ and $\vect{v}$ denote vectors along two adjacent sides of a rhombus, as shown in the diagram. Then the diagonals are $\vect{u} - \vect{v}$ and $\vect{u} + \vect{v}$, and we compute
\begin{align*}
(\vect{u} - \vect{v}) \dotprod (\vect{u} + \vect{v}) &= \vect{u} \dotprod (\vect{u} + \vect{v}) - \vect{v} \dotprod (\vect{u} + \vect{v})\\
&= \vect{u} \dotprod \vect{u} + \vect{u} \dotprod \vect{v} - \vect{v} \dotprod \vect{u} - \vect{v} \dotprod \vect{v} \\
&= \vectlength \vect{u} \vectlength^2 - \vectlength \vect{v} \vectlength^2 \\
&= 0
\end{align*}
because $\vectlength\vect{u}\vectlength = \vectlength\vect{v}\vectlength$ (it is a rhombus). Hence $\vect{u} - \vect{v}$ and $\vect{u} + \vect{v}$ are orthogonal.

\end{solution}
\end{example}

\subsection*{Projections}


In applications of vectors, it is frequently useful to write a vector as the sum of two orthogonal vectors. Here is an example.\index{projections}


\begin{example}{}{011910}
Suppose
 a ten-kilogram block is placed on a flat surface inclined $30^{\circ}$  to the 
horizontal as in the diagram. Neglecting friction, how much force is 
required to keep the block from sliding down the surface?



\begin{wrapfigure}[6]{l}{5cm} 
\centering
\begin{tikzpicture}[scale=1.0]
\draw[dkbluevect, thick] (0, 0)--(-1, 0);
\draw[dkbluevect, thick] (0, 0)--(150:3.5cm);
\draw[dkbluevect, thick] (-0.5, 0) arc [start angle=180, end angle=150, radius=0.5] node[left, text=black, pos=0.7] {\scriptsize $30^{\circ}$};

%rotation matrix
%x' = x cos theta - y sin theta
%y' = x sin theta + y cos theta
\path[rotate=-30](0, 0)--(-2.75, 0.25) node (block) {};
\filldraw[dkbluevect, fill=ltbluevect, rotate=-30] (-3, 0) rectangle +(0.5, 0.5);
\draw[dkgreenvect, thick, -latex] (block.center)-- ++(0, -1.5) node (ptW) {};
\draw[dkgreenvect, thick, -latex, rotate=-30] (-2.75, 0.25)-- +(0.75, 0) node (ptW1) {};
\draw[dkgreenvect, thick, -latex, rotate=-30] (-2.75, 0.25)-- +(0, -1.299) node (ptW2) {}; %cos 30 * 1.5 = 1.299

\draw[dkbluevect, thick, dashed] (ptW2.center)--(ptW.center)--(ptW1.center);
\node [above=0.25cm of ptW] (ptWArc) {};
\draw[dkbluevect, thick] (ptWArc) arc [start angle=90, end angle=60, radius=0.5] node[above, text=black, pos=0.9] {\scriptsize $30^{\circ}$};

%vector labels
\path (ptW) + (0.1, 0.4) node[left] {\small $\vect{w}$};
\node[above] at (ptW1) {\small $\vect{w}_1$};
\node[above left] at (ptW2) {\small $\vect{w}_2$};
\end{tikzpicture}
%\captionof{figure}{\label{fig:011925}}
\end{wrapfigure}

\setlength{\rightskip}{0pt plus 200pt}
\begin{solution}  Let $\vect{w}$ denote the weight (force due to gravity) exerted on the block. Then $\vectlength\vect{w}\vectlength = 10$ kilograms and the direction of $\vect{w}$ is vertically down as in the diagram. The idea is to write $\vect{w}$ as a sum $\vect{w} = \vect{w}_{1} + \vect{w}_{2}$ where $\vect{w}_{1}$ is parallel to the inclined surface and $\vect{w}_{2}$ is perpendicular to the surface. Since there is no friction, the force required is $-\vect{w}_{1}$ because the force $\vect{w}_{2}$ has no effect parallel to the surface. As the angle between $\vect{w}$ and $\vect{w}_{2}$ is $30^{\circ}$  in the diagram, we have $\frac{\vectlength \vect{w}_{1} \vectlength}{\vectlength \vect{w} \vectlength} = \sin30^\circ = \frac{1}{2}$. Hence $\vectlength \vect{w}_{1} \vectlength = \frac{1}{2} \vectlength \vect{w} \vectlength = \frac{1}{2}10 = 5$. Thus the required force has a magnitude of $5$ kilograms weight directed up the surface.

\end{solution}
\end{example}


\newpage
\begin{wrapfigure}[11]{l}{5cm}
\vspace{-1em}
\centering
\begin{tikzpicture}[scale=0.9, rotate=30]
\coordinate(ptQ) at (0, 0);
\coordinate(ptP) at (2, 1.5);
\coordinate(ptP1) at (2, 0);
\coordinate(ptD) at (3, 0);

%lines
\draw[dkbluevect, thick] (1.8, 0)-- ++(0, 0.2)-- ++(0.2, 0);
\draw[dkbluevect, thick] (-0.5, 0)--(3.5, 0);
\draw[dkgreenvect, thick, -latex] (ptQ)--(ptD);
\draw[dkgreenvect, thick, -latex] (ptP1)--(ptP);
\draw[dkgreenvect, thick, -latex] (ptQ)--(ptP) node [above=0.2, text=black, pos=0.3] {\footnotesize $\vect{u}$};
\draw[dkgreenvect, thick, -latex] (0.1, -0.2)--(1.9, -0.2) node[below, text=black, midway] {\footnotesize $\vect{u}_1$};

%dots
\fill[black] (ptQ) circle (2pt);
\fill[black] (ptP) circle (2pt);
\fill[black] (ptP1) circle (2pt);

%labels
\node[below] at (ptQ) {\footnotesize $Q$};
\node[above] at (ptP) {\footnotesize $P$};
\node[below right] at (ptP1) {\footnotesize $P_1$};
\node[above] at (ptD) {\footnotesize $\vect{d}$};
\path (ptP1)--(ptP) node [right, pos=0.9] {\footnotesize $\vect{u} - \vect{u}_1$};
\node[left] at (-0.5, -0.5) {\footnotesize (a)};
\end{tikzpicture}

\begin{tikzpicture}[scale=0.9, rotate=30]
\coordinate(ptQ) at (1.5, 0);
\coordinate(ptP1) at (0, 0);
\coordinate(ptP) at (0, 1.5);
\coordinate(ptD) at (3, 0);

%lines
\draw[dkbluevect, thick] (-0.2, 0)-- ++(0, 0.2)-- ++(0.2, 0);
\draw[dkbluevect, thick] (-1, 0)--(3.5, 0);
\draw[dkgreenvect, thick, -latex] (ptQ)--(ptD);
\draw[dkgreenvect, thick, -latex] (ptP1)--(ptP);
\draw[dkgreenvect, thick, -latex] (ptQ)--(ptP) node [above, text=black, midway] {\footnotesize $\vect{u}$};
\draw[dkgreenvect, thick, -latex] (ptQ)--(ptP1) node[below, text=black, midway] {\footnotesize $\vect{u}_1$};

%dots
\fill[black] (ptQ) circle (2pt);
\fill[black] (ptP) circle (2pt);
\fill[black] (ptP1) circle (2pt);

%labels
\node[below] at (ptQ) {\footnotesize $Q$};
\node[above] at (ptP) {\footnotesize $P$};
\node[below right] at (ptP1) {\footnotesize $P_1$};
\node[above] at (ptD) {\footnotesize $\vect{d}$};
\path (ptP1)--(ptP) node [left, pos=0.5] {\footnotesize $\vect{u} - \vect{u}_1$};
\node[left] at (-1, -0.8) {\footnotesize (b)};
\end{tikzpicture}

\vspace*{-1em}\caption{\label{fig:011945}}
\end{wrapfigure}

If a nonzero vector $\vect{d}$ is specified, the key idea in Example~\ref{exa:011910} is to be able to write an arbitrary vector $\vect{u}$ as a sum of two vectors,
\begin{equation*}
\vect{u} = \vect{u}_{1} + \vect{u}_{2}
\end{equation*}
where $\vect{u}_{1}$ is parallel to $\vect{d}$ and $\vect{u}_{2} = \vect{u} - \vect{u}_{1}$ is orthogonal to $\vect{d}$. Suppose that $\vect{u}$ and $\vect{d} \neq \vect{0}$ emanate from a common tail $Q$ (see Figure~\ref{fig:011945}). Let $P$ be the tip of $\vect{u}$, and let $P_{1}$ denote the foot of the perpendicular from $P$ to the line through $Q$ parallel to $\vect{d}$. 

Then $\vect{u}_{1} = \longvect{QP}_{1}$ has the required properties:
\begin{enumerate}
\item $\vect{u}_{1}$ \textit{is parallel to} $\vect{d}$.

\item $\vect{u}_{2} = \vect{u} - \vect{u}_{1}$ \textit{is orthogonal to} $\vect{d}$.

\item $\vect{u} = \vect{u}_{1} + \vect{u}_{2}$.

\end{enumerate}

\hfill\begin{definition}{Projection in $\RR^3$}{011946}
The vector $\vect{u}_{1} = \longvect{QP}_{1}$ in Figure~\ref{fig:011945} is called \textbf{the projection}\index{vector geometry!projections} of $\vect{u}$ on $\vect{d}$. It is denoted
\begin{equation*}
\vect{u}_1 = \proj{\vect{d}}{\vect{u}}
\end{equation*}
\end{definition}


\noindent In Figure~\ref{fig:011945}(a) the vector $\vect{u}_{1} = \proj{\vect{d}}{\vect{u}}$ has the same direction as $\vect{d}$; however, $\vect{u}_{1}$ and $\vect{d}$ have opposite directions if the angle between $\vect{u}$ and $\vect{d}$ is greater than $\frac{\pi}{2}$ (Figure~\ref{fig:011945}(b)). Note that the projection $\vect{u}_1 = \proj{\vect{d}}{\vect{u}}$ is zero if and only if $\vect{u}$ and $\vect{d}$ are orthogonal.


Calculating the projection of $\vect{u}$ on $\vect{d} \neq \vect{0}$ is remarkably easy.


\begin{theorem}{}{011958}
Let $\vect{u}$ and $\vect{d} \neq \vect{0}$ be vectors.


\begin{enumerate}
\item The projection of $\vect{u}$ on $\vect{d}$ is given by $\proj{\vect{d}}{\vect{u}} = \frac{\vect{u} \dotprod \vect{d}}{\vectlength \vect{d} \vectlength^2} \vect{d}$.


\item The vector $\vect{u} - \proj{\vect{d}}{\vect{u}}$ is orthogonal to $\vect{d}$.

\end{enumerate}
\end{theorem}

\begin{proof}
The vector $\vect{u}_{1} = \proj{\vect{d}}{\vect{u}}$ is parallel to $\vect{d}$ and so has the form $\vect{u}_{1} = t\vect{d}$ for some scalar $t$. The requirement that $\vect{u} - \vect{u}_{1}$ and $\vect{d}$ are orthogonal determines $t$. In fact, it means that $(\vect{u} - \vect{u}_{1}) \dotprod \vect{d} = 0$ by Theorem~\ref{thm:011886}. If $\vect{u}_{1} = t\vect{d}$ is substituted here, the condition is
\begin{equation*}
0 = (\vect{u} - t\vect{d}) \dotprod \vect{d} = \vect{u} \dotprod \vect{d} - t(\vect{d} \dotprod \vect{d}) = \vect{u} \dotprod \vect{d} - t\vectlength \vect{d} \vectlength^2
\end{equation*}
It follows that $t = \frac{\vect{u} \dotprod \vect{d}}{\vectlength \vect{d} \vectlength^2}$, where the assumption that $\vect{d} \neq \vect{0}$ guarantees that $\vectlength \vect{d} \vectlength^{2} \neq 0$.
\end{proof}

\begin{example}{}{011981}
Find the projection of 
$\vect{u} = \leftB
\begin{array}{r}
2 \\
-3 \\
1
\end{array}
\rightB$
 on 
$\vect{d} = \leftB
\begin{array}{r}
1\\
-1 \\
3
\end{array}
\rightB$
 and express $\vect{u} = \vect{u}_{1} + \vect{u}_{2}$ where $\vect{u}_{1}$ is parallel to $\vect{d}$ and $\vect{u}_{2}$ is orthogonal to $\vect{d}$.


\begin{solution}
  The projection $\vect{u}_{1}$ of $\vect{u}$ on $\vect{d}$ is
\begin{equation*}
\vect{u}_{1} = \proj{\vect{d}}{\vect{u}} = \frac{\vect{u} \dotprod \vect{d}}{\vectlength \vect{d} \vectlength^2}\vect{d} = \frac{2 + 3 + 3}{1^2 + (-1)^2 + 3^2} 
\leftB
\begin{array}{r}
1\\
-1 \\
3
\end{array}
\rightB
= \frac{8}{11}\leftB
\begin{array}{r}
1\\
-1 \\
3
\end{array}
\rightB
\end{equation*}
Hence $\vect{u}_{2} = \vect{u} - \vect{u}_{1} = 
\frac{1}{11}\leftB
\begin{array}{r}
14\\
-25 \\
-13
\end{array}
\rightB$, and this is orthogonal to $\vect{d}$ by Theorem~\ref{thm:011958} (alternatively, observe that $\vect{d} \dotprod \vect{u}_{2} = 0$). Since $\vect{u} = \vect{u}_{1} + \vect{u}_{2}$, we are done.
\end{solution}
\end{example}

\begin{example}{}{012000}
\begin{wrapfigure}[3]{l}{4cm}
\vspace*{-1em}
\centering
\begin{tikzpicture}[scale=0.8, rotate=30]
\coordinate(ptP) at (2, 1.25);
\coordinate(ptP0) at (0, 0);
\coordinate(ptQ) at (2, 0);
\coordinate(ptD) at (3, 0);

%lines
\draw[dkbluevect, thick] (ptP0)+(-0.5, 0) -- (ptD) -- +(0.5, 0);
\draw[dkgreenvect, -latex, thick] (ptP0)--(ptP) node[above=0.2, text=black, pos=0.4] {\footnotesize $\vect{u}$};
\draw[dkgreenvect, -latex, thick] (ptP0)--(ptQ) node[above, text=black, pos=0.7] {\footnotesize $\vect{u}_1$};
\draw[dkgreenvect, -latex, thick] (ptQ)--(ptD);
\draw[dkgreenvect, -latex, thick] (ptQ)--(ptP) node[right, text=black, pos=0.7] {\footnotesize $\vect{u} - \vect{u}_1$};

%dots
\fill[black] (ptQ) circle (2pt);
\fill[black] (ptP) circle (2pt);
\fill[black] (ptP0) circle (2pt);

\node[below] at (ptQ) {\footnotesize $Q$};
\path(ptP) + (1, -0.3) node {\footnotesize $P(1, 3, -2)$};
\path(ptP0) + (1, -0.5) node[below] {\footnotesize $P_0(2, 0, -1)$};
\node[below] at (ptD) {\footnotesize $\vect{d}$};
\end{tikzpicture}
%\captionof{figure}{\label{fig:012005}}
\end{wrapfigure}

\setlength{\rightskip}{0pt plus 200pt}
Find the shortest distance (see diagram) from the point $P(1, 3, -2)$ to the line through $P_{0}(2, 0, -1)$ with direction vector $ \vect{d} = \leftB
\begin{array}{r}
1\\
-1 \\
0
\end{array}
\rightB$. Also find the point $Q$ that lies on the line and is closest to $P$.

\begin{solution}
Let 
$\vect{u} = \leftB
\begin{array}{r}
1\\
3 \\
-2
\end{array}
\rightB
-
\leftB
\begin{array}{r}
2\\
0 \\
-1
\end{array}
\rightB
=
\leftB
\begin{array}{r}
-1\\
3 \\
-1
\end{array}
\rightB$
 denote the vector from $P_{0}$ to $P$, and let $\vect{u}_{1}$ denote the projection of $\vect{u}$ on $\vect{d}$. Thus
\begin{equation*}
\vect{u}_{1} = \frac{\vect{u} \dotprod \vect{d}}{\vectlength \vect{d} \vectlength^2}\vect{d} = \frac{-1 - 3 + 0}{1^2 + (-1)^2 + 0^2}\vect{d} = -2\vect{d} = 
\leftB
\begin{array}{r}
-2\\
2 \\
0
\end{array}
\rightB
\end{equation*}
by Theorem~\ref{thm:011958}. We see geometrically that the point $Q$ on the line is closest to $P$, so the distance is
\begin{equation*}
\vectlength \longvect{QP} \vectlength = \vectlength \vect{u} - \vect{u}_{1} \vectlength = \left\vectlength 
\leftB
\begin{array}{r}
1\\
1 \\
-1
\end{array}
\rightB \right\vectlength
= \sqrt{3}
\end{equation*}
To find the coordinates of $Q$, let $\vect{p}_{0}$ and $\vect{q}$ denote the vectors of $P_{0}$ and $Q$, respectively. Then $\vect{p}_{0} = \leftB
\begin{array}{r}
2\\
0 \\
-1
\end{array}
\rightB$
 and $\vect{q} = \vect{p}_{0} + \vect{u}_{1} = \leftB
 \begin{array}{r}
 0\\
 2 \\
 -1
 \end{array}
 \rightB$. Hence $Q(0, 2, -1)$ is the required point. It can be checked that the distance from $Q$ to $P$ is $\sqrt{3}$, as expected.
\end{solution}
\end{example}

\subsection*{Planes}


It
 is evident geometrically that among all planes that are perpendicular 
to a given straight line there is exactly one containing any given 
point. This fact can be used to give a very simple description of a 
plane. To do this, it is necessary to introduce the following notion:\index{planes}\index{vector geometry!planes}

\begin{definition}{Normal Vector in a Plane}{012024}
A nonzero vector $\vect{n}$ is called a \textbf{normal}\index{normal} for a plane if it is orthogonal to every vector in the plane.
\end{definition}

\begin{wrapfigure}[6]{l}{5cm} 
\centering
\begin{tikzpicture}[scale=0.9]
\filldraw[color=dkbluevect, fill=ltbluevect, thick] (0, 0)--(4, 0)--(3.5, -2)--(-0.5, -1)--cycle;
\coordinate (ptP0) at (0.5, -0.5);
\coordinate (ptP) at (3.5, 0.5);
\draw[dkgreenvect, thick, -latex] (ptP0)--(0.5, 0.5) node[above right, text=black, pos = 0.6] {\small $\vect{n}$};
\draw[dkgreenvect, thick, -latex] (ptP0)--(ptP);
\fill[black] (ptP0) circle (2pt);
\fill[black] (ptP) circle (2pt);

\node[left] at (ptP0) {\small $P_0$};
\node[above right] at (ptP) {\small $P$};
\end{tikzpicture}
\caption{\label{fig:012038}}
\end{wrapfigure}

\noindent For example, the coordinate vector $\vect{k}$ is a normal for the $x$-$y$ plane.

Given a point $P_{0} = P_{0}(x_{0}, y_{0}, z_{0})$ and a nonzero vector $\vect{n}$, there is a unique plane through $P_{0}$ with normal $\vect{n}$, shaded in Figure~\ref{fig:012038}. A point $P = P(x, y, z)$ lies on this plane if and only if the vector $\longvect{P_{0}P}$ is orthogonal to $\vect{n}$---that is, if and only if $\vect{n} \dotprod \longvect{P_{0}P} = 0$. Because $\longvect{P_{0}P} = \leftB
\begin{array}{c}
x - x_{0}\\
y - y_{0}\\
z - z_{0}
\end{array}
\rightB$ this gives the following result:
\vspace{1em}

\begin{theorem*}{Scalar Equation of a Plane}{012039}
The plane through $P_{0}(x_{0}, y_{0}, z_{0})$ with normal $\vect{n} = \leftB
\begin{array}{c}
a\\
b\\
c
\end{array}
\rightB
\neq \vect{0}$ 
 as a normal vector is given by
\begin{equation*}
a(x - x_{0}) + b(y - y_{0}) + c(z - z_{0}) = 0
\end{equation*}
In other words, a point $P(x, y, z)$ is on this plane if and only if $x$, $y$, and $z$ satisfy this equation.\index{scalar equation of a plane}
\end{theorem*}

\begin{example}{}{012049}
Find an equation of the plane through $P_{0}(1, -1, 3)$ with $\vect{n} = \leftB
\begin{array}{r}
3\\
-1\\
2
\end{array}
\rightB$
 as normal.


\begin{solution}
  Here the general scalar equation becomes
\begin{equation*}
3(x - 1) - (y + 1) + 2(z - 3) = 0
\end{equation*}
This simplifies to $3x - y + 2z = 10$.
\end{solution}
\end{example}

If we write $d = ax_{0} + by_{0} + cz_{0}$, the scalar equation shows that every plane with normal $\vect{n} = \leftB
\begin{array}{r}
a\\
b\\
c
\end{array}
\rightB$
 has a linear equation of the form
\begin{equation} \label{eq:linerformeq} 
ax + by + cz = d
\end{equation}
for some constant $d$. Conversely, the graph of this equation is a plane with $\vect{n} = \leftB
\begin{array}{r}
a\\
b\\
c
\end{array}
\rightB$ as a normal vector (assuming that $a$, $b$, and $c$ are not all zero).


\begin{example}{}{012066}
Find an equation of the plane through $P_{0}(3, -1, 2)$ that is parallel to the plane with equation $2x - 3y = 6$.


\begin{solution}
The plane with equation $2x -3y = 6$ has normal $\vect{n} = \leftB
\begin{array}{r}
2\\
-3\\
0
\end{array}
\rightB$. Because the two planes are parallel, $\vect{n}$ serves as a normal for the plane we seek, so the equation is $2x - 3y = d$ for some $d$ by Equation \ref{eq:linerformeq}. Insisting that $P_{0}(3, -1, 2)$ lies on the plane determines $d$; that is, $d = 2 \cdot 3 - 3(-1) = 9$. Hence, the equation is $2x - 3y = 9$.
\end{solution}
\end{example}

\noindent Consider points $P_{0}(x_{0}, y_{0}, z_{0})$ and $P(x, y, z)$ with vectors $\vect{p}_{0} = \leftB
\begin{array}{r}
x_{0}\\
y_{0}\\
z_{0}
\end{array}
\rightB$
and 
$\vect{p}= \leftB
\begin{array}{r}
x\\
y\\
z
\end{array}
\rightB$.
Given a nonzero vector $\vect{n}$, the scalar equation of the plane through $P_{0}(x_{0}, y_{0}, z_{0})$ with normal $\vect{n} = \leftB
\begin{array}{r}
a\\
b\\
c
\end{array}
\rightB$ takes the vector form:


\begin{theorem*}{Vector Equation of a Plane}{012088}
The plane with normal $\vect{n} \neq \vect{0}$ through the point with vector $\vect{p}_{0}$ is given by
\begin{equation*}
\vect{n} \dotprod (\vect{p} - \vect{p}_{0}) = 0
\end{equation*}
In other words, the point with vector $\vect{p}$ is on the plane if and only if $\vect{p}$ satisfies this condition.\index{vector equation of a plane}
\end{theorem*}

\noindent Moreover, Equation \ref{eq:linerformeq} translates as follows:
\begin{center}
Every plane with normal $\vect{n}$ has vector equation $\vect{n} \dotprod \vect{p} = d$ for some number $d$.
\end{center}
\noindent This is useful in the second solution of Example~\ref{exa:012097}.


\begin{example}{}{012097}
Find the shortest distance from the point $P(2, 1, -3)$ to the plane with equation $3x - y + 4z = 1$. Also find the point $Q$ on this plane closest to $P$.

\newpage
\begin{wrapfigure}[5]{l}{5cm} 
\centering
\begin{tikzpicture}[scale=1.3]
\filldraw[color=dkbluevect, fill=ltbluevect, thick] (0,0)--(1.2,0.75)--(3,0)--(1.9,-1)--cycle;
\draw[thick, dkbluevect] (0.6,0)--(1.7,0)--(1.7,1.25);
\draw[thick, dashed, dkbluevect](0.6,1.25)--(1.7,1.25);
\draw[dkgreenvect, thick, -latex] (0.6,0)--(1.7,1.25) node[midway,below right, text=black]{\footnotesize $\vect{u}$};
\draw[dkgreenvect, thick, -latex] (0.6,0)--(0.6,1.25);
\draw[dkgreenvect, thick, -latex] (0.6,1.25)--(0.6,1.6);
\draw[dkbluevect](1.55,0)--(1.55,0.15)--(1.7,0.15);
\fill (0.6,0) circle (2pt);
\fill (1.7,0) circle (2pt);
\fill (1.7,1.25) circle (2pt);
\node[below] at (0.5,0) {\footnotesize $P_0(0, -1, 0)$};
\node[right] at (1.7,0) {\footnotesize $Q(x, y, z)$};
\node[right] at (1.7,1.25) {\footnotesize $P(2, 1, -3)$};
\node[left] at (0.6,1.15){\footnotesize $\vect{u}_1$};
\node[left] at (0.6,1.5){\footnotesize $\vect{n}$};
\end{tikzpicture}

%\captionof{figure}{\label{fig:012112}}
\end{wrapfigure}

\setlength{\rightskip}{0pt plus 200pt}
 \begin{solution}[1] The plane in question has normal $\vect{n} = \leftB
 \begin{array}{r}
 3\\
 -1\\
 4
 \end{array}
 \rightB$. Choose any point $P_{0}$ on the plane---say $P_{0}(0, -1, 0)$---and let $Q(x, y, z)$ be the point on the plane closest to $P$ (see the diagram). The vector from $P_{0}$ to $P$ is $\vect{u} = \leftB
 \begin{array}{r}
 2\\
 2\\
 -3
 \end{array}
 \rightB$. Now erect $\vect{n}$ with its tail at $P_{0}$. Then $\longvect{QP} = \vect{u}_{1}$ and $\vect{u}_{1}$ is the projection of $\vect{u}$ on $\vect{n}$:
\begin{equation*}
\vect{u}_{1} = \frac{\vect{n} \dotprod \vect{u}}{\vectlength \vect{n} \vectlength^2}\vect{n} = \frac{-8}{26}
\leftB
\begin{array}{r}
3\\
-1 \\
4
\end{array}
\rightB
= \frac{-4}{13}
\leftB
\begin{array}{r}
3\\
-1 \\
4
\end{array}
\rightB
\end{equation*}
Hence the distance is $\vectlength \longvect{QP} \vectlength = \vectlength \vect{u}_{1} \vectlength = \frac{4\sqrt{26}}{13}$. To calculate the point $Q$, let $\vect{q} = \leftB
\begin{array}{r}
x\\
y \\
z
\end{array}
\rightB$
and 
$\vect{p}_{0} = \leftB
\begin{array}{r}
0\\
-1 \\
0
\end{array}
\rightB$
 be the vectors of $Q$ and $P_{0}$. Then
\begin{equation*}
\vect{q} = \vect{p}_{0} + \vect{u} - \vect{u}_{1} = 
\leftB
\begin{array}{r}
0\\
-1 \\
0
\end{array}
\rightB
+
\leftB
\begin{array}{r}
2\\
2 \\
-3
\end{array}
\rightB
+ \frac{4}{13}
\leftB
\begin{array}{r}
3\\
-1 \\
4
\end{array}
\rightB
=
\leftB \def\arraystretch{1.5}
\begin{array}{r}
\frac{38}{13}\\
\frac{9}{13}\\
\frac{-23}{13}
\end{array}
\rightB
\end{equation*}
This gives the coordinates of $Q(\frac{38}{13}, \frac{9}{13}, \frac{-23}{13})$.
\end{solution}

\begin{solution}[2]
Let $\vect{q}= \leftB
\begin{array}{c}
x\\
y \\
z
\end{array}
\rightB$
and $\vect{p} = \leftB
\begin{array}{r}
2\\
1 \\
-3
\end{array}
\rightB$
 be the vectors of $Q$ and $P$. Then $Q$ is on the line through $P$ with direction vector $\vect{n}$, so $\vect{q} = \vect{p} + t\vect{n}$ for some scalar $t$. In addition, $Q$ lies on the plane, so $\vect{n} \dotprod \vect{q} = 1$. This determines $t$:
\begin{equation*}
1 = \vect{n} \dotprod \vect{q} = \vect{n} \dotprod (\vect{p} + t\vect{n}) = \vect{n} \dotprod \vect{p} + t\vectlength \vect{n} \vectlength^2 = -7 + t(26)
\end{equation*}
This gives $t = \frac{8}{26} = \frac{4}{13}$, so
\begin{equation*}
\leftB
\begin{array}{c}
x\\
y \\
z
\end{array}
\rightB 
= \vect{q} = \vect{p} + t\vect{n} = 
\leftB
\begin{array}{r}
2\\
1 \\
-3
\end{array}
\rightB
+ \frac{4}{13}
\leftB
\begin{array}{r}
3\\
-1 \\
4
\end{array}
\rightB
= \frac{1}{13}
\leftB
\begin{array}{r}
38\\
9 \\
-23
\end{array}
\rightB
\end{equation*}
as before. This determines $Q$ (in the diagram), and the reader can verify that the required distance is $\vectlength \longvect{QP} \vectlength = \frac{4}{13}\sqrt{26}$, as before.
\end{solution}
\end{example}

\vspace*{-1em}
\subsection*{The Cross Product}

If $P$, $Q$, and $R$ are three distinct points in $\RR^3$ that are not all on some line, it is clear geometrically that there is a unique plane containing all three. The vectors $\longvect{PQ}$ and $\longvect{PR}$ both lie in this plane, so finding a normal amounts to finding a nonzero vector orthogonal to both $\longvect{PQ}$ and $\longvect{PR}$. The cross product provides a systematic way to do this.\index{cross product!and dot product}\index{dot product!and cross product}


\begin{definition}{Cross Product}{012135}
Given vectors $\vect{v}_{1}= \leftB
\begin{array}{c}
x_{1}\\
y_{1} \\
z_{1}
\end{array}
\rightB$
 \textit{and} 
$\vect{v}_{2}= \leftB
\begin{array}{c}
x_{2}\\
y_{2} \\
z_{2}
\end{array}
\rightB$, define the \textbf{cross product}\index{vector geometry!cross product}\index{cross product!defined} $\vect{v}_{1} \times \vect{v}_{2}$ by
\begin{equation*}
\vect{v}_{1} \times \vect{v}_{2} = \leftB
\begin{array}{c}
y_{1}z_{2} - z_{1}y_{2}\\
-(x_{1}z_{2} - z_{1}x_{2}) \\
x_{1}y_{2} - y_{1}x_{2}
\end{array}
\rightB
\end{equation*}
\end{definition}

\begin{wrapfigure}[9]{l}{5cm} 
\centering
\begin{tikzpicture}[scale=1.0]
%set up of axis environment
\begin{axis}[view/h=120,disabledatascaling, 
    width=5cm, 
    height=5cm, 
    xlabel={$x$}, 
    ylabel={$y$},
    zlabel={$z$},
    axis lines=center,
    axis on top,
    xtick=\empty,
    ytick=\empty,
    ztick=\empty,
    xticklabels=\empty, 
    yticklabels=\empty, 
    zticklabels=\empty, 
    every axis x label/.style={
      at={(ticklabel* cs:1.05)},
      anchor=north,
    },
    every axis y label/.style={
      at={(ticklabel* cs:1.05)},
      anchor=west,
    },
   every axis z label/.style={
      at={(ticklabel* cs:1.05)},
      anchor=south,
    },
    domain=-5:5, 
    samples=100, 
    xmin=0, 
    xmax=2, 
    ymin=0, 
    ymax=2,
    zmin=0,
    zmax=2]
    
    \coordinate (origin) at (0, 0, 0);
    \coordinate (ptI) at (1, 0, 0);
    \coordinate (ptJ) at (0, 1, 0);
    \coordinate (ptK) at (0, 0, 1);
\end{axis}

\draw[dkgreenvect, thick, -latex] (origin)--(ptI) node [above, text=black, midway] {\small $\vect{i}$};
\draw[dkgreenvect, thick, -latex] (origin)--(ptJ) node [above, text=black, midway] {\small $\vect{j}$};
\draw[dkgreenvect, thick, -latex] (origin)--(ptK) node [right, text=black, midway] {\small $\vect{k}$};

\fill (ptI) circle (2pt);
\fill (ptJ) circle (2pt);
\fill (ptK) circle (2pt);
\node[below] at (origin) {\small $O$};
\end{tikzpicture}

\caption{\label{fig:012149}}
\end{wrapfigure}

\noindent (Because it is a vector, $\vect{v}_{1} \times \vect{v}_{2}$ is often called the \textbf{vector product}\index{vector product}.) There is an easy way to remember this definition using the \textbf{coordinate vectors}\index{vectors!coordinate vectors}\index{coordinate vectors}\index{cross product!coordinate vectors}:
\begin{equation*}
\vect{i}= \leftB
\begin{array}{c}
1\\
0 \\
0
\end{array}
\rightB, \
\vect{j}= \leftB
\begin{array}{c}
0\\
1 \\
0
\end{array}
\rightB,
\mbox{ and }
\vect{k}= \leftB
\begin{array}{c}
0\\
0 \\
1
\end{array}
\rightB
\end{equation*}
They are vectors of length $1$ pointing along the positive $x$, $y$, and $z$ axes, respectively, as in Figure~\ref{fig:012149}. The reason for the name is that any vector can be written as
\begin{equation*}
\leftB
\begin{array}{c}
x\\
y \\
z
\end{array}
\rightB
= x\vect{i} + y\vect{j} + z\vect{k}
\end{equation*}

With this, the cross product can be described as follows:


\begin{theorem*}{Determinant Form of the Cross Product}{012151}
If 
$\vect{v}_{1}= \leftB
\begin{array}{c}
x_{1}\\
y_{1}\\
z_{1}
\end{array}
\rightB$
and 
$\vect{v}_{2}= \leftB
\begin{array}{c}
x_{2}\\
y_{2}\\
z_{2}
\end{array}
\rightB$
 are two vectors, then
\begin{equation*}
\vect{v}_{1} \times \vect{v}_{2} = \func{det}\leftB
\begin{array}{ccc}
\vect{i} & x_{1} & x_{2}\\
\vect{j} & y_{1} & y_{2}\\
\vect{k} & z_{1} & z_{2}
\end{array}
\rightB
=
\left|
\begin{array}{cc}
y_{1} & y_{2}\\
z_{1} & z_{2}
\end{array}
\right|\vect{i}
-
\left|
\begin{array}{cc}
x_{1} & x_{2}\\
z_{1} & z_{2}
\end{array}
\right|\vect{j}
+
\left|
\begin{array}{cc}
x_{1} & x_{2}\\
y_{1} & y_{2}
\end{array}
\right|\vect{k}
\end{equation*}
where the determinant is expanded along the first column.\index{cross product!determinant form}\index{determinants!cross product}
\end{theorem*} 

\begin{example}{}{012157}
If $\vect{v} = \leftB
\begin{array}{r}
2\\
-1\\
4
\end{array}
\rightB$
and $\vect{w} = \leftB
\begin{array}{r}
1\\
3\\
7
\end{array}
\rightB$, then
\begin{align*}
\vect{v}_{1} \times \vect{v}_{2} = \func{det}\leftB
\begin{array}{rrr}
\vect{i} & 2 & 1\\
\vect{j} & -1 & 3\\
\vect{k} & 4 & 7
\end{array}
\rightB
&=
\left|
\begin{array}{rr}
-1 & 3\\
4 & 7
\end{array}
\right|\vect{i}
-
\left|
\begin{array}{rr}
2 & 1\\
4 & 7
\end{array}
\right|\vect{j}
+
\left|
\begin{array}{rr}
2 & 1\\
-1 & 3
\end{array}
\right|\vect{k}\\
&= -19\vect{i} - 10\vect{j} + 7\vect{k}\\
&= \leftB
\begin{array}{r}
-19\\
-10\\
7
\end{array}
\rightB
\end{align*}
\end{example}

Observe that $\vect{v} \times \vect{w}$ is orthogonal to both $\vect{v}$ and $\vect{w}$ in Example~\ref{exa:012157}. This holds in general as can be verified directly by computing $\vect{v} \dotprod (\vect{v} \times \vect{w})$ and $\vect{w} \dotprod (\vect{v} \times \vect{w})$, and is recorded as the first part of the following theorem. It will follow from a more general result which, together with the second part, will be proved in Section~\ref{sec:4_3} where a more detailed study of the cross product will be undertaken.

\begin{theorem}{}{012164}
Let $\vect{v}$ and $\vect{w}$ be vectors in $\RR^3$.


\begin{enumerate}
\item $\vect{v} \times \vect{w}$ is a vector orthogonal to both $\vect{v}$ and $\vect{w}$.

\item If $\vect{v}$ and $\vect{w}$ are nonzero, then $\vect{v} \times \vect{w} = \vect{0}$ if and only if $\vect{v}$ and $\vect{w}$ are parallel.

\end{enumerate}
\end{theorem}

\noindent It is interesting to contrast Theorem~\ref{thm:012164}(2) with the assertion (in Theorem~\ref{thm:011886}) that
\begin{equation*}
\vect{v} \dotprod \vect{w} = 0 \quad \mbox{ if and only if }\vect{v}\mbox{ and }\vect{w}\mbox{ are orthogonal.}
\end{equation*}
\begin{example}{}{012175}
Find the equation of the plane through $P(1, 3, -2)$, $Q(1, 1, 5)$, and $R(2, -2, 3)$.

\begin{solution}
The vectors 
$\longvect{PQ} = \leftB
\begin{array}{r}
0\\
-2\\
7
\end{array}
\rightB$
and 
$\longvect{PR} = \leftB
\begin{array}{r}
1\\
-5\\
5
\end{array}
\rightB$
lie in the plane, so
\begin{equation*}
\longvect{PQ} \times \longvect{PR} = \func{det}\leftB
\begin{array}{rrr}
\vect{i} & 0 & 1\\
\vect{j} & -2 & -5\\
\vect{k} & 7 & 5
\end{array}
\rightB
= 25\vect{i} + 7\vect{j} + 2\vect{k}
= \leftB
\begin{array}{r}
25\\
7\\
2
\end{array}
\rightB
\end{equation*}
is a normal for the plane (being orthogonal to both $\longvect{PQ}$ and $\longvect{PR}$). Hence the plane has equation
\begin{equation*}
25x + 7y + 2z = d  \quad \mbox{ for some number }d.
\end{equation*}
Since $P(1, 3, -2)$ lies in the plane we have $25 \cdot 1 + 7 \cdot 3 + 2(-2) = d$. Hence $d = 42$ and the equation is $25x + 7y + 2z = 42$. Incidentally, the same equation is obtained (verify) if $\longvect{QP}$ and $\longvect{QR}$, or $\longvect{RP}$ and $\longvect{RQ}$, are used as the vectors in the plane.
\end{solution}
\end{example}



\begin{example}{}{012186}
Find the shortest distance between the nonparallel lines\index{cross product!shortest distance between nonparallel lines}\index{line!shortest distance between nonparallel lines}
\begin{equation*}
\leftB
\begin{array}{c}
x\\
y\\
z
\end{array}
\rightB
=
\leftB
\begin{array}{r}
1\\
0\\
-1
\end{array}
\rightB
+t
\leftB
\begin{array}{r}
2\\
0\\
1
\end{array}
\rightB \quad
\mbox{ and } \quad
\leftB
\begin{array}{c}
x\\
y\\
z
\end{array}
\rightB
=
\leftB
\begin{array}{r}
3\\
1\\
0
\end{array}
\rightB
+s
\leftB
\begin{array}{r}
1\\
1\\
-1
\end{array}
\rightB
\end{equation*}
Then find the points $A$ and $B$ on the lines that are closest together.



\begin{solution}Direction vectors for the two lines are $\vect{d}_{1} = \leftB
\begin{array}{r}
2\\
0\\
1
\end{array}
\rightB$
and 
$\vect{d}_{2} = \leftB
\begin{array}{r}
1\\
1\\
-1
\end{array}
\rightB$, so
\begin{equation*}
\vect{n} = \vect{d}_{1} \times \vect{d}_{2} = \func{det}\leftB
\begin{array}{rrr}
\vect{i} & 2 & 1\\
\vect{j} & 0 & 1\\
\vect{k} & 1 & -1
\end{array}
\rightB
= \leftB
\begin{array}{r}
-1\\
3\\
2
\end{array}
\rightB
\end{equation*}
\begin{wrapfigure}[7]{l}{5cm} 
\vspace*{-2em}
\centering
\begin{tikzpicture}[scale=0.9]
\coordinate (ptA) at (0, 0);
\coordinate (ptB) at (0, 1);
\coordinate (ptN) at (0, 2);
\coordinate (ptP1) at (0.5, -0.25);
\coordinate (ptP2) at (1, 1.5);
\coordinate (ptP2Perp) at (1, 0.5);
\coordinate (ptPlaneLeft) at (-2, 0.5);

\draw[color=dkbluevect, fill=ltbluevect, thick] (ptPlaneLeft)-- ++(2.5, -1.25) -- ++(2, 1) -- ++(-2.5, 1.25)--cycle;

%blue lines
\draw[dkbluevect, thick, -latex] (ptA)--(ptN);
\draw[dkbluevect, thick] (ptPlaneLeft)+(-0.5, +0.75)-- (2, -1); %m = -1/2
\draw[dkbluevect, thick] (ptP2)+(0.5, 0.25)-- ++(-3, -1.5); %line through P2 and B, m = 1/2
\draw[dkbluevect, thick] (ptP2Perp)--(ptP2);

%right angles
{
\def \angleWidth {0.2}
\def \rightAngleX {0.179} %x change for right angles BAP1 and P2_A
\draw[dkbluevect, thick] (ptA)++(0, \angleWidth)-- ++(\rightAngleX, -{\rightAngleX / 2}) -- ++(0, -\angleWidth); %BAP1
\draw[dkbluevect, thick] (ptP2Perp)++(0, \angleWidth)-- ++(-\rightAngleX, {-\rightAngleX / 2}) -- ++(0, -\angleWidth); %AP2
\def \rightAnglePP {0.111} %m for p1 to point under p2 = 1.5.

\draw[dkbluevect, thick] (ptP2Perp)++(0, \angleWidth)-- ++( -\rightAnglePP, -\rightAnglePP * 1.5) -- ++(0, -\angleWidth); %P1P2
}

%dashed lines
\draw[dkbluevect, thick, dashed] (ptA)--(ptP2Perp)
	(ptP1)--(ptP2Perp);

%u line
\draw[dkgreenvect, thick, -latex] (ptP1)--(ptP2) node[left, text=black, midway] {\footnotesize $\vect{u}$};

\fill (ptA) circle (2pt);
\fill (ptB) circle (2pt);
\fill (ptP1) circle (2pt);
\fill (ptP2) circle (2pt);

\node[above] at (ptP2) {\footnotesize $P_2$};
\node[right] at (ptN) {\footnotesize $\vect{n}$};
\path (ptB)-- ++(-0.2, 0.1) node {\footnotesize $B$};
\path (ptA)-- ++(-0.2, -0.1) node {\footnotesize $A$};
\node[below] at (ptP1) {\footnotesize $P_1$};
\end{tikzpicture}
%\captionof{figure}{\label{fig:012204}}
\end{wrapfigure}

\setlength{\rightskip}{0pt plus 200pt} 
is perpendicular to both lines. Consider the plane shaded in the diagram containing the first line with $\vect{n}$ as normal. This plane contains $P_{1}(1, 0, -1)$ and is parallel to the second line. Because $P_{2}(3, 1, 0)$ is on the second line, the distance in question is just the shortest distance between $P_{2}(3, 1, 0)$ and this plane. The vector $\vect{u}$ from $P_{1}$ to $P_{2}$ is $\vect{u} = \longvect{P_{1}P}_{2} = \leftB
\begin{array}{r}
2\\
1\\
1
\end{array}
\rightB$
and so, as in Example~\ref{exa:012097}, the distance is the length of the projection of $\vect{u}$ on $\vect{n}$.
\begin{equation*}
\mbox{distance }= \left\vectlength \frac{\vect{u} \dotprod \vect{n}}{\vectlength \vect{n} \vectlength^2}\vect{n} \right\vectlength = \frac{|\vect{u} \dotprod \vect{n}|}{\vectlength \vect{n} \vectlength} = \frac{3}{\sqrt{14}} = \frac{3\sqrt{14}}{14}
\end{equation*}

Note that it is necessary that $\vect{n} = \vect{d}_{1} \times \vect{d}_{2}$ be nonzero for this calculation to be possible. As is shown later (Theorem~\ref{thm:012738}), this is guaranteed by the fact that $\vect{d}_{1}$ and $\vect{d}_{2}$ are \textit{not} parallel.


The points $A$ and $B$ have coordinates $A(1 + 2t, 0, t - 1)$ and $B(3 + s, 1 + s, -s)$ for some $s$ and $t$, so $\longvect{AB} = \leftB
\begin{array}{c}
2 + s - 2t\\
1 + s\\
1 - s - t
\end{array}
\rightB$. This vector is orthogonal to both $\vect{d}_{1}$ and $\vect{d}_{2}$, and the conditions $\longvect{AB} \dotprod \vect{d}_{1} = 0$ and $\longvect{AB} \dotprod \vect{d}_{2} = 0$ give equations $5t - s = 5$ and $t -3s = 2$. The solution is $s = \frac{-5}{14}$
 and $t = \frac{13}{14}$, so the points are $A(\frac{40}{14}, 0, \frac{-1}{14})$ and $B(\frac{37}{14}, \frac{9}{14}, \frac{5}{14})$. We have $\vectlength \longvect{AB} \vectlength = \frac{3\sqrt{14}}{14}$, as before.
\end{solution}
\end{example}


\section*{Exercises for \ref{sec:4_2}}

\begin{Filesave}{solutions}
\solsection{Section~\ref{sec:4_2}}
\end{Filesave}

\begin{multicols}{2}
\begin{ex}
Compute $\vect{u} \dotprod \vect{v}$ where:

\begin{enumerate}[label={\alph*.}]
\item $\vect{u} = \leftB
\begin{array}{r}
2\\
-1\\
3
\end{array}
\rightB$, 
$\vect{v} = \leftB
\begin{array}{r}
-1\\
1\\
1
\end{array}
\rightB
$

\item $\vect{u} = \leftB
\begin{array}{r}
1\\
2\\
-1
\end{array}
\rightB$,
$\vect{v} = \vect{u}
$


\item $\vect{u} = \leftB
\begin{array}{r}
1\\
1\\
-3
\end{array}
\rightB$, 
$\vect{v} = \leftB
\begin{array}{r}
2\\
-1\\
1
\end{array}
\rightB
$

\item $\vect{u} = \leftB
\begin{array}{r}
3\\
-1\\
5
\end{array}
\rightB$, 
$\vect{v} = \leftB
\begin{array}{r}
6\\
-7\\
-5
\end{array}
\rightB
$

\item$\vect{u} = \leftB
\begin{array}{r}
x\\
y\\
z
\end{array}
\rightB$, 
$\vect{v} = \leftB
\begin{array}{r}
a\\
b\\
c
\end{array}
\rightB
$

\item $\vect{u} = \leftB
\begin{array}{r}
a\\
b\\
c
\end{array}
\rightB$, 
$\vect{v} = \vect{0} $

\end{enumerate}
\begin{sol}
\begin{enumerate}[label={\alph*.}]
\setcounter{enumi}{1}
\item  $6$

\setcounter{enumi}{3}
\item  $0$

\setcounter{enumi}{5}
\item  $0$

\end{enumerate}
\end{sol}
\end{ex}

\begin{ex}
Find the angle between the following pairs of vectors.


\begin{enumerate}[label={\alph*.}]
\item $\vect{u} = \leftB
\begin{array}{r}
1\\
0\\
3
\end{array}
\rightB$, 
$\vect{v} = \leftB
\begin{array}{r}
2\\
0\\
1
\end{array}
\rightB
$

\item $\vect{u} = \leftB
\begin{array}{r}
3\\
-1\\
0
\end{array}
\rightB$, 
$\vect{v} = \leftB
\begin{array}{r}
-6\\
2\\
0
\end{array}
\rightB
$

\item $\vect{u} = \leftB
\begin{array}{r}
7\\
-1\\
3
\end{array}
\rightB$, 
$\vect{v} = \leftB
\begin{array}{r}
1\\
4\\
-1
\end{array}
\rightB
$

\item $\vect{u} = \leftB
\begin{array}{r}
2\\
1\\
-1
\end{array}
\rightB$, 
$\vect{v} = \leftB
\begin{array}{r}
3\\
6\\
3
\end{array}
\rightB
$

\item $\vect{u} = \leftB
\begin{array}{r}
1\\
-1\\
0
\end{array}
\rightB$, 
$\vect{v} = \leftB
\begin{array}{r}
0\\
1\\
1
\end{array}
\rightB
$

\item $\vect{u} = \leftB
\begin{array}{r}
0\\
3\\
4
\end{array}
\rightB$, 
$\vect{v} = \leftB
\begin{array}{r}
5\sqrt{2}\\
-7\\
-1
\end{array}
\rightB
$


\end{enumerate}
\begin{sol}
\begin{enumerate}[label={\alph*.}]
\setcounter{enumi}{1}
\item  $\pi$ or $180^\circ$ 

\setcounter{enumi}{3}
\item  $\frac{\pi}{3}$ or $60^\circ$

\setcounter{enumi}{5}
\item  $\frac{2\pi}{3}$ or $120^\circ$


\end{enumerate}
\end{sol}
\end{ex}

\begin{ex}
Find all real numbers $x$ such that:


\begin{enumerate}[label={\alph*.}]
\item $\leftB
\begin{array}{r}
2\\
-1\\
3
\end{array}
\rightB$
and 
$\leftB
\begin{array}{r}
x\\
-2\\
1
\end{array}
\rightB$
are orthogonal.

\item $\leftB
\begin{array}{r}
2\\
-1\\
1
\end{array}
\rightB$
and 
$\leftB
\begin{array}{r}
1\\
x\\
2
\end{array}
\rightB$
are at an angle of $\frac{\pi}{3}$.

\end{enumerate}
\begin{sol}
\begin{enumerate}[label={\alph*.}]
\setcounter{enumi}{1}
\item  $1$ or $-17$

\end{enumerate}
\end{sol}
\end{ex}

\begin{ex}
Find all vectors $\vect{v} = \leftB
\begin{array}{r}
x\\
y\\
z
\end{array}
\rightB$ orthogonal to both:


\begin{enumerate}[label={\alph*.}]
\item $\vect{u}_{1} = \leftB
\begin{array}{r}
-1\\
-3\\
2
\end{array}
\rightB$,
$\vect{u}_{2} = \leftB
\begin{array}{r}
0\\
1\\
1
\end{array}
\rightB
$

\item $\vect{u}_{1} = \leftB
\begin{array}{r}
3\\
-1\\
2
\end{array}
\rightB$, 
$\vect{u}_{2} = \leftB
\begin{array}{r}
2\\
0\\
1
\end{array}
\rightB
$

\item $\vect{u}_{1} = \leftB
\begin{array}{r}
2\\
0\\
-1
\end{array}
\rightB$, 
$\vect{u}_{2} = \leftB
\begin{array}{r}
-4\\
0\\
2
\end{array}
\rightB
$

\item $\vect{u}_{1} = \leftB
\begin{array}{r}
2\\
-1\\
3
\end{array}
\rightB$, 
$\vect{u}_{2} = \leftB
\begin{array}{r}
0\\
0\\
0
\end{array}
\rightB
$


\end{enumerate}
\begin{sol}
\begin{enumerate}[label={\alph*.}]
\setcounter{enumi}{1}
\item  $t \leftB
\begin{array}{r}
-1\\
1\\
2
\end{array}
\rightB
$

\setcounter{enumi}{3}
\item $s \leftB
\begin{array}{r}
1\\
2\\
0
\end{array}
\rightB
+
t \leftB
\begin{array}{r}
0\\
3\\
1
\end{array}
\rightB
$


\end{enumerate}
\end{sol}
\end{ex}

\begin{ex}
Find two orthogonal vectors that are both orthogonal to $\vect{v} = \leftB
\begin{array}{r}
1\\
2\\
0
\end{array}
\rightB$.
\end{ex}

\begin{ex}
Consider the triangle with vertices $P(2, 0, -3)$, $Q(5, -2, 1)$, and $R(7, 5, 3)$.


\begin{enumerate}[label={\alph*.}]
\item Show that it is a right-angled triangle.

\item Find the lengths of the three sides and verify the Pythagorean theorem.

\end{enumerate}
\begin{sol}
\begin{enumerate}[label={\alph*.}]
\setcounter{enumi}{1}
\item  $29 + 57 = 86$

\end{enumerate}
\end{sol}
\end{ex}

\begin{ex}
Show that the triangle with vertices $A(4, -7, 9)$, $B(6, 4, 4)$, and $C(7, 10, -6)$ is not a right-angled triangle.
\end{ex}

\begin{ex}
Find the three internal angles of the triangle with vertices:


\begin{enumerate}[label={\alph*.}]
\item $A(3, 1, -2)$, $B(3, 0, -1)$, and $C(5, 2, -1)$

\item $A(3, 1, -2)$, $B(5, 2, -1)$, and $C(4, 3, -3)$

\end{enumerate}
\begin{sol}
\begin{enumerate}[label={\alph*.}]
\setcounter{enumi}{1}
\item $A = B = C = \frac{\pi}{3}$ or $60^\circ$ 

\end{enumerate}
\end{sol}
\end{ex}

\begin{ex}
Show that the line through $P_{0}(3, 1, 4)$ and $P_{1}(2, 1, 3)$ is perpendicular to the line through $P_{2}(1, -1, 2)$ and $P_{3}(0, 5, 3)$.
\end{ex}

\begin{ex}
In each case, compute the projection of $\vect{u}$ on $\vect{v}$.


\begin{enumerate}[label={\alph*.}]
\item $\vect{u} = \leftB
\begin{array}{r}
5\\
7\\
1
\end{array}
\rightB$, 
$\vect{v} = \leftB
\begin{array}{r}
2\\
-1\\
3
\end{array}
\rightB
$

\item $\vect{u} = \leftB
\begin{array}{r}
3\\
-2\\
1
\end{array}
\rightB$, 
$\vect{v} = \leftB
\begin{array}{r}
4\\
1\\
1
\end{array}
\rightB
$

\item $\vect{u} = \leftB
\begin{array}{r}
1\\
-1\\
2
\end{array}
\rightB$, 
$\vect{v} = \leftB
\begin{array}{r}
3\\
-1\\
1
\end{array}
\rightB
$

\item $\vect{u} = \leftB
\begin{array}{r}
3\\
-2\\
-1
\end{array}
\rightB$, 
$\vect{v} = \leftB
\begin{array}{r}
-6\\
4\\
2
\end{array}
\rightB
$

\end{enumerate}
\begin{sol}
\begin{enumerate}[label={\alph*.}]
\setcounter{enumi}{1}
\item  $\frac{11}{18}\vect{v}$

\setcounter{enumi}{3}
\item  $-\frac{1}{2}\vect{v}$

\end{enumerate}
\end{sol}
\end{ex}

\begin{ex}
In each case, write $\vect{u} = \vect{u}_{1} + \vect{u}_{2}$, where $\vect{u}_{1}$ is parallel to $\vect{v}$ and $\vect{u}_{2}$ is orthogonal to $\vect{v}$.

\begin{enumerate}[label={\alph*.}]
\item $\vect{u} = \leftB
\begin{array}{r}
2\\
-1\\
1
\end{array}
\rightB$, 
$\vect{v} = \leftB
\begin{array}{r}
1\\
-1\\
3
\end{array}
\rightB
$

\item $\vect{u} = \leftB
\begin{array}{r}
3\\
1\\
0
\end{array}
\rightB$, 
$\vect{v} = \leftB
\begin{array}{r}
-2\\
1\\
4
\end{array}
\rightB
$

\item $\vect{u} = \leftB
\begin{array}{r}
2\\
-1\\
0
\end{array}
\rightB$, 
$\vect{v} = \leftB
\begin{array}{r}
3\\
1\\
-1
\end{array}
\rightB
$

\item $\vect{u} = \leftB
\begin{array}{r}
3\\
-2\\
1
\end{array}
\rightB$, 
$\vect{v} = \leftB
\begin{array}{r}
-6\\
4\\
-1
\end{array}
\rightB
$

\end{enumerate}
\begin{sol}
\begin{enumerate}[label={\alph*.}]
\setcounter{enumi}{1}
\item  $\frac{5}{21}\leftB
\begin{array}{r}
2\\
-1\\
-4
\end{array}
\rightB
+
\frac{1}{21}\leftB
\begin{array}{r}
53\\
26\\
20
\end{array}
\rightB$

\setcounter{enumi}{3}
\item  
$\frac{27}{53}\leftB
\begin{array}{r}
6\\
-4\\
1
\end{array}
\rightB
+
\frac{1}{53}\leftB
\begin{array}{r}
-3\\
2\\
26
\end{array}
\rightB$


\end{enumerate}
\end{sol}
\end{ex}

\begin{ex}
Calculate the distance from the point $P$ to the line in each case and find the point $Q$ on the line closest to $P$.


\begin{enumerate}[label={\alph*.}]
\item $P(3,2-1) \quad $ \\ line:  $\leftB
\begin{array}{r}
x\\
y\\
z
\end{array}
\rightB
=
\leftB
\begin{array}{r}
2\\
1\\
3
\end{array}
\rightB
+t
\leftB
\begin{array}{r}
3\\
-1\\
-2
\end{array}
\rightB
$

\item $P(1,-1,3) \quad $ \\ line: $ \leftB
\begin{array}{r}
x\\
y\\
z
\end{array}
\rightB
=
\leftB
\begin{array}{r}
1\\
0\\
-1
\end{array}
\rightB
+t
\leftB
\begin{array}{r}
3\\
1\\
4
\end{array}
\rightB
$

\end{enumerate}
\begin{sol}
\begin{enumerate}[label={\alph*.}]
\setcounter{enumi}{1}
\item  $\frac{1}{26}\sqrt{5642}$, $Q(\frac{71}{26}, \frac{15}{26}, \frac{34}{26})$


\end{enumerate}
\end{sol}
\end{ex}

\begin{ex}
Compute $\vect{u} \times \vect{v}$ where:


\begin{enumerate}[label={\alph*.}]
\item $\vect{u} = \leftB
\begin{array}{r}
1\\
2\\
3
\end{array}
\rightB$, 
$\vect{v} = \leftB
\begin{array}{r}
1\\
1\\
2
\end{array}
\rightB
$

\item $\vect{u} = \leftB
\begin{array}{r}
3\\
-1\\
0
\end{array}
\rightB$, 
$\vect{v} = \leftB
\begin{array}{r}
-6\\
2\\
0
\end{array}
\rightB
$

\item $\vect{u} = \leftB
\begin{array}{r}
3\\
-2\\
1
\end{array}
\rightB$, 
$\vect{v} = \leftB
\begin{array}{r}
1\\
1\\
-1
\end{array}
\rightB
$

\item $\vect{u} = \leftB
\begin{array}{r}
2\\
0\\
-1
\end{array}
\rightB$, 
$\vect{v} = \leftB
\begin{array}{r}
1\\
4\\
7
\end{array}
\rightB
$

\end{enumerate}
\begin{sol}
\begin{enumerate}[label={\alph*.}]
\setcounter{enumi}{1}
\item  
$\leftB
\begin{array}{r}
0\\
0\\
0
\end{array}
\rightB
$


\setcounter{enumi}{1}
\item 
$\leftB
\begin{array}{r}
4\\
-15\\
8
\end{array}
\rightB
$

\end{enumerate}
\end{sol}
\end{ex}

\begin{ex}
Find an equation of each of the following planes.


\begin{enumerate}[label={\alph*.}]
\item Passing through $A(2, 1, 3)$, $B(3, -1, 5)$, and $C(1, 2, -3)$.

\item Passing through $A(1, -1, 6)$, $B(0, 0, 1)$, and $C(4, 7, -11)$.

\item Passing through $P(2, -3, 5)$ and parallel to the plane with equation $3x - 2y - z = 0$.

\item Passing through $P(3, 0, -1)$ and parallel to the plane with equation $2x - y + z = 3$.

\item Containing $P(3, 0, -1)$ and the line
\newline $\leftB
\begin{array}{r}
x\\
y\\
z
\end{array}
\rightB
=
\leftB
\begin{array}{r}
0\\
0\\
2
\end{array}
\rightB
+t
\leftB
\begin{array}{r}
1\\
0\\
1
\end{array}
\rightB.
$
\item Containing $P(2, 1, 0)$ and the line

$
\leftB
\begin{array}{r}
x\\
y\\
z
\end{array}
\rightB
=
\leftB
\begin{array}{r}
3\\
-1\\
2
\end{array}
\rightB
+t
\leftB
\begin{array}{r}
1\\
0\\
-1
\end{array}
\rightB.$
\item Containing the lines

$\leftB
\begin{array}{r}
x\\
y\\
z
\end{array}
\rightB
=
\leftB
\begin{array}{r}
1\\
-1\\
2
\end{array}
\rightB
+t
\leftB
\begin{array}{r}
1\\
1\\
1
\end{array}
\rightB 
$ and
\newline $\leftB
\begin{array}{r}
x\\
y\\
z
\end{array}
\rightB
=
\leftB
\begin{array}{r}
0\\
0\\
2
\end{array}
\rightB
+t
\leftB
\begin{array}{r}
1\\
-1\\
0
\end{array}
\rightB$.

\item Containing the lines $\leftB
\begin{array}{r}
x\\
y\\
z
\end{array}
\rightB
=
\leftB
\begin{array}{r}
3\\
1\\
0
\end{array}
\rightB
+t
\leftB
\begin{array}{r}
1\\
-1\\
3
\end{array}
\rightB 
$ and
$\leftB
\begin{array}{r}
x\\
y\\
z
\end{array}
\rightB
=
\leftB
\begin{array}{r}
0\\
-2\\
5
\end{array}
\rightB
+t
\leftB
\begin{array}{r}
2\\
1\\
-1
\end{array}
\rightB$.

\item Each point of which is equidistant from $P(2, -1, 3)$ and $Q(1, 1, -1)$.

\item Each point of which is equidistant from $P(0, 1, -1)$ and $Q(2, -1, -3)$.

\end{enumerate}
\begin{sol}
\begin{enumerate}[label={\alph*.}]
\setcounter{enumi}{1}
\item  $-23x + 32y + 11z = 11$

\setcounter{enumi}{3}
\item  $2x - y + z = 5$

\setcounter{enumi}{5}
\item  $2x + 3y + 2z = 7$

\setcounter{enumi}{7}
\item  $2x - 7y - 3z = -1$

\setcounter{enumi}{9}
\item  $x - y - z = 3$

\end{enumerate}
\end{sol}
\end{ex}

\begin{ex}
In each case, find a vector equation of the line.


\begin{enumerate}[label={\alph*.}]
\item Passing through $P(3, -1, 4)$ and perpendicular to the plane $3x - 2y - z = 0$.

\item Passing through $P(2, -1, 3)$ and perpendicular to the plane $2x + y = 1$.

\item Passing through $P(0, 0, 0)$ and perpendicular to the lines
$\leftB
\begin{array}{r}
x\\
y\\
z
\end{array}
\rightB
=
\leftB
\begin{array}{r}
1\\
1\\
0
\end{array}
\rightB
+t
\leftB
\begin{array}{r}
2\\
0\\
-1
\end{array}
\rightB 
$ and
$\leftB
\begin{array}{r}
x\\
y\\
z
\end{array}
\rightB
=
\leftB
\begin{array}{r}
2\\
1\\
-3
\end{array}
\rightB
+t
\leftB
\begin{array}{r}
1\\
-1\\
5
\end{array}
\rightB$.
\item Passing through $P(1, 1, -1)$, and perpendicular to the lines

$\leftB
\begin{array}{r}
x\\
y\\
z
\end{array}
\rightB
=
\leftB
\begin{array}{r}
2\\
0\\
1
\end{array}
\rightB
+t
\leftB
\begin{array}{r}
1\\
1\\
-2
\end{array}
\rightB 
$ and
 \\$\leftB
\begin{array}{r}
x\\
y\\
z
\end{array}
\rightB
=
\leftB
\begin{array}{r}
5\\
5\\
-2
\end{array}
\rightB
+t
\leftB
\begin{array}{r}
1\\
2\\
-3
\end{array}
\rightB$.

\item Passing through $P(2, 1, -1)$, intersecting the line 
$\leftB
\begin{array}{r}
x\\
y\\
z
\end{array}
\rightB
=
\leftB
\begin{array}{r}
1\\
2\\
-1
\end{array}
\rightB
+t
\leftB
\begin{array}{r}
3\\
0\\
1
\end{array}
\rightB$, and perpendicular to that line.

\item Passing through $P(1, 1, 2)$, intersecting the line 
$\leftB
\begin{array}{r}
x\\
y\\
z
\end{array}
\rightB
=
\leftB
\begin{array}{r}
2\\
1\\
0
\end{array}
\rightB
+t
\leftB
\begin{array}{r}
1\\
1\\
1
\end{array}
\rightB$, and perpendicular to that line.

\end{enumerate}
\begin{sol}
\begin{enumerate}[label={\alph*.}]
\setcounter{enumi}{1}
\item  
$\leftB
\begin{array}{r}
x\\
y\\
z
\end{array}
\rightB
=
\leftB
\begin{array}{r}
2\\
-1\\
3
\end{array}
\rightB
+t
\leftB
\begin{array}{r}
2\\
1\\
0
\end{array}
\rightB
$

\setcounter{enumi}{3}
\item 
$\leftB
\begin{array}{r}
x\\
y\\
z
\end{array}
\rightB
=
\leftB
\begin{array}{r}
1\\
1\\
-1
\end{array}
\rightB
+t
\leftB
\begin{array}{r}
1\\
1\\
1
\end{array}
\rightB
$

\setcounter{enumi}{5}
\item  
$\leftB
\begin{array}{r}
x\\
y\\
z
\end{array}
\rightB
=
\leftB
\begin{array}{r}
1\\
1\\
2
\end{array}
\rightB
+t
\leftB
\begin{array}{r}
4\\
1\\
-5
\end{array}
\rightB
$

\end{enumerate}
\end{sol}
\end{ex}

\begin{ex}
In each case, find the shortest distance from the point $P$ to the plane and find the point $Q$ on the plane closest to $P$.


\begin{enumerate}[label={\alph*.}]
\item $P(2, 3, 0)$; plane with equation $5x + y + z = 1$.

\item $P(3, 1, -1)$; plane with equation $2x + y - z = 6$.

\end{enumerate}
\begin{sol}
\begin{enumerate}[label={\alph*.}]
\setcounter{enumi}{1}
\item  $\frac{\sqrt{6}}{3}$, $Q(\frac{7}{3}, \frac{2}{3}, \frac{-2}{3})$

\end{enumerate}
\end{sol}
\end{ex}

\begin{ex}
\begin{enumerate}[label={\alph*.}]
\item Does the line through $P(1, 2, -3)$ with direction vector $\vect{d} = \leftB
\begin{array}{r}
1\\
2\\
-3
\end{array}
\rightB$ lie in the plane $2x - y - z = 3$? Explain.

\item Does the plane through $P(4, 0, 5)$, $Q(2, 2, 1)$, and $R(1, -1, 2)$ pass through the origin? Explain.

\end{enumerate}
\begin{sol}
\begin{enumerate}[label={\alph*.}]
\setcounter{enumi}{1}
\item  Yes. The equation is $5x -3y - 4z = 0$.

\end{enumerate}
\end{sol}
\end{ex}

\begin{ex}
Show that every plane containing $P(1, 2, -1)$ and $Q(2, 0, 1)$ must also contain $R(-1, 6, -5)$.
\end{ex}

\begin{ex}
Find the equations of the line of intersection of the following planes.


\begin{enumerate}[label={\alph*.}]
\item $2x -3y + 2z = 5$ and $x + 2y - z = 4$.

\item $3x + y -2z = 1$ and $x + y + z = 5$.

\end{enumerate}
\begin{sol}
\begin{enumerate}[label={\alph*.}]
\setcounter{enumi}{1}
\item  $(-2, 7, 0) + t(3, -5, 2)$

\end{enumerate}
\end{sol}
\end{ex}

\begin{ex}
In each case, find all points of intersection of the given plane and the line \newline$\leftB
\begin{array}{r}
x\\
y\\
z
\end{array}
\rightB
=
\leftB
\begin{array}{r}
1\\
-2\\
3
\end{array}
\rightB
+t
\leftB
\begin{array}{r}
2\\
5\\
-1
\end{array}
\rightB$.

\begin{exenumerate}
\exitem $x -3y + 2z = 4$
\exitem $2x - y - z = 5$
\exitem $3x - y + z = 8$
\exitem $-x -4y -3z = 6$
\end{exenumerate}
\begin{sol}
\begin{enumerate}[label={\alph*.}]
\setcounter{enumi}{1}
\item  None

\setcounter{enumi}{3}
\item  $P(\frac{13}{19}, \frac{-78}{19}, \frac{65}{19})$


\end{enumerate}
\end{sol}
\end{ex}

\begin{ex}
Find the equation of \textit{all} planes:


\begin{enumerate}[label={\alph*.}]
\item Perpendicular to the line  \\
$\leftB
\begin{array}{r}
x\\
y\\
z
\end{array}
\rightB
=
\leftB
\begin{array}{r}
2\\
-1\\
3
\end{array}
\rightB
+t
\leftB
\begin{array}{r}
2\\
1\\
3
\end{array}
\rightB$.

\item Perpendicular to the line  \\
$\leftB
\begin{array}{r}
x\\
y\\
z
\end{array}
\rightB
=
\leftB
\begin{array}{r}
1\\
0\\
-1
\end{array}
\rightB
+t
\leftB
\begin{array}{r}
3\\
0\\
2
\end{array}
\rightB$.

\item Containing the origin.

\item Containing $P(3, 2, -4)$.

\item Containing $P(1, 1, -1)$ and $Q(0, 1, 1)$.

\item Containing $P(2, -1, 1)$ and $Q(1, 0, 0)$.

\item Containing the line \\
$\leftB
\begin{array}{r}
x\\
y\\
z
\end{array}
\rightB
=
\leftB
\begin{array}{r}
2\\
1\\
0
\end{array}
\rightB
+t
\leftB
\begin{array}{r}
1\\
-1\\
0
\end{array}
\rightB$.

\item Containing the line \\
$\leftB
\begin{array}{r}
x\\
y\\
z
\end{array}
\rightB
=
\leftB
\begin{array}{r}
3\\
0\\
2
\end{array}
\rightB
+t
\leftB
\begin{array}{r}
1\\
-2\\
-1
\end{array}
\rightB$.

\end{enumerate}
\begin{sol}
\begin{enumerate}[label={\alph*.}]
\setcounter{enumi}{1}
\item  $3x + 2z = d$, $d$ arbitrary

\setcounter{enumi}{3}
\item  $a(x - 3) + b(y - 2) + c(z + 4) = 0$; $a$, $b$, and $c$ not all zero

\setcounter{enumi}{5}
\item  $ax + by + (b - a)z = a$; $a$ and $b$ not both zero

\setcounter{enumi}{7}
\item $ax + by + (a - 2b)z = 5a - 4b$; $a$ and $b$ not both zero

\end{enumerate}
\end{sol}
\end{ex}

\begin{ex}
If a plane contains two distinct points $P_{1}$ and $P_{2}$, show that it contains every point on the line through $P_{1}$ and $P_{2}$.
\end{ex}

\begin{ex}
Find the shortest distance between the following pairs of parallel lines.

\begin{enumerate}[label={\alph*.}]
\item $\leftB
\begin{array}{r}
x\\
y\\
z
\end{array}
\rightB
=
\leftB
\begin{array}{r}
2\\
-1\\
3
\end{array}
\rightB
+t
\leftB
\begin{array}{r}
1\\
-1\\
4
\end{array}
\rightB; $ \\ $
\leftB
\begin{array}{r}
x\\
y\\
z
\end{array}
\rightB
=
\leftB
\begin{array}{r}
1\\
0\\
1
\end{array}
\rightB
+t
\leftB
\begin{array}{r}
1\\
-1\\
4
\end{array}
\rightB$

\item 
$\leftB
\begin{array}{r}
x\\
y\\
z
\end{array}
\rightB
=
\leftB
\begin{array}{r}
3\\
0\\
2
\end{array}
\rightB
+t
\leftB
\begin{array}{r}
3\\
1\\
0
\end{array}
\rightB; $ \\ $
\leftB
\begin{array}{r}
x\\
y\\
z
\end{array}
\rightB
=
\leftB
\begin{array}{r}
-1\\
2\\
2
\end{array}
\rightB
+t
\leftB
\begin{array}{r}
3\\
1\\
0
\end{array}
\rightB$

\end{enumerate}
\begin{sol}
\begin{enumerate}[label={\alph*.}]
\setcounter{enumi}{1}
\item  $\sqrt{10}$

\end{enumerate}
\end{sol}
\end{ex}

\begin{ex}
Find the shortest distance between the following pairs of nonparallel lines and find the points on the lines that are closest together.

\begin{enumerate}[label={\alph*.}]
\item 
$\leftB
\begin{array}{r}
x\\
y\\
z
\end{array}
\rightB
=
\leftB
\begin{array}{r}
3\\
0\\
1
\end{array}
\rightB
+s
\leftB
\begin{array}{r}
2\\
1\\
-3
\end{array}
\rightB; $ \\ $
\leftB
\begin{array}{r}
x\\
y\\
z
\end{array}
\rightB
=
\leftB
\begin{array}{r}
1\\
1\\
-1
\end{array}
\rightB
+t
\leftB
\begin{array}{r}
1\\
0\\
1
\end{array}
\rightB$

\item 
$\leftB
\begin{array}{r}
x\\
y\\
z
\end{array}
\rightB
=
\leftB
\begin{array}{r}
1\\
-1\\
0
\end{array}
\rightB
+s
\leftB
\begin{array}{r}
1\\
1\\
1
\end{array}
\rightB; $ \\ $
\leftB
\begin{array}{r}
x\\
y\\
z
\end{array}
\rightB
=
\leftB
\begin{array}{r}
2\\
-1\\
3
\end{array}
\rightB
+t
\leftB
\begin{array}{r}
3\\
1\\
0
\end{array}
\rightB$

\item 
$\leftB
\begin{array}{r}
x\\
y\\
z
\end{array}
\rightB
=
\leftB
\begin{array}{r}
3\\
1\\
-1
\end{array}
\rightB
+s
\leftB
\begin{array}{r}
1\\
1\\
-1
\end{array}
\rightB; $ \\ $
\leftB
\begin{array}{r}
x\\
y\\
z
\end{array}
\rightB
=
\leftB
\begin{array}{r}
1\\
2\\
0
\end{array}
\rightB
+t
\leftB
\begin{array}{r}
1\\
0\\
2
\end{array}
\rightB$

\item 
$\leftB
\begin{array}{r}
x\\
y\\
z
\end{array}
\rightB
=
\leftB
\begin{array}{r}
1\\
2\\
3
\end{array}
\rightB
+s
\leftB
\begin{array}{r}
2\\
0\\
-1
\end{array}
\rightB; $ \\ $
\leftB
\begin{array}{r}
x\\
y\\
z
\end{array}
\rightB
=
\leftB
\begin{array}{r}
3\\
-1\\
0
\end{array}
\rightB
+t
\leftB
\begin{array}{r}
1\\
1\\
0
\end{array}
\rightB$

\end{enumerate}
\begin{sol}
\begin{enumerate}[label={\alph*.}]
\setcounter{enumi}{1}
\item  $\frac{\sqrt{14}}{2}$, $A(3, 1, 2)$, $B(\frac{7}{2}, -\frac{1}{2}, 3)$

\setcounter{enumi}{3}
\item  $\frac{\sqrt{6}}{6}$, $A(\frac{19}{3}, 2, \frac{1}{3})$, $B(\frac{37}{6}, \frac{13}{6}, 0)$

\end{enumerate}
\end{sol}
\end{ex}

\begin{ex}
Show that two lines in the plane with slopes $m_{1}$ and $m_{2}$ are perpendicular if and only if \\ $m_{1}m_{2} = -1$. [\textit{Hint}: Example~\ref{exa:011343}.]
\end{ex}

\begin{ex}
\begin{enumerate}[label={\alph*.}]
\item Show that, of the four diagonals of a cube, no pair is perpendicular.

\item Show that each diagonal is perpendicular to the face diagonals it does not meet.

\end{enumerate}
\begin{sol}
\begin{enumerate}[label={\alph*.}]
\setcounter{enumi}{1}
\item  Consider the diagonal $\vect{d} = \leftB
\begin{array}{r}
	a\\
	a\\
	a
\end{array}
\rightB$ 
The six face diagonals in question are $\pm\leftB
\begin{array}{r}
a\\
0\\
-a
\end{array}
\rightB$, 
$\pm\leftB
\begin{array}{r}
0\\
a\\
-a
\end{array}
\rightB$, $
\pm\leftB
\begin{array}{r}
a\\
-a\\
0
\end{array}
\rightB$. All of these are orthogonal to $\vect{d}$. The result works for the other diagonals by symmetry.

\end{enumerate}
\end{sol}
\end{ex}

\begin{ex}
Given a rectangular solid with sides of lengths $1$, $1$, and $\sqrt{2}$, find the angle between a diagonal and one of the longest sides.
\end{ex}

\begin{ex}
Consider a rectangular solid with sides of lengths $a$, $b$, and $c$. Show that it has two orthogonal diagonals if and only if the sum of two of $a^{2}$, $b^{2}$, and $c^{2}$ equals the third.

\begin{sol}
The four diagonals are $(a, b, c)$, $(-a, b, c)$, $(a, -b, c)$ and $(a, b, -c)$ or their negatives. The dot products are $\pm(-a^{2} + b^{2} + c^{2})$, $\pm(a^{2} - b^{2} + c^{2})$, and $\pm(a^{2} + b^{2} - c^{2})$.
\end{sol}
\end{ex}

\begin{ex}
Let $A$, $B$, and $C(2, -1, 1)$ be the vertices of a triangle where $\longvect{AB}$ is parallel to $\leftB
\begin{array}{r}
1\\
-1\\
1
\end{array}
\rightB$, $\longvect{AC}$ is parallel to $\leftB
\begin{array}{r}
2\\
0\\
-1
\end{array}
\rightB$, and angle $C = 90^\circ$ . Find the equation of the line through $B$ and $C$.
\end{ex}

\begin{ex}
If the diagonals of a parallelogram have equal length, show that the parallelogram is a rectangle.
\end{ex}

\begin{ex}
Given $\vect{v} = \leftB
\begin{array}{r}
x\\
y\\
z
\end{array}
\rightB$
 in component form, show that the projections of $\vect{v}$ on $\vect{i}$, $\vect{j}$, and $\vect{k}$ are $x\vect{i}$, $y\vect{j}$, and $z\vect{k}$, respectively.
\end{ex}

\begin{ex}
\begin{enumerate}[label={\alph*.}]
\item Can $\vect{u} \dotprod \vect{v} = -7$ if $\vectlength\vect{u}\vectlength = 3$ and $\vectlength\vect{v}\vectlength = 2$? Defend your answer.

\item Find $\vect{u} \dotprod \vect{v}$ if $\vect{u} = \leftB
\begin{array}{r}
2\\
-1\\
2
\end{array}
\rightB$, $\vectlength \vect{v} \vectlength = 6$, and the angle between $\vect{u}$ and $\vect{v}$ is $\frac{2\pi}{3}$.

\end{enumerate}
\end{ex}

\begin{ex}
Show $(\vect{u} + \vect{v}) \dotprod (\vect{u} - \vect{v}) = \vectlength\vect{u}\vectlength^{2} - \vectlength\vect{v}\vectlength^{2}$ for any vectors $\vect{u}$ and $\vect{v}$.
\end{ex}

\begin{ex}
\begin{enumerate}[label={\alph*.}]
\item Show $\vectlength\vect{u} + \vect{v}\vectlength^{2} + \vectlength\vect{u} - \vect{v}\vectlength^{2} = 2(\vectlength\vect{u}\vectlength^{2} + \vectlength\vect{v}\vectlength^{2})$ for any vectors $\vect{u}$ and $\vect{v}$.

\item What does this say about parallelograms?

\end{enumerate}
\begin{sol}
\begin{enumerate}[label={\alph*.}]
\setcounter{enumi}{1}
\item  The sum of the squares of the lengths of the diagonals equals the sum of the squares of the lengths of the four sides.

\end{enumerate}
\end{sol}
\end{ex}

\begin{ex}
Show that if the diagonals of a parallelogram are perpendicular, it is necessarily a rhombus. [\textit{Hint}: Example~\ref{exa:011899}.]
\end{ex}

\begin{ex}
Let $A$ and $B$ be the end points of a diameter of a circle (see the diagram). If $C$ is any point on the circle, show that $AC$ and $BC$ are perpendicular. [\textit{Hint}: Express $\longvect{AB} \dotprod (\longvect{AB} \times \longvect{AC}) = 0$ and $\longvect{BC}$ in terms of $\vect{u} = \longvect{OA}$ and $\vect{v} = \longvect{OC}$, where $O$ is the centre.]


\begin{figure}[H]
\centering
\begin{tikzpicture}[scale=0.7]
\coordinate (ptO) at (0, 0);
\coordinate (ptA) at (-2, 0);
\coordinate (ptB) at (2, 0);

\draw[dkgreenvect, thick] (ptO)--(60:2cm) node(ptC){}
 (ptC.center)--(ptB)
 (ptB)--(ptA)
 (ptA)--(ptC.center);
\draw[dkbluevect, thick] (ptO) circle (2);

\node[below] at (ptO) {\small $O$};
\node[left] at (ptA) {\small $A$};
\node[right] at (ptB) {\small $B$};
\node[above] at (ptC) {\small $C$};
\end{tikzpicture}
	
%\caption{\label{fig:012573}}
\end{figure}
\end{ex}

\begin{ex}
Show that $\vect{u}$ and $\vect{v}$ are orthogonal, if and only if $\vectlength\vect{u} + \vect{v}\vectlength^{2} = \vectlength\vect{u}\vectlength^{2} + \vectlength\vect{v}\vectlength^{2}$.
\end{ex}

\begin{ex}
Let $\vect{u}$, $\vect{v}$, and $\vect{w}$ be pairwise orthogonal vectors.


\begin{enumerate}[label={\alph*.}]
\item Show that $\vectlength\vect{u} + \vect{v} + \vect{w}\vectlength^{2} = \vectlength\vect{u}\vectlength^{2} + \vectlength\vect{v}\vectlength^{2} + \vectlength\vect{w}\vectlength^{2}$.

\item If $\vect{u}$, $\vect{v}$, and $\vect{w}$ are all the same length, show that they all make the same angle with $\vect{u} + \vect{v} + \vect{w}$.

\end{enumerate}
\begin{sol}
\begin{enumerate}[label={\alph*.}]
\setcounter{enumi}{1}
\item  The angle $\theta$ between $\vect{u}$ and $(\vect{u} + \vect{v} + \vect{w})$ is given by $\cos\theta = \frac{\vect{u} \dotprod (\vect{u} + \vect{v} + \vect{w})}{\vectlength \vect{u} \vectlength \vectlength \vect{u} + \vect{v} + \vect{w} \vectlength} = \frac{\vectlength \vect{u} \vectlength}{\sqrt{\vectlength \vect{u} \vectlength^2 + \vectlength \vect{v} \vectlength^2 + \vectlength \vect{w} \vectlength^2}} = \frac{1}{\sqrt{3}}$ because $\vectlength \vect{u} \vectlength = \vectlength \vect{v} \vectlength = \vectlength \vect{w} \vectlength$. Similar remarks apply to the other angles.

\end{enumerate}
\end{sol}
\end{ex}

\begin{ex}
\begin{enumerate}[label={\alph*.}]
\item Show that $\vect{n} = \leftB
\begin{array}{r}
a\\
b
\end{array}
\rightB$ is orthogonal to every vector along the line $ax + by + c = 0$.

\item Show that the shortest distance from $P_{0}(x_{0}, y_{0})$ to the line is $\frac{|ax_{0} + by_{0} + c|}{\sqrt{a^2 + b^2}}$.


[\textit{Hint}: If $P_{1}$ is on the line, project $\vect{u} = \longvect{P_{1}P}_{0}$ on $\vect{n}$.]

\end{enumerate}
\begin{sol}
\begin{enumerate}[label={\alph*.}]
\setcounter{enumi}{1}
\item  Let $\vect{p}_{0}$, $\vect{p}_{1}$ be the vectors of $P_{0}$, $P_{1}$, so $\vect{u} = \vect{p}_{0} - \vect{p}_{1}$. Then $\vect{u} \cdot \vect{n} = \vect{p}_{0} \cdot \vect{n}$ -- $\vect{p}_{1} \cdot \vect{n} = (ax_{0} + by_{0}) - (ax_{1} + by_{1}) = ax_{0} + by_{0} + c$. Hence the distance is 
\begin{equation*}
\left\vectlength \left( \frac{\vect{u} \dotprod \vect{n}}{\vectlength \vect{n} \vectlength^2}\right)\vect{n} \right\vectlength = \frac{|\vect{u} \dotprod \vect{n}|}{\vectlength \vect{n} \vectlength}
\end{equation*}
 as required.

\end{enumerate}
\end{sol}
\end{ex}

\begin{ex}
Assume $\vect{u}$ and $\vect{v}$ are nonzero vectors that are not parallel. Show that $\vect{w} = \vectlength\vect{u}\vectlength\vect{v} + \vectlength\vect{v}\vectlength\vect{u}$ is a nonzero vector that bisects the angle between $\vect{u}$ and $\vect{v}$.
\end{ex}

\begin{ex}
Let $\alpha$, $\beta$, and $\gamma$ be the angles a vector $\vect{v} \neq \vect{0}$ makes with the positive $x$, $y$, and $z$ axes, respectively. Then $\cos \alpha$, $\cos \beta$, and $\cos \gamma$ are called the \textbf{direction cosines}\index{direction cosines} of the vector $\vect{v}$.

\begin{enumerate}[label={\alph*.}]
\item If $\vect{v} = \leftB
\begin{array}{r}
a\\
b\\
c
\end{array}
\rightB$, show that $\cos\alpha = \frac{a}{\vectlength \vect{v} \vectlength}$, $\cos\beta = \frac{b}{\vectlength \vect{v} \vectlength}$, and $\cos\gamma= \frac{c}{\vectlength \vect{v} \vectlength}$.

\item Show that $\cos^{2} \alpha + \cos^2 \beta + \cos^2 \gamma = 1$.

\end{enumerate}
\begin{sol}
\begin{enumerate}[label={\alph*.}]
\setcounter{enumi}{1}
\item  This follows from \textbf{(a)} because $\vectlength\vect{v}\vectlength^{2} = a^{2} + b^{2} + c^{2}$.

\end{enumerate}
\end{sol}
\end{ex}

\begin{ex}
Let $\vect{v} \neq \vect{0}$ be any nonzero vector and suppose that a vector $\vect{u}$ can be written as $\vect{u} = \vect{p} + \vect{q}$, where $\vect{p}$ is parallel to $\vect{v}$ and $\vect{q}$ is orthogonal to $\vect{v}$. Show that $\vect{p}$ must equal the projection of $\vect{u}$ on $\vect{v}$. [\textit{Hint}: Argue as in the proof of Theorem~\ref{thm:011958}.]
\end{ex}

\begin{ex}
Let $\vect{v} \neq \vect{0}$ be a nonzero vector and let $a \neq 0$ be a scalar. If $\vect{u}$ is any vector, show that the projection of $\vect{u}$ on $\vect{v}$ equals the projection of $\vect{u}$ on $a\vect{v}$.
\end{ex}

\begin{ex}
\begin{enumerate}[label={\alph*.}]
\item Show that the \textbf{Cauchy-Schwarz inequality}\index{Cauchy-Schwarz inequality} $|\vect{u} \dotprod \vect{v}| \leq \vectlength\vect{u}\vectlength\vectlength\vect{v}\vectlength$ holds for all vectors $\vect{u}$ and $\vect{v}$. [\textit{Hint}: $|\cos \theta| \leq 1$ for all angles $\theta$.]

\item Show that $|\vect{u} \dotprod \vect{v}| = \vectlength\vect{u}\vectlength\vectlength\vect{v}\vectlength$ if and only if $\vect{u}$ and $\vect{v}$ are parallel.


[\textit{Hint}: When is $\cos \theta = \pm 1$?]

\item Show that $|x_{1}x_{2} + y_{1}y_{2} + z_{1}z_{2}| \\ \leq \sqrt{x_{1}^2 + y_{1}^2 + z_{1}^2}\sqrt{x_{2}^2 + y_{2}^2 + z_{2}^2}$

holds for all numbers $x_{1}$, $x_{2}$, $y_{1}$, $y_{2}$, $z_{1}$, and $z_{2}$.

\item Show that $|xy + yz + zx| \leq x^{2} + y^{2} + z^{2}$ for all $x$, $y$, and $z$.

\item Show that $(x + y + z)^{2} \leq 3(x^{2} + y^{2} + z^{2})$ holds for all $x$, $y$, and $z$.

\end{enumerate}
\begin{sol}
\begin{enumerate}[label={\alph*.}]
\setcounter{enumi}{3}
\item  Take $\leftB
\begin{array}{c}
x_{1}\\
y_{1}\\
z_{1}
\end{array}
\rightB
= 
\leftB
\begin{array}{c}
x\\
y\\
z
\end{array}
\rightB$ and $\leftB
\begin{array}{c}
x_{2}\\
y_{2}\\
z_{2}
\end{array}
\rightB
= 
\leftB
\begin{array}{c}
y\\
z\\
x
\end{array}
\rightB$ in (\textbf{c}).


\end{enumerate}
\end{sol}
\end{ex}

\begin{ex}
Prove that the \textbf{triangle inequality}\index{triangle!inequality}\index{triangle inequality} $\vectlength \vect{u} + \vect{v} \vectlength \leq \vectlength\vect{u}\vectlength + \vectlength\vect{v}\vectlength$ holds for all vectors $\vect{u}$ and $\vect{v}$. [\textit{Hint}: Consider the triangle with $\vect{u}$ and $\vect{v}$ as two sides.]
\end{ex}
\end{multicols}

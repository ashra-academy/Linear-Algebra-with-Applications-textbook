\section{Linear Operators on $\RR^3$}
\label{sec:4_4}

Recall that a transformation $T : \RR^n \to \RR^m$ is called \textit{linear} if $T(\vect{x} + \vect{y}) = T(\vect{x}) + T(\vect{y})$ and $T(a\vect{x}) = aT(\vect{x})$ holds for all $\vect{x}$ and $\vect{y}$ in $\RR^n$ and all scalars $a$. In this case we showed (in Theorem~\ref{thm:005789}) that there exists an $m \times n$ matrix $A$ such that $T(\vect{x}) = A\vect{x}$ for all $\vect{x}$ in $\RR^n$, and we say that $T$ is the \textbf{matrix transformation induced}\index{matrix transformation induced}\index{linear transformations!matrix transformation induced} by $A$.\index{set of all ordered $n$-tuples ($\RR^n$)!linear operators}


\begin{definition}{Linear Operator on $\RR^n$}{012946}
A linear transformation
\begin{equation*}
T: \RR^n \to \RR^n
\end{equation*}
is called a \textbf{linear operator}\index{vector geometry!linear operators}\index{linear operator!defined} on $\RR^n$.
\end{definition}


\noindent In Section~\ref{sec:2_6} we investigated three important linear operators on $\RR^2$: rotations about the origin, reflections in a line through the origin, and projections on this line.


In this section we investigate the analogous operators on $\RR^3$:
 Rotations about a line through the origin, reflections in a plane 
through the origin, and projections onto a plane or line through the 
origin in $\RR^3$. In every case we show that the operator is linear, and we find the matrices of all the reflections and projections.


To do this we must prove that these reflections, projections, and rotations are actually \textit{linear} operators on $\RR^3$. In the case of reflections and rotations, it is convenient to examine a more general situation. A transformation $T : \RR^3 \to \RR^3$ is said to be \textbf{distance preserving}\index{linear transformations!distance preserving}\index{distance preserving}\index{linear operator!distance preserving} if the distance between $T(\vect{v})$ and $T(\vect{w})$ is the same as the distance between $\vect{v}$ and $\vect{w}$ for all $\vect{v}$ and $\vect{w}$ in $\RR^3$; that is,
\begin{equation}\label{eq:distancePresEq}
\vectlength T(\vect{v}) - T(\vect{w}) \vectlength = \vectlength \vect{v} - \vect{w} \vectlength \mbox{ for all } \vect{v} \mbox{ and } \vect{w} \mbox{ in } \RR^3
\end{equation}
Clearly reflections and rotations are distance preserving, and both carry $\vect{0}$ to $\vect{0}$, so the following theorem shows that they are both linear.


\begin{theorem}{}{012963}
If $T : \RR^3 \to \RR^3$ is distance preserving, and if $T(\vect{0}) = \vect{0}$, then $T$ is linear.
\end{theorem}

\begin{wrapfigure}[12]{l}{5cm} 
\centering
\begin{tikzpicture}
%set up of axis environment
\begin{axis}[view/h=120,disabledatascaling, 
width=5cm, 
height=5cm, 
xlabel={$x$}, 
ylabel={$y$},
zlabel={$z$},
axis lines=center,
axis on top=false,
xtick=\empty,
ytick=\empty,
ztick=\empty,
xticklabels=\empty, 
yticklabels=\empty, 
zticklabels=\empty, 
every axis x label/.style={
	at={(ticklabel* cs:1.05)},
	anchor=north,
},
every axis y label/.style={
	at={(ticklabel* cs:1.05)},
	anchor=west,
},
every axis z label/.style={
	at={(ticklabel* cs:1.05)},
	anchor=south,
},
clip mode=individual,
domain=-5:5, 
samples=100, 
xmin=0, 
xmax=1.5, 
ymin=0, 
ymax=2.9,
zmin=0,
zmax=2.9]

\coordinate (origin) at (0, 0, 0);
\def \angleV {-55}
\def \angleW {25} %angle from w to y-axis
\def \angleTv {\angleV + 140}
\def \angleTw {\angleW + 140}

\coordinate (ptW) at (0, {cos(\angleW)}, {sin(\angleW)}); %length of w = 1
\coordinate (ptV) at (0, {2 * cos(\angleV)}, {2 * sin(\angleV)}); %length of v = 2
\coordinate (ptVPlusW) at ($(ptW) + (ptV)$);

\coordinate (ptTw) at (0, { cos(\angleTw) }, { sin(\angleTw) });
\coordinate (ptTv) at (0, { 2 * cos(\angleTv) }, { 2 * sin(\angleTv) });
\coordinate (ptTvPlusTw) at ($(ptTv) + (ptTw)$);
\coordinate (ptLabelSumT) at ($ (ptTvPlusTw) + (0, -0.3, 0.25) $);

%right shape
\filldraw[color=dkbluevect, fill=ltbluevect, thick] (origin)--(ptW)--(ptVPlusW)--(ptV)--cycle;
\draw[dkgreenvect, thick, -latex] (origin)--(ptW);
\draw[dkgreenvect, thick, -latex] (origin)--(ptV);
\draw[dkgreenvect, thick, -latex] (origin)--(ptVPlusW);
\node[above] at (ptW) {\footnotesize $\vect{w}$};
\node[right] at (ptVPlusW) {\footnotesize $\vect{v} + \vect{w}$};
\node[left=0.05] at (ptV) {\footnotesize $\vect{v}$};

%left shape
\filldraw[color=dkbluevect, fill=ltbluevect, thick] (origin)--(ptTv)--(ptTvPlusTw)--(ptTw)--cycle;
\draw[dkgreenvect, thick, -latex] (origin)--(ptTw);
\draw[dkgreenvect, thick, -latex] (origin)--(ptTv);
\draw[dkgreenvect, thick, -latex] (origin)--(ptTvPlusTw);
\node[left] at (ptTw) {\footnotesize $T(\vect{w})$};
\node at (ptLabelSumT) {\footnotesize $T(\vect{v} + \vect{w})$};
\node[right] at (ptTv) {\footnotesize $T(\vect{v})$};
\end{axis}
\end{tikzpicture}
\caption{\label{fig:012980}}
\end{wrapfigure}

\begin{proof}
Since $T(\vect{0}) = \vect{0}$, taking $\vect{w} = \vect{0}$ in (\ref{eq:distancePresEq}) shows that $\vectlength T(\vect{v})\vectlength = \vectlength\vect{v}\vectlength$ for all $\vect{v}$ in $\RR^3$, that is $T$ preserves length. Also, $\vectlength T(\vect{v}) - T(\vect{w})\vectlength^{2} = \vectlength\vect{v} - \vect{w}\vectlength^{2}$ by (\ref{eq:distancePresEq}). Since $\vectlength\vect{v} - \vect{w}\vectlength^{2} = \vectlength\vect{v}\vectlength^{2} - 2\vect{v} \dotprod \vect{w} + \vectlength\vect{w}\vectlength^{2}$ always holds, it follows that $T(\vect{v}) \dotprod T(\vect{w}) = \vect{v} \dotprod \vect{w}$ for all $\vect{v}$ and $\vect{w}$. Hence (by Theorem~\ref{thm:011851}) the angle between $T(\vect{v})$ and $T(\vect{w})$ is the same as the angle between $\vect{v}$ and $\vect{w}$ for all (nonzero) vectors $\vect{v}$ and $\vect{w}$ in $\RR^3$.


With this we can show that $T$ is linear. Given nonzero vectors $\vect{v}$ and $\vect{w}$ in $\RR^3$, the vector $\vect{v} + \vect{w}$ is the diagonal of the parallelogram determined by $\vect{v}$ and $\vect{w}$. By the preceding paragraph, the effect of $T$ is to carry this \textit{entire parallelogram} to the parallelogram determined by $T(\vect{v})$ and $T(\vect{w})$, with diagonal $T(\vect{v} + \vect{w})$. But this diagonal is $T(\vect{v}) + T(\vect{w})$ by the parallelogram law (see Figure~\ref{fig:012980}).

In other words, $T(\vect{v} + \vect{w}) = T(\vect{v}) + T(\vect{w})$. A similar argument shows that $T(a\vect{v}) = aT(\vect{v})$ for all scalars $a$, proving that $T$ is indeed linear.
\end{proof}

\noindent Distance-preserving linear operators are called \textbf{isometries}\index{linear operator!isometries}\index{isometries}, and we return to them in Section~\ref{sec:10_4}.


\subsection*{Reflections and Projections}

\index{linear operator!projection}\index{linear operator!reflections}\index{linear transformations!projections}\index{linear transformations!reflections}\index{projection!linear operators}\index{reflections!linear operators}
In Section~\ref{sec:2_6} we studied the reflection $Q_{m} : \RR^2 \to \RR^2$ in the line $y = mx$ and projection $P_{m} : \RR^2 \to \RR^2$ on the same line. We found (in Theorems~\ref{thm:006096} and~\ref{thm:006137}) that they are both linear and
\begin{equation*}
Q_{m} \mbox{ has matrix } \frac{1}{1 + m^2}\leftB
\begin{array}{cc}
1 - m^2 & 2m \\
2m & m^2 -1
\end{array}
\rightB
\quad \mbox{ and } \quad P_{m} \mbox{ has matrix }\frac{1}{1 + m^2}\leftB
\begin{array}{cr}
1  & m \\
m & m^2
\end{array}
\rightB.
\end{equation*}

\begin{wrapfigure}[9]{l}{5cm} 
  \vspace*{-1em}
	\centering
	\begin{tikzpicture}
\coordinate(origin) at (0, 0);
\draw[dkbluevect, thick] (origin) -- (-150:0.5);
\draw[dkbluevect, thick] (origin) -- (30:2.8) node[right, text=black] {\footnotesize $L$};
\draw[dkgreenvect, thick, -latex] (origin)--(30:cos{30} * 2) node[right=0.15, text=black] {\footnotesize $P_L(\vect{v})$};
\draw[dkgreenvect, thick, -latex] (origin)--(60:2) node (ptV) {};
\draw[dkgreenvect, thick, -latex] (origin)--(2, 0) node (ptQ) {};
\draw[dkbluevect, thick, dashed] (ptQ.center)--(ptV.center);

\node[below] at (origin) {\footnotesize $\vect{0}$};
\node[left] at (ptV) {\footnotesize $\vect{v}$};
\node[below] at (ptQ) {\footnotesize $Q_L(\vect{v})$};
\end{tikzpicture}

	\caption{\label{fig:013008}}
\end{wrapfigure}

We now look at the analogues in $\RR^3$.

Let $L$ denote a line through the origin in $\RR^3$. Given a vector $\vect{v}$ in $\RR^3$, the reflection $Q_{L}(\vect{v})$ of $\vect{v}$ in $L$ and the projection $P_{L}(\vect{v})$ of $\vect{v}$ on $L$ are defined in Figure~\ref{fig:013008}. In the same figure, we see that
\begin{equation}\label{eq:refProjEq}
P_{L}(\vect{v}) = \vect{v} + \frac{1}{2}[Q_{L}(\vect{v}) - \vect{v}] = \frac{1}{2}[Q_{L}(\vect{v}) + \vect{v}]
\end{equation}
so the fact that $Q_{L}$ is linear (by Theorem~\ref{thm:012963}) shows that $P_{L}$ is also linear.\footnote{Note that Theorem~\ref{thm:012963} does \textit{not} apply to $P_{L}$ since it does not preserve distance.}

 However, Theorem~\ref{thm:011958} gives us the matrix of $P_{L}$ directly. In fact, if 
 $\vect{d} = \leftB
 \begin{array}{c}
 a \\
 b \\
 c
 \end{array}
 \rightB \neq \vect{0}$
 is a direction vector for $L$, and we write 
$\vect{v} = \leftB
\begin{array}{c}
x \\
y \\
z
\end{array}
\rightB$, then
\begin{equation*}
P_{L}(\vect{v}) = \frac{\vect{v} \dotprod \vect{d}}{\vectlength \vect{d} \vectlength^2}\vect{d} = \frac{ax + by + cz}{a^2 + b^2 + c^2} \leftB
\begin{array}{c}
a \\
b \\
c
\end{array}
\rightB = \frac{1}{a^2 + b^2 + c^2} \leftB
\begin{array}{ccc}
a^2 & ab & ac \\
ab & b^2 & bc \\
ac & bc & c^2
\end{array}
\rightB \leftB
\begin{array}{c}
x \\
y \\
z
\end{array}
\rightB
\end{equation*}
as the reader can verify. Note that this shows directly that $P_{L}$ is a matrix transformation and so gives another proof that it is linear.


\begin{theorem}{}{013009}
Let $L$ denote the line through the origin in $\RR^3$ with direction vector $\vect{d} = \leftB
\begin{array}{c}
a \\
b \\
c
\end{array}
\rightB \neq \vect{0}$. Then $P_{L}$ and $Q_{L}$ are both linear and
\begin{equation*}
P_{L} \mbox{ has matrix }\frac{1}{a^2 + b^2 + c^2} \leftB
\begin{array}{ccc}
a^2 & ab & ac \\
ab & b^2 & bc \\
ac & bc & c^2
\end{array}
\rightB
\end{equation*}
\begin{equation*}
Q_{L} \mbox{ has matrix }\frac{1}{a^2 + b^2 + c^2} \leftB
\begin{array}{ccc}
a^2 - b^2 - c^2 & 2ab & 2ac \\
2ab & b^2 - a^2 - c^2 & 2bc \\
2ac & 2bc & c^2 - a^2 - b^2
\end{array}
\rightB
\end{equation*}
\end{theorem}

\begin{proof}
It remains to find the matrix of $Q_{L}$. But (\ref{eq:refProjEq}) implies that $Q_{L}(\vect{v}) = 2P_{L}(\vect{v}) - \vect{v}$ for each $\vect{v}$ in $\RR^3$, so if $\vect{v} = \leftB
\begin{array}{c}
x \\
y \\
z
\end{array}
\rightB$ we obtain (with some matrix arithmetic):
\begin{align*}
Q_{L}(\vect{v}) &= \left\lbrace \frac{2}{a^2 + b^2 + c^2} \leftB
\begin{array}{ccc}
a^2 & ab & ac \\
ab & b^2 & bc \\
ac & bc & c^2
\end{array}
\rightB
- 
\leftB
\begin{array}{ccc}
1 & 0 & 0 \\
0 & 1 & 0 \\
0 & 0 & 1
\end{array}
\rightB
\right\rbrace \leftB
\begin{array}{c}
x \\
y \\
z
\end{array}
\rightB \\ 
& = 
\frac{1}{a^2 + b^2 + c^2} \leftB
\begin{array}{ccc}
a^2 - b^2 - c^2 & 2ab & 2ac \\
2ab & b^2 - a^2 - c^2 & 2bc \\
2ac & 2bc & c^2 - a^2 - b^2
\end{array}
\rightB  \leftB
\begin{array}{c}
x \\
y \\
z
\end{array}
\rightB
\end{align*}
as required.
\end{proof}

\begin{wrapfigure}[9]{l}{5cm} 
\vspace{-2em}
\centering
\begin{tikzpicture}[scale=0.8]
\coordinate (ptO) at (0, 0);
\coordinate (ptV) at (2, 2);
\coordinate (ptPMv) at (2, 0);
\coordinate (ptQMv) at (2, -2);
\coordinate (ptPlaneBottomLeft) at (-0.5, -0.75);
\path [name path=planeLine] (ptPlaneBottomLeft)++(3, 0);

%lines below plane
\draw[dkgreenvect, thick, -latex] (ptO)--(ptQMv);
\draw[dkbluevect, thick] (2, -0.5)--(ptQMv);
\filldraw[color=dkbluevect, fill=ltbluevect, thick] (ptPlaneBottomLeft)-- ++(1, 2.4)-- ++(2.75, 0) -- ++(-1, -2.4)--cycle;

%lines above plane
\draw[dkbluevect, thick] (ptPMv)++(0, 0.2) -- ++(-0.2, 0) -- ++(0, -0.2);
\draw[dkbluevect, thick] (ptPMv)++(0, 0.2) -- ++(45:0.2) -- ++(0, -0.2);
\draw[dkgreenvect, thick, -latex] (ptO)--(ptV);
\draw[dkgreenvect, thick, -latex] (ptO)--(ptPMv);
\draw[dkbluevect, thick] (ptPMv)--(ptV);

\draw[dkbluevect, thick] (ptPMv)++(0.35, 0) arc [start angle=170, end angle=30, radius=0.2] node (labelM) {};
\path (labelM) ++ (0.3, 0) node {\footnotesize $M$};

%labels
\node[left] at (ptV) {\footnotesize $\vect{v}$};
\node[below] at (ptO) {\footnotesize $O$};
\node[below left] at (ptPMv) {\footnotesize $P_M(\vect{v})$};
\node[left=0.1] at (ptQMv) {\footnotesize $Q_M(\vect{v})$};
\end{tikzpicture}
\caption{\label{fig:013036}}
\end{wrapfigure}

In $\RR^3$ we can reflect in planes as well as lines. Let $M$ denote a plane through the origin in $\RR^3$. Given a vector $\vect{v}$ in $\RR^3$, the reflection $Q_{M}(\vect{v})$ of $\vect{v}$ in $M$ and the projection $P_{M}(\vect{v})$ of $\vect{v}$ on $M$ are defined in Figure~\ref{fig:013036}. As above, we have
\begin{equation*}
P_{M}(\vect{v}) = \vect{v} + \frac{1}{2}[Q_{M}(\vect{v}) - \vect{v}] = \frac{1}{2}[Q_{M}(\vect{v}) + \vect{v}]
\end{equation*}
so the fact that $Q_{M}$ is linear (again by Theorem~\ref{thm:012963}) shows that $P_{M}$ is also linear. 

Again we can obtain the matrix directly. If $\vect{n}$ is a normal for the plane $M$, then Figure~\ref{fig:013036} shows that
\begin{equation*}
P_{M}(\vect{v}) = \vect{v} - \proj{\vect{n}}{\vect{v}} =  \vect{v} - \frac{\vect{v} \dotprod \vect{n}}{\vectlength \vect{n} \vectlength^2}\vect{n} \mbox{ for all vectors }\vect{v}.
\end{equation*}
If 
$\vect{n} = \leftB
\begin{array}{c}
a \\
b \\
c
\end{array}
\rightB \neq \vect{0}$
 and 
$\vect{v} = \leftB
\begin{array}{c}
x \\
y \\
z
\end{array}
\rightB$, a computation like the above gives
\begin{align*}
P_{M}(\vect{v}) &=  
\leftB
\begin{array}{ccc}
1 & 0 & 0 \\
0 & 1 & 0 \\
0 & 0 & 1
\end{array}
\rightB
\leftB
\begin{array}{c}
x \\
y \\
z
\end{array}
\rightB 
- \frac{ax + by + cz}{a^2 + b^2 + c^2} \leftB
\begin{array}{c}
a \\
b \\
c
\end{array}
\rightB 
\\
&=
\frac{1}{a^2 + b^2 + c^2} \leftB
\begin{array}{ccc}
b^2 + c^2 & -ab & -ac \\
-ab & a^2 + c^2 & -bc \\
-ac & -bc & b^2 + c^2 
\end{array}
\rightB  \leftB
\begin{array}{c}
x \\
y \\
z
\end{array}
\rightB
\end{align*}
This proves the first part of

\begin{theorem}{}{013042}
Let $M$ denote the plane through the origin in $\RR^3$ with normal $\vect{n} = \leftB
\begin{array}{c}
a \\
b \\
c
\end{array}
\rightB \neq \vect{0}$. Then $P_{M}$ and $Q_{M}$ are both linear and
\begin{equation*}
P_{M} \mbox{ has matrix }\frac{1}{a^2 + b^2 + c^2} \leftB
\begin{array}{ccc}
b^2 + c^2 & -ab & -ac \\
-ab & a^2 + c^2 & -bc \\
-ac & -bc & a^2 + b^2  
\end{array}
\rightB
\end{equation*}
\begin{equation*}
Q_{M} \mbox{ has matrix }\frac{1}{a^2 + b^2 + c^2} \leftB
\begin{array}{ccc}
b^2 + c^2 - a^2 & -2ab & -2ac \\
-2ab & a^2 + c^2 - b^2 & -2bc \\
-2ac & -2bc & a^2 + b^2 - c^2
\end{array}
\rightB
\end{equation*}
\end{theorem}

\begin{proof}
It remains to compute the matrix of $Q_{M}$. Since $Q_{M}(\vect{v}) = 2P_{M}(\vect{v}) - \vect{v}$ for each $\vect{v}$ in $\RR^3$, the computation is similar to the above and is left as an exercise for the reader.
\end{proof}

\subsection*{Rotations}

\index{linear operator!rotations}\index{linear transformations!rotations}\index{rotations!linear operators}
In Section~\ref{sec:2_6} we studied the rotation $R_{\theta} : \RR^2 \to \RR^2$ counterclockwise about the origin through the angle $\theta$. Moreover, we showed in Theorem~\ref{thm:006021} that $R_{\theta}$ is linear and has matrix $\leftB
\begin{array}{cc}
\cos\theta & -\sin\theta \\
\sin\theta & \cos\theta
\end{array}
\rightB$. One extension of this is given in the following example.


\begin{example}{}{013065}
Let $R_{z,\theta} : \RR^3 \to \RR^3$ denote rotation of $\RR^3$ about the $z$ axis through an angle $\theta$ from the positive $x$ axis toward the positive $y$ axis. Show that $R_{z,\theta}$ is linear and find its matrix.



\begin{wrapfigure}[10]{l}{5cm} 
\centering
\begin{tikzpicture}
%set up of axis environment
\begin{axis}[view/h=110,disabledatascaling, 
width=5cm, 
height=5cm, 
xlabel={$x$}, 
ylabel={$y$},
zlabel={$z$},
axis lines=center,
axis on top=false,
xtick=\empty,
ytick=\empty,
ztick=\empty,
xticklabels=\empty, 
yticklabels=\empty, 
zticklabels=\empty, 
every axis x label/.style={
	at={(ticklabel* cs:1.05)},
	anchor=north,
},
every axis y label/.style={
	at={(ticklabel* cs:1.05)},
	anchor=west,
},
every axis z label/.style={
	at={(ticklabel* cs:1.05)},
	anchor=south,
},
clip mode=individual,
domain=-5:5, 
samples=100, 
xmin=-0.4, 
xmax=1.3, 
ymin=-0.3, 
ymax=1.5,
zmin=0,
zmax=1.5]
\coordinate (origin) at (0, 0, 0);
\coordinate (ptI) at (1, 0, 0);
\coordinate (ptJ) at (0, 1, 0);
\coordinate (ptK) at (0, 0, 1);
\def \angleRzi {40}
\coordinate (ptRzi) at (1.1 * cos{\angleRzi}, 1.1 * sin{\angleRzi}, 0);
\coordinate (ptRzj) at (0, 1.1 * cos{\angleRzi}, 1.1 * sin{\angleRzi});

\coordinate (ptITheta) at (0.4, 0, 0);
\coordinate (ptRziTheta) at (0.4 * cos{\angleRzi}, 0.4 * sin{\angleRzi}, 0);
\coordinate (ptJTheta) at (0, 0.4, 0);
\coordinate (ptRzjTheta) at (0, 0.4 * cos{\angleRzi}, 0.4 * sin{\angleRzi});

%lines
\draw[dkgreenvect, thick, -latex] (origin)--(ptI);
\draw[dkgreenvect, thick, -latex] (origin)--(ptJ);
\draw[dkgreenvect, thick, -latex] (origin)--(ptK);
\draw[dkgreenvect, thick, -latex] (origin)--(ptRzi);
\draw[dkgreenvect, thick, -latex] (origin)--(ptRzj);

%arcs
\draw[dkbluevect, thick, dashed] (ptI) to [bend right] (ptRzi);
\draw[dkbluevect, thick, -latex] (ptITheta) to [bend right] (ptRziTheta);
\draw[dkbluevect, thick, dashed] (ptJ) to [bend right] (ptRzj);
\draw[dkbluevect, thick, -latex] (ptJTheta) to [bend right] (ptRzjTheta);

%labels
\path(ptITheta) -- (ptRziTheta) node[below, pos=0.4] {\footnotesize $\theta$};
\path(ptJTheta) -- (ptRzjTheta) node[right, pos=0.6] {\footnotesize $\theta$};
\path(ptI) ++ (0, 0, 0.3) node {\footnotesize $\vect{i}$};
\node[below] at (ptJ) {\footnotesize $\vect{j}$};
\node[left] at (ptK) {\footnotesize $\vect{k}$};
\path ($(ptRzi) + (0, 0, -0.3) $) node {\footnotesize $R_z(\vect{i})$};
\node[above] at (ptRzj) {\footnotesize $R_z(\vect{j})$};
\end{axis}
\end{tikzpicture}
\caption{\label{fig:013089}}
\end{wrapfigure}

\setlength{\rightskip}{0pt plus 200pt}
\begin{solution} First $R$ is distance preserving and so is linear by Theorem~\ref{thm:012963}. Hence we apply Theorem~\ref{thm:005789} to obtain the matrix of  $R_{z,\theta}$.

Let $\vect{i} = \leftB
\begin{array}{c}
1 \\
0 \\
0
\end{array}
\rightB$, 
$\vect{j} = \leftB
\begin{array}{c}
0 \\
1 \\
0
\end{array}
\rightB$, and 
$\vect{k} = \leftB
\begin{array}{c}
0 \\
0 \\
1
\end{array}
\rightB$ denote the standard basis of $\RR^3$; we must find  $R_{z,\theta}(\vect{i})$, $R_{z,\theta}(\vect{j})$, and  $R_{z,\theta}(\vect{k})$. Clearly  $R_{z,\theta}(\vect{k}) = \vect{k}$. The effect of $R_{z,\theta}$ on the $x$-$y$ plane is to rotate it counterclockwise through the angle $\theta$. Hence Figure~\ref{fig:013089} gives
\begin{equation*}
R_{z,\theta}(\vect{i}) = \leftB
\begin{array}{c}
\cos\theta \\
\sin\theta \\
0
\end{array}
\rightB,\
R_{z,\theta}(\vect{j}) = \leftB
\begin{array}{c}
-\sin\theta \\
\cos\theta \\
0
\end{array}
\rightB
\end{equation*}
so, by Theorem~\ref{thm:005789}, $R_{z,\theta}$ has matrix
\begin{equation*}
\leftB
\begin{array}{ccc}
R_{z,\theta}(\vect{i}) &  R_{z,\theta}(\vect{j}) & R_{z,\theta}(\vect{k})
\end{array}
\rightB = \leftB
\begin{array}{ccc}
\cos\theta & -\sin\theta & 0 \\
sin\theta & \cos\theta & 0\\
0 & 0 & 1
\end{array}
\rightB
\end{equation*}

\end{solution}
\end{example}

Example~\ref{exa:013065} begs to be generalized. Given a line $L$ through the origin in $\RR^3$, every rotation about $L$ through a fixed angle is clearly distance preserving, and so is a linear operator by Theorem~\ref{thm:012963}.
 However, giving a precise description of the matrix of this rotation is
 not easy and will have to wait until more techniques are available.


\subsection*{Transformations of Areas and Volumes}

\index{area!linear transformations of}\index{volume!linear transformations of}\index{linear operator!transformations of areas and volumes}\index{linear transformations!of areas}\index{linear transformations!of volume}
\begin{wrapfigure}{l}{5cm} 
\centering
\begin{tikzpicture}[scale=0.8, rotate=30]
\coordinate(origin) at (0, 0);
\coordinate(ptSV) at (3, 0);
\coordinate(ptV) at (5, 0);

\draw[dkgreenvect, thick, -latex] (origin)--(ptSV);
\draw[dkgreenvect, thick, -latex] (ptSV)--(ptV);

\fill[black] (origin) circle (2pt);
\fill[black] (ptSV) circle (2pt);

\node[right=0.2] at (origin) {\small Origin};
\node[below=0.1] at (ptSV) {\small s$\vect{v}$};
\node[below=0.1] at (ptV) {\small $\vect{v}$};
\end{tikzpicture}
\caption{\label{fig:013099}}
\end{wrapfigure}

Let $\vect{v}$ be a nonzero vector in $\RR^3$. Each vector in the same direction as $\vect{v}$ whose length is a fraction $s$ of the length of $\vect{v}$ has the form $s\vect{v}$ (see Figure~\ref{fig:013099}).

With this, scrutiny of Figure~\ref{fig:013100} shows that a vector $\vect{u}$ is in the parallelogram determined by $\vect{v}$ and $\vect{w}$ if and only if it has the form $\vect{u} = s\vect{v} + t\vect{w}$ where $0 \leq s \leq 1$ and $0 \leq t \leq 1$. But then, if $T : \RR^3 \to \RR^3$ is a linear transformation, we have
\begin{equation*}
T(s\vect{v} + t\vect{w}) = T(s\vect{v}) + T(t\vect{w}) = sT(\vect{v}) + tT(\vect{w})
\end{equation*}

\begin{wrapfigure}[6]{l}{5cm} 
\vspace{-2em}
\centering
\begin{tikzpicture}[scale=0.9]
\coordinate(ptO) at (0, 0);
\coordinate(ptSV) at (0.8, 1.6);
\coordinate(ptV) at (1, 2);
\coordinate(ptTW) at (2, 0.2);
\coordinate(ptW) at (3, 0.3);

%green arrows
\draw[dkgreenvect, thick, -latex] (ptO)--(ptSV);
\draw[dkgreenvect, thick, -latex] (ptSV)--(ptV);
\draw[dkgreenvect, thick, -latex] (ptO)--(ptTW);
\draw[dkgreenvect, thick, -latex] (ptTW)--(ptW);
\draw[dkgreenvect, thick, -latex] (ptO)--($(ptSV) + (ptTW)$) node (ptSum1) {};

%dashed lines
\draw[dkbluevect, thick, dashed] (ptSV)-- ++(ptTW)--(ptTW);
\draw[dkbluevect, thick, dashed] (ptV)-- ++(ptW)--(ptW);

%points
\fill[black] (ptO) circle (2pt);
\fill[black] (ptSV) circle (2pt);
\fill[black] (ptV) circle (2pt);
\fill[black] (ptTW) circle (2pt);
\fill[black] (ptW) circle (2pt);
\fill[black] (ptSum1) circle (2pt);

%labels
\node[left] at (ptO){\small $O$};
\node[left] at (ptSV){\small $s\vect{v}$};
\node[left] at (ptV){\small $\vect{v}$};
\path(ptO)--(ptSum1) node[above, pos=0.5, rotate=30] {\small $s\vect{v}+t\vect{w}$};
\node[below] at (ptTW){\small $t\vect{w}$};
\node[below] at (ptW){\small $\vect{w}$};
\end{tikzpicture}
\caption{\label{fig:013100}}
\end{wrapfigure}

\noindent Hence $T(s\vect{v} + t\vect{w})$ is in the parallelogram determined by $T(\vect{v})$ and $T(\vect{w})$. Conversely, every vector in this parallelogram has the form $T(s\vect{v} + t\vect{w})$ where $s\vect{v} + t\vect{w}$ is in the parallelogram determined by $\vect{v}$ and $\vect{w}$. For this reason, the parallelogram determined by $T(\vect{v})$ and $T(\vect{w})$ is called the \textbf{image}\index{image!of the parallelogram}\index{parallelogram!image} of the parallelogram determined by $\vect{v}$ and $\vect{w}$. We record this discussion as:

\begin{wrapfigure}[11]{l}{5cm} 
\vspace*{-2em}
\centering
\begin{tikzpicture}[scale=0.9]
\coordinate(ptO) at (0, 0);
\coordinate(ptV) at (1, 2);
\coordinate(ptW) at (1.5, 0.5);

%blue lines
\draw[dkbluevect, thick, name path=line1] (0.25, 0.5)-- ++(ptW);
\draw[dkbluevect, thick] (0.5, 1.0)-- ++(ptW);
\draw[dkbluevect, thick] (0.75, 1.5)-- ++(ptW);
\draw[dkbluevect, thick] (ptV)-- ++(ptW)--(ptW);

%green lines
\draw[dkgreenvect, thick, -latex] (ptO)--(ptV) node[left, text=black, pos=0.7] {\small $\vect{v}$};
\draw[dkgreenvect, thick, -latex] (ptO)--(ptW) node[below, text=black] {\small $\vect{w}$};
% u line
\path[name path=lineVW] (ptV)--(ptW);
\path[name intersections={of=line1 and lineVW, by=ptU}];
\draw[dkgreenvect, thick, -latex] (ptO)--(ptU.center) node[above, text=black, pos=0.8] {\small $\vect{u}$};

%points
\fill (ptO) circle (2pt);
\fill (ptV) circle (2pt);
\fill (ptW) circle (2pt);
\node[left] at (ptO) {\small $O$};
\end{tikzpicture}

\begin{tikzpicture}
\coordinate(ptO) at (0, 0);
\coordinate(ptTv) at (2, -1);
\coordinate(ptTw) at (1.5, 2);

%blue lines
\draw[dkbluevect, thick, name path=line1] (0.5, -0.25)-- ++(ptTw);
\draw[dkbluevect, thick] (1, -0.5)-- ++(ptTw);
\draw[dkbluevect, thick] (1.5, -0.75)-- ++(ptTw);
\draw[dkbluevect, thick] (ptTv)-- ++(ptTw)--(ptTw);

%green lines
\draw[dkgreenvect, thick, -latex] (ptO)--(ptTv) node[right, text=black] {\footnotesize $T(\vect{v})$};
\draw[dkgreenvect, thick, -latex] (ptO)--(ptTw) node[left, text=black] {\footnotesize $T(\vect{w})$};
%T(u) line
\path[name path=lineTVTW] (ptTv)--(ptTw);
\path[name intersections={of=line1 and lineTVTW, by=ptTu}];
\draw[dkgreenvect, thick, -latex] (ptO)--(ptTu.center);
\path(ptTu) -- ++(0, -0.5) node {\footnotesize $T(\vect{u})$};

%points
\fill (ptO) circle (2pt);
\fill (ptTv) circle (2pt);
\fill (ptTw) circle (2pt);
\node[left] at (ptO) {\footnotesize $O$};
\end{tikzpicture}
\caption{\label{fig:013110}}
\end{wrapfigure}

\hspace{0em}\hfill\begin{theorem}{}{013102}
If $T : \RR^3 \to \RR^3$ (or $\RR^2 \to \RR^2$) is a linear operator, the image of the parallelogram determined by vectors $\vect{v}$ and $\vect{w}$ is the parallelogram determined by $T(\vect{v})$ and $T(\vect{w})$.
\end{theorem}

\noindent This result is illustrated in Figure~\ref{fig:013110}, and was used in Examples~\ref{exa:003088} and~\ref{exa:003128} to reveal the effect of expansion and shear transformations.

We now describe the effect of a linear transformation $T: \RR^3 \to \RR^3$ on the parallelepiped determined by three vectors $\vect{u}$, $\vect{v}$, and $\vect{w}$ in $\RR^3$ (see the discussion preceding Theorem~\ref{thm:012765}). If $T$ has matrix $A$, Theorem~\ref{thm:013102} shows that this parallelepiped is carried to the parallelepiped determined by $T(\vect{u}) = A\vect{u}$, $T(\vect{v}) = A\vect{v}$, and $T(\vect{w}) = A\vect{w}$. In particular, we want to discover how the volume changes, and it turns out to be closely related to the determinant of the matrix $A$.\index{parallelepiped}

\begin{theorem}{}{013115}
Let $\func{vol}(\vect{u}, \vect{v}, \vect{w})$ denote the volume of the parallelepiped determined by three vectors $\vect{u}$, $\vect{v}$, and $\vect{w}$ in $\RR^3$, and let area $(\vect{p}, \vect{q})$ denote the area of the parallelogram determined by two vectors $\vect{p}$ and $\vect{q}$ in $\RR^2$. Then:\index{volume!of parallelepiped}

\begin{enumerate}
\item If $A$ is a $3 \times 3$ matrix, then $\func{vol}(A\vect{u}, A\vect{v}, A\vect{w}) = |\func{det}(A)| \cdot \func{vol}(\vect{u}, \vect{v}, \vect{w})$.

\item If $A$ is a $2 \times 2$ matrix, then $\func{area}(A\vect{p}, A\vect{q}) = |\func{det}(A)| \cdot \func{area}(\vect{p}, \vect{q})$.

\end{enumerate}
\end{theorem}

\begin{proof}

\begin{enumerate}
\item  Let $\leftB \begin{array}{ccc} \vect{u} & \vect{v} & \vect{w} \end{array}\rightB$ denote the $3 \times 3$ matrix with columns $\vect{u}$, $\vect{v}$, and $\vect{w}$. Then
\begin{equation*}
\func{vol}(A\vect{u}, A\vect{v}, A\vect{w}) = |A\vect{u} \dotprod (A\vect{v} \times A\vect{w})|
\end{equation*}
by Theorem~\ref{thm:012765}. Now apply Theorem~\ref{thm:012676} twice to get
\begin{align*}
A\vect{u} \dotprod (A\vect{v} \times A\vect{w}) = \func{det}\leftB \begin{array}{ccc} A\vect{u} & A\vect{v} & A\vect{w}\end{array}\rightB 
&= \func{det}(A\leftB \begin{array}{ccc} \vect{u} & \vect{v} & \vect{w} \end{array}\rightB) \\
&= \func{det}(A)\func{det}\leftB \begin{array}{ccc} \vect{u} & \vect{v} & \vect{w} \end{array}\rightB \\
&= \func{det}(A)(\vect{u} \dotprod (\vect{v} \times \vect{w}))
\end{align*}
where we used Definition~\ref{def:003447} and the product theorem for determinants. Finally (1) follows from Theorem~\ref{thm:012765} by taking absolute values.
\end{enumerate}

\begin{wrapfigure}{l}{5cm} 
\centering
\begin{tikzpicture}[scale=0.9]
\coordinate (origin) at(0, 0);
\coordinate (ptK) at (0, 1.25);
\coordinate (ptP1) at (3, -1);
\coordinate (ptQ1) at (1, 0.4);

\filldraw [color=dkbluevect, fill=ltbluevect, thick] (origin)--(ptQ1)-- ++(ptP1)--(ptP1)--cycle;
\draw [dkbluevect, thick] (ptK) -- ++(ptQ1) -- ++(ptP1) -- ($(ptK) + (ptP1)$)--cycle;
\draw [dkbluevect, thick] (ptQ1) -- +(ptK)
	(ptP1) -- +(ptK)
	($(ptQ1) + (ptP1)$) -- +(ptK);

\draw [dkgreenvect, thick, -latex] (origin)--(ptK) node[left, text=black, pos=0.9] {\small $\vect{k}$};
\draw [dkgreenvect, thick, -latex] (origin)--(ptP1) node[below, text=black, pos=0.8] {\small $\vect{p}_1$};
\draw [dkgreenvect, thick, -latex] (origin)--(ptQ1) node[above, text=black, pos=0.7] {\small $\vect{q}_1$};
\end{tikzpicture}
\end{wrapfigure}

\hspace{0em}\begin{enumerate}
\setcounter{enumi}{1}
\item Given $\vect{p} = \leftB
\begin{array}{c}
x\\
y 
\end{array}
\rightB$
 in $\RR^2$, 
$\vect{p}_{1} = \leftB
\begin{array}{c}
x \\
y \\
0
\end{array}
\rightB$
in $\RR^3$. By the diagram, $\func{area}(\vect{p}, \vect{q}) = \func{vol}(\vect{p}_{1}, \vect{q}_{1}, \vect{k})$ where $\vect{k}$ is the (length $1$) coordinate vector along the $z$ axis. If $A$ is a $2 \times 2$ matrix, write $A_{1} = \leftB
\begin{array}{cc}
A & 0 \\
0 & 1
\end{array}
\rightB$ in block form, and observe that $(A\vect{v})_{1} = (A_{1}\vect{v}_{1})$ for all $\vect{v}$ in $\RR^2$ and $A_{1}\vect{k} = \vect{k}$. Hence part (1) of this theorem shows
\begin{align*}
\func{area}(A\vect{p}, A\vect{q})
&= \func{vol}(A_{1}\vect{p}_{1}, A_{1}\vect{q}_{1}, A_{1}\vect{k}) \\
&= |\func{det}(A_{1})|\func{vol}(\vect{p}_{1}, \vect{q}_{1}, \vect{k})\\
&= |\func{det}(A)|\func{area}(\vect{p}, \vect{q})
\end{align*}
as required.
\end{enumerate}
\vspace*{-2em}\end{proof}

Define the \textbf{unit square}\index{unit square} and \textbf{unit cube}\index{unit cube} to be the square and cube corresponding to the coordinate vectors in $\RR^2$ and $\RR^3$, respectively\index{coordinate vectors}\index{vectors!coordinate vectors}. Then Theorem~\ref{thm:013115} gives a geometrical meaning to the determinant of a matrix $A$:


\begin{itemize}
\item If A is a $2 \times 2$ matrix, then $|\func{det}(A)|$ is the area of the image of the unit square under multiplication by $A$;

\item If A is a $3 \times 3$ matrix, then $|\func{det}(A)|$ is the volume of the image of the unit cube under multiplication by $A$.

\end{itemize}

\noindent These results, together with the importance of areas and volumes in geometry,
 were among the reasons for the initial development of determinants.\index{determinants!initial development of}


\section*{Exercises for \ref{sec:4_4}}

\begin{Filesave}{solutions}
\solsection{Section~\ref{sec:4_4}}
\end{Filesave}

\begin{multicols}{2}
\begin{ex}
In each case show that that $T$ is either projection on a line, reflection in a line, or rotation through an angle, and find the line or angle.

\begin{enumerate}[label={\alph*.}]
\item $T\leftB
\begin{array}{c}
x\\
y 
\end{array}
\rightB
= \frac{1}{5}
\leftB
\begin{array}{c}
x + 2y\\
2x + 4y 
\end{array}
\rightB$
\item $T\leftB
\begin{array}{c}
x\\
y 
\end{array}
\rightB
= \frac{1}{2}
\leftB
\begin{array}{c}
x - y\\
y - x
\end{array}
\rightB$
\item $T\leftB
\begin{array}{c}
x\\
y 
\end{array}
\rightB
= \frac{1}{\sqrt{2}}
\leftB
\begin{array}{c}
-x - y\\
x - y 
\end{array}
\rightB$
\item $T\leftB
\begin{array}{c}
x\\
y 
\end{array}
\rightB
= \frac{1}{5}
\leftB
\begin{array}{c}
-3x + 4y\\
4x + 3y 
\end{array}
\rightB$
\item $T\leftB
\begin{array}{c}
x\\
y 
\end{array}
\rightB =
\leftB
\begin{array}{c}
-y\\
-x 
\end{array}
\rightB$
\item $T\leftB
\begin{array}{c}
x\\
y 
\end{array}
\rightB
= \frac{1}{2}
\leftB
\begin{array}{c}
x - \sqrt{3}y\\
\sqrt{3}x + y 
\end{array}
\rightB$
\end{enumerate}
\begin{sol}
\begin{enumerate}[label={\alph*.}]
\setcounter{enumi}{1}
\item  $A = \leftB
\begin{array}{rr}
1 & -1\\
-1 & 1\\
\end{array}
\rightB$, projection on $y = -x$.

\setcounter{enumi}{3}
\item $A = \frac{1}{5}\leftB
\begin{array}{rr}
-3 & 4\\
4 & 3\\
\end{array}
\rightB$, reflection in $y = 2x$.

\setcounter{enumi}{5}
\item  $A = \frac{1}{2}\leftB
\begin{array}{rr}
1 & -\sqrt{3}\\
\sqrt{3} & 1\\
\end{array}
\rightB$,
 rotation through $\frac{\pi}{3}$.

\end{enumerate}
\end{sol}
\end{ex}

\begin{ex}
Determine the effect of the following transformations.


\begin{enumerate}[label={\alph*.}]
\item Rotation through $\frac{\pi}{2}$, followed by projection on the $y$ axis, followed by reflection in the line $y = x$.

\item Projection on the line $y = x$ followed by projection on the line $y = -x$.

\item Projection on the $x$ axis followed by reflection in the line $y = x$.

\end{enumerate}
\begin{sol}
\begin{enumerate}[label={\alph*.}]
\setcounter{enumi}{1}
\item  The zero transformation.

\end{enumerate}
\end{sol}
\end{ex}

\begin{ex}
In each case solve the problem by finding the matrix of the operator.


\begin{enumerate}[label={\alph*.}]
\item Find the projection of 
$\vect{v} = \leftB
\begin{array}{r}
1\\
-2\\
3 
\end{array}
\rightB$
 on the plane with equation $3x - 5y + 2z = 0$.

\item Find the projection of 
$\vect{v} = \leftB
\begin{array}{r}
0\\
1\\
-3 
\end{array}
\rightB$
 on the plane with equation $2x - y + 4z = 0$.

\item Find the reflection of 
$\vect{v} = \leftB
\begin{array}{r}
1\\
-2\\
3 
\end{array}
\rightB$
 in the plane with equation $x - y + 3z = 0$.

\item Find the reflection of 
$\vect{v} = \leftB
\begin{array}{r}
0\\
1\\
-3 
\end{array}
\rightB$
 in the plane with equation $2x + y -5z = 0$.

\item Find the reflection of 
$\vect{v} = \leftB
\begin{array}{r}
2\\
5\\
-1 
\end{array}
\rightB$
in the line with equation 
$\leftB
\begin{array}{r}
x\\
y\\
z 
\end{array}
\rightB
= t
\leftB
\begin{array}{r}
1\\
1\\
-2 
\end{array}
\rightB$.

\item Find the projection of 
$\vect{v} = \leftB
\begin{array}{r}
1\\
-1\\
7 
\end{array}
\rightB$
 on the line with equation $\leftB
 \begin{array}{r}
 x\\
 y\\
 z 
 \end{array}
 \rightB
 = t
 \leftB
 \begin{array}{r}
 3\\
 0\\
 4 
 \end{array}
 \rightB$.

\item Find the projection of 
$\vect{v} = \leftB
\begin{array}{r}
1\\
1\\
-3
\end{array}
\rightB$
 on the line with equation 
 $\leftB
 \begin{array}{r}
 x\\
 y\\
 z 
 \end{array}
 \rightB
 = t
 \leftB
 \begin{array}{r}
 2\\
 0\\
 -3 
 \end{array}
 \rightB$.

\item Find the reflection of 
$\vect{v} = \leftB
\begin{array}{r}
2\\
-5\\
0 
\end{array}
\rightB$
 in the line with equation 
 $\leftB
 \begin{array}{r}
 x\\
 y\\
 z 
 \end{array}
 \rightB
 = t
 \leftB
 \begin{array}{r}
 1\\
 1\\
 -3
 \end{array}
 \rightB$.

\end{enumerate}
\begin{sol}
\begin{enumerate}[label={\alph*.}]
\setcounter{enumi}{1}
\item 
$\frac{1}{21}\leftB
\begin{array}{rrr}
	17 & 2 & -8\\
	2 & 20 & 4\\
	-8 & 4 & 5
\end{array}
\rightB
\leftB
\begin{array}{r}
0\\
1\\
-3
\end{array}
\rightB$

\setcounter{enumi}{3}
\item  
$\frac{1}{30}\leftB
\begin{array}{rrr}
22 & -4 & 20\\
-4 & 28 & 10\\
20 & 10 & -20
\end{array}
\rightB
\leftB
\begin{array}{r}
0\\
1\\
-3
\end{array}
\rightB$

\setcounter{enumi}{5}
\item  
$\frac{1}{25}\leftB
\begin{array}{rrr}
9 & 0 & 12\\
0 & 0 & 0\\
12 & 0 & 16
\end{array}
\rightB
\leftB
\begin{array}{r}
1\\
-1\\
7
\end{array}
\rightB$

\setcounter{enumi}{7}
\item 
 $\frac{1}{11}\leftB
\begin{array}{rrr}
-9 & 2 & -6\\
2 & -9 & -6\\
-6 & -6 & 7
\end{array}
\rightB
\leftB
\begin{array}{r}
2\\
-5\\
0
\end{array}
\rightB$

\end{enumerate}
\end{sol}
\end{ex}

\begin{ex}
\begin{enumerate}[label={\alph*.}]
\item Find the rotation of 
$\vect{v} = \leftB
\begin{array}{r}
2\\
3\\
-1 
\end{array}
\rightB$
 about the $z$ axis through $\theta = \frac{\pi}{4}$.

\item Find the rotation of 
$\vect{v} = \leftB
\begin{array}{r}
1\\
0\\
3 
\end{array}
\rightB$
 about the $z$ axis through $\theta = \frac{\pi}{6}$.

\end{enumerate}
\begin{sol}
\begin{enumerate}[label={\alph*.}]
\setcounter{enumi}{1}
\item  
$\frac{1}{2}\leftB
\begin{array}{rrr}
\sqrt{3} & -1 & 0\\
1 & \sqrt{3} & 0\\
0 & 0 & 1
\end{array}
\rightB
\leftB
\begin{array}{r}
1\\
0\\
3
\end{array}
\rightB$

\end{enumerate}
\end{sol}
\end{ex}

\begin{ex}
Find the matrix of the rotation in $\RR^3$ about the $x$ axis through the angle $\theta$ (from the positive $y$ axis to the positive $z$ axis).
\end{ex}

\begin{ex}
Find the matrix of the rotation about the $y$ axis through the angle $\theta$ (from the positive $x$ axis to the positive $z$ axis).

\begin{sol}
$\leftB
\begin{array}{ccc}
\cos\theta & 0 & -\sin\theta\\
0 & 1 & 0\\
\sin\theta & 0 & \cos\theta
\end{array}
\rightB$
\end{sol}
\end{ex}

\begin{ex}
If $A$ is $3 \times 3$, show that the image of the line in $\RR^3$ through $\vect{p}_{0}$ with direction vector $\vect{d}$ is the line through $A\vect{p}_{0}$ with direction vector $A\vect{d}$, assuming that $A\vect{d} \neq \vect{0}$. What happens if $A\vect{d} = \vect{0}$?
\end{ex}

\begin{ex}
If $A$ is $3 \times 3$ and invertible, show that the image of the plane through the origin with normal $\vect{n}$ is the plane through the origin with normal $\vect{n}_{1} = B\vect{n}$ where $B = (A^{-1})^{T}$. [\textit{Hint}: Use the fact that $\vect{v} \dotprod  \vect{w} = \vect{v}^{T}\vect{w}$ to show that $\vect{n}_{1} \dotprod (A\vect{p}) = \vect{n} \dotprod \vect{p}$ for each $\vect{p}$ in $\RR^3$.]
\end{ex}

\columnbreak
\begin{ex}
Let $L$ be the line through the origin in $\RR^2$ with direction vector $\vect{d} = \leftB
\begin{array}{r}
a\\
b\\
\end{array}
\rightB \neq 0$.

\begin{enumerate}[label={\alph*.}]
\item If $P_{L}$ denotes projection on $L$, show that $P_{L}$ has matrix $\frac{1}{a^2 + b^2}\leftB
\begin{array}{cc}
a^2 & ab\\
ab & b^2\\
\end{array}\rightB$.

\item If $Q_{L}$ denotes reflection in $L$, show that $Q_{L}$ has matrix $\frac{1}{a^2 + b^2}\leftB
\begin{array}{cc}
a^2 - b^2 & 2ab\\
2ab & b^2 - a^2\\
\end{array}\rightB$.

\end{enumerate}
\begin{sol}
\begin{enumerate}[label={\alph*.}]
\item  Write $\vect{v} = \leftB
\begin{array}{r}
x\\
y
\end{array}
\rightB$.

\begin{align*}
P_{L}(\vect{v}) = \left(\frac{\vect{v} \dotprod \vect{d}}{\vectlength \vect{d} \vectlength^2}\right)\vect{d} & = \frac{ax + by}{a^2 + b^2}\leftB
\begin{array}{r}
a\\
b
\end{array}
\rightB  \\
&  = \frac{1}{a^2 + b^2}\leftB
\begin{array}{c}
a^2x + aby\\
abx + b^2y
\end{array}
\rightB \\
& = \frac{1}{a^2 + b^2}\leftB
\begin{array}{c}
a^2 + ab\\
ab + b^2
\end{array}
\rightB \leftB
\begin{array}{r}
x\\
y
\end{array}
\rightB
\end{align*}

\end{enumerate}
\end{sol}
\end{ex}

\begin{ex}
Let $\vect{n}$ be a nonzero vector in $\RR^3$, let $L$ be the line through the origin with direction vector $\vect{n}$, and let $M$ be the plane through the origin with normal $\vect{n}$. Show that $P_{L}(\vect{v}) = Q_{L}(\vect{v}) + P_{M}(\vect{v})$ for all $\vect{v}$ in $\RR^3$. [In this case, we say that $P_{L} = Q_{L} + P_{M}$.]
\end{ex}

\begin{ex}
If $M$ is the plane through the origin in $\RR^3$ with normal $\vect{n} = \leftB
\begin{array}{r}
a\\
b\\
c 
\end{array}
\rightB$, show that $Q_{M}$ has matrix
\begin{equation*}{\small
\frac{1}{a^2 + b^2 + c^2}}{\footnotesize \leftB
\begin{array}{ccc}
b^2 + c^2 - a^2 & -2ab & -2ac \\
-2ab & a^2 + c^2 - b^2 & -2bc \\
-2ac & -2bc & a^2 + b^2 - c^2
\end{array}
\rightB}
\end{equation*}
\end{ex}
\end{multicols}

\section{The Cofactor Expansion}
\label{sec:3_1}

In Section~\ref{sec:2_4} we defined the determinant of a $2 \times 2$ matrix
$A = \leftB
\begin{array}{cc}
a & b \\
c & d
\end{array}
\rightB$ 
 as follows:\footnote{Determinants are commonly written $|A| = \func{det } A$ using vertical bars. We will use both notations.\index{determinants!notation}}
\begin{equation*}
\func{det } A = \left| \begin{array}{cc}
a & b \\
c & d 
\end{array} \right|  = ad - bc
\end{equation*}
and showed (in Example~\ref{exa:004261}) that $A$ has an inverse if and only if det $A \neq 0$. One objective of this chapter is to do this for \textit{any} square matrix  A. There is no difficulty for $1 \times 1$ matrices: If $A = \leftB a \rightB$, we define $\func{det} A = \func{det} \leftB a \rightB  = a$ and note that $A$ is invertible if and only if $a \neq 0$.\index{determinants!defined}\index{determinants!and inverses}\index{determinants!square matrices}\index{inverses!determinants}\index{invertible matrix!determinants}\index{square matrix ($n \times n$ matrix)!determinants}


If $A$ is $3 \times 3$ and invertible, we look for a suitable definition of $\func{det } A$ by trying to carry $A$ to the identity matrix by row operations. The first column is not zero ($A$ is invertible); suppose the (1, 1)-entry $a$ is not zero. Then row operations give
\begin{equation*}
A = \leftB \begin{array}{ccc}
a & b & c \\
d & e & f \\
g & h & i 
\end{array} \rightB
\rightarrow
\leftB \begin{array}{ccc}
a & b & c \\
ad & ae & af \\
ag & ah & ai 
\end{array} \rightB
\rightarrow
\leftB \begin{array}{ccc}
a & b & c \\
0 & ae-bd & af-cd \\
0 & ah-bg & ai-cg 
\end{array} \rightB
=
\leftB \begin{array}{ccc}
a & b & c \\
0 & u & af-cd \\
0 & v & ai-cg 
\end{array} \rightB
\end{equation*}
where $u = ae - bd$ and $v = ah - bg$. Since $A$ is invertible, one of $u$ and $v$ is nonzero (by Example \ref{exa:004627}); suppose that $u \neq 0$. Then the reduction proceeds
\begin{equation*}
A \rightarrow \leftB \begin{array}{ccc}
a & b & c \\
0 & u & af-cd \\
0 & v & ai-cg 
\end{array} \rightB
\rightarrow
\leftB \begin{array}{ccc}
a & b & c \\
0 & u & af-cd \\
0 & uv & u(ai-cg) 
\end{array} \rightB
\rightarrow
\leftB \begin{array}{ccc}
a & b & c \\
0 & u & af-cd \\
0 & 0 & w 
\end{array} \rightB
\end{equation*}
where $w = u(ai - cg)- v(af - cd) = a(aei + bfg + cdh - ceg - afh - bdi)$. We define
\begin{equation}\label{eq:detdefinition}
\func{det } A = aei + bfg + cdh - ceg - afh - bdi
\end{equation}
and observe that $\func{det } A \neq 0$ because $a \func{det } A = w \neq 0$ (is invertible).


To motivate the definition below, collect the terms in Equation \ref{eq:detdefinition} involving the entries $a$, $b$, and $c$ in row 1 of $A$:
\begin{align*}
\func{det } A = \left| \begin{array}{ccc}
a & b & c \\
d & e & f \\
g & h & i 
\end{array} \right| &= aei + bfg + cdh - ceg - afh - bdi \\
&= a (ei-fh) - b(di-fg) + c(dh-eg) \\
&= a \left| \begin{array}{cc}
e & f \\
h & i 
\end{array} \right| -  b \left| \begin{array}{cc}
d & f \\
g & i 
\end{array} \right|
+ c \left| \begin{array}{cc}
d & e \\
g & h 
\end{array} \right|
\end{align*}
This last expression can be described as follows: To compute the determinant of a $3 \times 3$ matrix $A$,
 multiply each entry in row 1 by a sign times the determinant of the $2 \times
 2$ matrix obtained by deleting the row and column of that entry, and add
 the results. The signs alternate down row 1, starting with $+$. It is 
this observation that we generalize below. \index{determinants!$3 \times 3$}

\begin{example}{}{007706}
\begin{align*}
\func{det}\leftB \begin{array}{rrr}
2 & 3 & 7 \\
-4 & 0 & 6 \\
1 & 5 & 0 
 \end{array} \rightB
&= 2 \left| \begin{array}{rr}
0 & 6 \\
5 & 0 
\end{array} \right| 
- 3 \left| \begin{array}{rr}
-4 & 6 \\
1 & 0 
\end{array} \right| 
+ 7 \left| \begin{array}{rr}
-4 & 0 \\
1 & 5 
\end{array} \right| \\
&= 2 (-30) - 3(-6) + 7(-20) \\ 
&= -182
\end{align*}
\end{example}

This suggests an inductive method of 
defining the determinant of any square matrix in terms of determinants 
of matrices one size smaller. The idea is to define determinants of $3 \times 3$
 matrices in terms of determinants of $2 \times 2$ matrices, then we do $4 \times 4$ 
matrices in terms of $3 \times 3$ matrices, and so on.\index{determinants!inductive method of determination}\index{induction!determinant!determination of}

To describe this, we need some terminology.

\begin{definition}{Cofactors of a Matrix}{007711}
Assume that determinants of $(n - 1) \times (n - 1)$ matrices have been defined. Given the $n \times n$ matrix $A$, let 
\begin{equation*}
A_{ij} \mbox{ denote the } (n - 1) \times (n - 1) \mbox{ matrix  obtained from } A \mbox{ by deleting row } i \mbox{ and column } j.
\end{equation*}
Then the $(i,j)$-\textbf{cofactor} $c_{ij}(A)$ is the scalar defined by \index{cofactor}\index{determinants!cofactor expansion}
\begin{equation*}
c_{ij}(A) = (-1)^{i+j} \func{det}(A_{ij})
\end{equation*}
Here $(-1)^{i+j}$ is called the \textbf{sign} of the $(i, j)$-position. \index{sign}
\end{definition}

\noindent The sign of a position is clearly $1$ or $-1$, and the following diagram is useful for remembering it:
\begin{equation*}
\leftB \begin{array}{ccccc}
+ & - & + & - & \cdots \\
- & + & - & + & \cdots \\
+ & - & + & - & \cdots \\
- & + & - & + & \cdots \\
\vdots & \vdots & \vdots & \vdots & \\
\end{array}\rightB
\end{equation*}
Note that the signs alternate along each row and column with $+$ in the upper left corner.


\begin{example}{}{007723}
Find the cofactors of positions $(1, 2), (3, 1)$, and $(2, 3)$ in the following matrix.
\begin{equation*}
A = \leftB \begin{array}{rrr}
3 & -1 & 6 \\
5 & 2 & 7 \\
8 & 9 & 4
\end{array} \rightB
\end{equation*}
\begin{solution}
Here $A_{12}$ is the matrix $\leftB \begin{array}{rr}
5 & 7 \\
8 & 4 
\end{array} \rightB$ 
 that remains when row $1$ and column $2$ are deleted. The sign of position $(1, 2)$ is $(-1)^{1+2} = -1$ (this is also the $(1, 2)$-entry in the sign diagram), so the $(1, 2)$-cofactor is
\begin{equation*}
c_{12}(A) = (-1)^{1+2} \left| \begin{array}{rr}
5 & 7 \\
8 & 4 
\end{array} \right| 
=
(-1)(5 \cdot 4 - 7 \cdot 8) = (-1)(-36)=36
\end{equation*}
Turning to position $(3, 1)$, we find
\begin{equation*}
c_{31}(A) = (-1)^{3+1}A_{31}= (-1)^{3+1} \left| \begin{array}{rr}
-1 & 6 \\
2 & 7 
\end{array} \right| 
=(+1)(-7-12)=-19
\end{equation*}
Finally, the $(2, 3)$-cofactor is
\begin{equation*}
c_{23}(A) = (-1)^{2+3}A_{23}= (-1)^{2+3} \left| \begin{array}{rr}
3 & -1 \\
8 & 9 
\end{array} \right| 
=
(-1)(27+8)=-35
\end{equation*}
Clearly other cofactors can be found---there are nine in all, one for each position in the matrix.
\end{solution}
\end{example}

We can now define $\func{det } A$ for any square matrix $A$

\begin{definition}{Cofactor expansion of a Matrix}{007740}
Assume that determinants of $(n - 1) \times (n - 1)$ matrices have been defined. If $A = \leftB a_{ij} \rightB$ is $n \times n$ define
\begin{equation*}
\func{det } A = a_{11}c_{11}(A) +  a_{12}c_{12}(A) + \cdots +  a_{1n}c_{1n}(A)
\end{equation*}
This is called the \textbf{cofactor expansion} of $\func{det } A$ along row $1$. \index{cofactor expansion}\index{determinants!cofactor expansion}\index{determinants!$n\times n$}
\end{definition}

It asserts that $\func{det } A$ can be computed by multiplying the entries of row $1$ by the corresponding 
cofactors, and adding the results. The astonishing thing is that $\func{det } A$ can be computed by taking the cofactor expansion along \textit{any row or column}: Simply multiply each entry of that row or column by the corresponding cofactor and add.


\begin{theorem}{Cofactor Expansion Theorem\footnotemark}{007747}
The determinant of an $n \times n$ matrix $A$
 can be computed by using the cofactor expansion along any row or column
 of $A$. That is $\func{det } A$ can be computed by multiplying each entry of the 
row or column by the corresponding cofactor and adding the results. \index{cofactor expansion theorem}
\end{theorem}
\footnotetext{The cofactor expansion is due to Pierre Simon de Laplace (1749--1827), who 
discovered it in 1772 as part of a study of linear differential 
equations. Laplace is primarily remembered for his work in astronomy and
 applied mathematics.\index{Laplace, Pierre Simon de} }

The proof will be given in Section~\ref{sec:3_6}.

\begin{example}{}{007753}
Compute the determinant of
$
A = \leftB \begin{array}{rrr}
3 & 4 & 5 \\
1 & 7 & 2 \\
9 & 8 & -6
\end{array}
\rightB$.

\begin{solution}
The cofactor expansion along the first row is as follows:
\begin{align*}
\func{det } A &= 3c_{11}(A) + 4c_{12}(A) + 5c_{13}(A) \\
&= 3 \left| \begin{array}{rr}
7 & 2 \\
8 & -6 
\end{array} \right| - 4 \left| \begin{array}{rr}
1 & 2 \\
9 & -6 
\end{array} \right| + 5 
\left| \begin{array}{rr}
1 & 7 \\
9 & 8 
\end{array} \right| \\ 
&= 3 (-58) - 4(-24) + 5(-55) \\
&= -353
\end{align*}
Note that the signs alternate along the row (indeed along \textit{any} row or column). Now we compute $\func{det } A$ by expanding along the first column.
\begin{align*}
\func{det } A &= 3c_{11}(A) + 1c_{21}(A) + 9c_{31}(A) \\
&= 3 \left| \begin{array}{rr}
7 & 2 \\
8 & -6 
\end{array} \right| - \left| \begin{array}{rr}
4 & 5 \\
8 & -6 
\end{array} \right| + 9 
\left| \begin{array}{rr}
4 & 5 \\
7 & 2 
\end{array} \right| \\ 
&= 3 (-58) - (-64) + 9(-27) \\
&= -353
\end{align*}
The reader is invited to verify that $\func{det } A$ can be computed by expanding along any other row or column.
\end{solution}
\end{example}

The fact that the cofactor expansion along \textit{any row or column} of a matrix $A$ always gives the same result (the determinant of $A$) is remarkable, to say the least. The choice of a particular row or column can simplify the calculation.

\begin{example}{}{007765}
Compute $\func{det } A$ where
$A = \leftB \begin{array}{rrrr}
3 & 0 & 0 & 0 \\
5 & 1 & 2 & 0 \\
2 & 6 & 0 & -1 \\
-6 & 3 & 1 & 0 
\end{array}
\rightB$.

\begin{solution}
The first choice we must make is which row or column to use in the 
cofactor expansion. The expansion involves multiplying entries by 
cofactors, so the work is minimized when the row or column contains as 
many zero entries as possible. Row $1$ is a best choice in this matrix 
(column $4$ would do as well), and the expansion is
\begin{align*}
\func{det } A &= 3c_{11}(A) + 0c_{12}(A) + 0c_{13}(A) + 0c_{14}(A) \\
&= 3 \left| \begin{array}{rrr}
1 & 2 & 0 \\
6 & 0 & -1 \\
3 & 1 & 0
\end{array}
\right|
\end{align*}
This is the first stage of the calculation, and we have succeeded in expressing the determinant of the $4 \times 4$ matrix $A$
 in terms of the determinant of a $3 \times 3$ matrix. The next stage involves 
this $3 \times 3$ matrix. Again, we can use any row or column for the cofactor 
expansion. The third column is preferred (with two zeros), so
\begin{align*}
\func{det } A &= 3 \left( 0 \left| \begin{array}{rr}
6 & 0 \\
3 & 1 
\end{array}
\right| - (-1)
 \left| \begin{array}{rr}
1 & 2 \\
3 & 1 
\end{array}
\right|
+ 0
 \left| \begin{array}{rr}
1 & 2 \\
6 & 0 
\end{array}
\right|
 \right) \\
&= 3 [ 0 + 1(-5) + 0] \\
&= -15
\end{align*}
This completes the calculation.
\end{solution}
\end{example}

Computing the determinant of a matrix $A$ can be tedious. For example, if $A$
 is a $4 \times 4$ matrix, the cofactor expansion along any row or column 
involves calculating four cofactors, each of which involves the 
determinant of a $3 \times 3$ matrix. And if $A$ is $5 \times 5$, the expansion 
involves five determinants of $4 \times 4$ matrices! There is a clear need for 
some techniques to cut down the work.\footnote{If $A = \leftB \begin{array}{rrr}
a & b & c \\
d & e & f \\
g & h & i 
\end{array} \rightB$ 
 we can calculate $\func{det } A$ by considering 
$\leftB \begin{array}{rrrrr}
a & b & c & a & b\\
d & e & f & d & e \\
g & h & i & g & h 
\end{array} \rightB$ 
obtained from $A$ by adjoining columns $1$ and $2$ on the right. Then $\func{det } A = aei + bfg + cdh - ceg - afh - bdi$, where the positive terms $aei, bfg,$ and $cdh$ are the products down and to the right starting at $a,b$, and $c$, and the negative terms $ceg, afh$, and $bdi$ are the products down and to the left starting at $c, a$, and $b$. \textbf{Warning}: This rule does \textbf{not} apply to $n \times n$ matrices where $n > 3$ or $n = 2$.}


The motivation for the method is the observation (see Example~\ref{exa:007765})
 that calculating a determinant is simplified a great deal when a row or
 column consists mostly of zeros. (In fact, when a row or column 
consists \textit{entirely} of zeros, the determinant is zero---simply expand along that row or column.)


Recall next that one method of \textit{creating}\index{matrix!zeros, creating in matrix}
 zeros in a matrix is to apply elementary row operations to it. Hence, a
 natural question to ask is what effect such a row operation has on the 
determinant of the matrix. It turns out that the effect is easy to 
determine and that elementary \textit{column} operations\index{columns!elementary column operations} can be used in 
the same way. These observations lead to a technique for evaluating 
determinants that greatly reduces the labour involved. The necessary 
information is given in Theorem~\ref{thm:007779}.


\begin{theorem}{}{007779}
Let $A$ denote an $n \times n$ matrix.

\begin{enumerate}
\item If A has a row or column of zeros, $\func{det } A = 0$.
\item If two distinct rows (or columns) of $A$ are interchanged, the determinant of the resulting matrix is $- \func{det } A$.

\item If a row (or column) of $A$ is multiplied by a constant $u$, the determinant of the resulting matrix is $u(\func{det } A)$.

\item If two distinct rows (or columns) of $A$ are identical, $\func{det } A = 0$.

\item If
 a multiple of one row of $A$ is added to a different row (or if a 
multiple of a column is added to a different column), the determinant of
 the resulting matrix is $\func{det } A$.
\end{enumerate}
\end{theorem}

\begin{proof}
We prove properties 2, 4, and 5 and leave the rest as exercises.

\textit{Property 2}. If $A$ is $n \times n$, this follows by induction on $n$. If $n = 2$, the verification is left to the reader. If $n > 2$ and two rows are interchanged, let $B$ denote the resulting matrix. Expand $\func{det } A$ and $\func{det } B$ along a row \textit{other than} the two that were interchanged. The entries in this row are the same for both $A$ and $B$, but the cofactors in $B$ are the negatives of those in $A$ (by induction) because the corresponding $(n - 1) \times (n - 1)$ matrices have two rows interchanged. Hence, $\func{det } B = - \func{det } A$, as required. A similar argument works if two columns are interchanged.


\textit{Property 4}. If two rows of $A$ are equal, let $B$ be the matrix obtained by interchanging them. Then $B = A$, so $\func{det } B = det A$. But $\func{det } B = -\func{det } A$ by property 2, so $\func{det } A = \func{det } B = 0$. Again, the same argument works for columns.


\textit{Property 5}. Let $B$ be obtained from $A = \leftB a_{ij}\rightB$ by adding $u$ times row $p$ to row $q$. Then row $q$ of $B$ is 
\begin{equation*}
(a_{q1} + ua_{p1},a_{q2} + ua_{p2}, \dots , a_{qn} + ua_{pn})
\end{equation*}
The cofactors of these elements in $B$ are the same as in $A$ (they do not involve row $q$): in symbols, $c_{qj}(B) = c_{qj}(A)$ for each $j$. Hence, expanding $B$ along row $q$ gives
\begin{align*}
\func{det } B &= (a_{q1} + ua_{p1})c_{q1}(A) + (a_{q2} + ua_{p2})c_{q2}(A) + \cdots + (a_{qn}+ua_{pn})c_{qn}(A) \\
&= [a_{q1}c_{q1}(A) + a_{q2}c_{q2}(A) + \cdots + a_{qn}c_{qn}(A)] + u[a_{p1}c_{q1}(A) + a_{p2}c_{q2}(A) + \cdots + a_{pn}c_{qn}(A)] \\
&= \func{det } A + u \func{det } C
\end{align*}
where $C$ is the matrix obtained from $A$ by replacing row $q$ by row $p$ (and both expansions are along row $q$). Because rows $p$ and $q$ of $C$ are equal, $\func{det } C = 0$ by property 4. Hence, $\func{det } B = \func{det } A$, as required. As before, a similar proof holds for columns.
\end{proof}

To illustrate Theorem~\ref{thm:007779}, consider the following determinants.

\begin{tabu}{lX[2]}
$
\left|
\begin{array}{rrr}
3 & -1 & 2 \\
2 & 5 & 1 \\
0 & 0 & 0 
\end{array} \right|  =0$  & (because the last row consists of zeros) \\
\\
$ \left| \begin{array}{rrr}
3 & -1 & 5 \\
2 & 8 & 7 \\
1 & 2 & -1 
\end{array} \right| = - \left| \begin{array}{rrr}
5 & -1 & 3 \\
7 & 8 & 2 \\
-1 & 2 & 1 \end{array}\right|$ & (because two columns are interchanged) \\
\\
$
\left|
\begin{array}{rrr}
8 & 1 & 2 \\
3 & 0 & 9 \\
1 & 2 & -1 
\end{array} \right|
= 3 \left|
\begin{array}{rrr}
8 & 1 & 2 \\
1 & 0 & 3 \\
1 & 2 & -1 
\end{array} \right|$  &  (because the second row of the matrix on the left is $3$ times the second row of the matrix on the right) \\
\\
$
\left|
\begin{array}{rrr}
2 & 1 & 2 \\
4 & 0 & 4 \\
1 & 3 & 1 
\end{array} \right| =0$  & (because two columns are identical) \\
\\
$
\left|
\begin{array}{rrr}
2 & 5 & 2 \\
-1 & 2 & 9 \\
3 & 1 & 1 
\end{array} \right|
= \left|
\begin{array}{rrr}
0 & 9 & 20 \\
-1 & 2 & 9 \\
3 & 1 & 1 
\end{array} \right|$  &  (because twice the second row of the matrix on the left was added to the first row) \\
\end{tabu}

\medskip

The following four examples illustrate how Theorem~\ref{thm:007779} is used to evaluate determinants.

\begin{example}{}{007817}
Evaluate $\func{det } A$ when 
$A = \leftB \begin{array}{rrr}
1 & -1 & 3 \\
1 & 0 & -1 \\
2 & 1 & 6 
\end{array}
\rightB$.

\begin{solution}
 The matrix does have zero entries, so expansion along (say) the second 
row would involve somewhat less work. However, a column operation can be
 used to get a zero in position $(2, 3$)---namely, add column 1 to column 3.
 Because this does not change the value of the determinant, we obtain
\begin{equation*}
\func{det } A = \left| \begin{array}{rrr}
1 & -1 & 3 \\
1 & 0 & -1 \\
2 & 1 & 6 
\end{array} \right| =  \left| \begin{array}{rrr}
1 & -1 & 4 \\
1 & 0 & 0 \\
2 & 1 & 8 
\end{array} \right|
= - \left| \begin{array}{rr}
-1 & 4 \\
1 & 8 
\end{array} \right|
=12
\end{equation*}
where we expanded the second $3 \times 3$ matrix along row 2.
\end{solution}
\end{example}

\begin{example}{}{007825}
If $\func{det} \leftB \begin{array}{rrr}
a & b & c \\
p & q & r \\
x & y & z
\end{array}
\rightB = 6$,
 evaluate $\func{det } A$ where $A = \leftB \begin{array}{ccc}
a+x & b+y & c+z \\
3x & 3y & 3z \\
-p & -q & -r
\end{array}\rightB$.
 
\begin{solution}
First take common factors out of rows 2 and 3.
\begin{equation*}
\func{det } A = 3(-1) \func{det} \leftB \begin{array}{ccc}
a+x & b+y & c+z \\
x & y & z \\
p & q & r 
\end{array}
\rightB
\end{equation*}
Now subtract the second row from the first and interchange the last two rows.
\begin{equation*}
\func{det } A = -3 \func{det} \leftB \begin{array}{ccc}
a & b & c \\
x & y & z \\
p & q & r 
\end{array}
\rightB
= 3 \func{det} \leftB \begin{array}{ccc}
a & b & c \\
p & q & r \\
x & y & z 
\end{array}
\rightB
 = 3 \cdot 6 = 18
\end{equation*}
\end{solution}
\end{example}

The determinant of a matrix is a sum of products of its entries\index{sum!of product of matrix entries}. In particular, if these entries are polynomials in $x$, then the determinant itself is a polynomial in $x$. It is often of interest to determine which values of $x$ make the determinant zero, so it is very useful if the determinant is given in factored form. Theorem~\ref{thm:007779} can help.\index{determinants!defined}\index{polynomials!as matrix entries and determinants}

\begin{example}{}{007837}
Find the values of $x$ for which $\func{det } A = 0$, where 
$A = \leftB \begin{array}{ccc}
1 & x & x \\
x & 1 & x \\
x & x & 1 
\end{array}
\rightB$.

\begin{solution}
To evaluate $\func{det } A$, first subtract $x$ times row 1 from rows 2 and 3.
\begin{equation*}
\func{det } A = \left| \begin{array}{ccc}
1 & x & x \\
x & 1 & x \\
x & x & 1
\end{array} \right|
= 
\left| \begin{array}{ccc}
1 & x & x \\
0 & 1-x^2 & x-x^2 \\
0 & x-x^2 & 1-x^2
\end{array} \right|
= 
\left| \begin{array}{cc}
1-x^2 & x-x^2  \\
x-x^2 & 1-x^2 
\end{array} \right|
\end{equation*}
At this stage we could simply evaluate the determinant (the result is $2x^3-3x^2+1$). But then we would have to factor this polynomial to find the values of $x$
 that make it zero. However, this factorization can be obtained directly
 by first factoring each entry in the determinant and taking a common 
factor of $(1-x)$ from each row.
\begin{align*}
\func{det } A = \left| \begin{array}{cc}
(1-x)(1+x) & x(1-x) \\
x(1-x) & (1-x)(1+x)
\end{array}
\right| &= 
(1-x)^2 \left| \begin{array}{cc}
1+x & x \\
x & 1+x
\end{array} \right| \\
&= (1-x)^2(2x+1)
\end{align*}
Hence, $\func{det } A = 0$ means $(1 - x)^2(2x + 1) = 0$, that is $x = 1$ or $x = -\frac{1}{2}$.
\end{solution}
\end{example}


\begin{example}{}{007851}
If $a_1$, $a_2$, and $a_3$ are given show that
\begin{equation*}
\func{det}\leftB \begin{array}{ccc}
1 & a_1 & a_1^2 \\
1 & a_2 & a_2^2 \\
1 & a_3 & a_3^2 
\end{array}
\rightB = (a_3-a_1)(a_3-a_2)(a_2-a_1)
\end{equation*}
\begin{solution}
Begin by subtracting row 1 from rows 2 and 3, and then expand along column 1:
\begin{equation*}
\func{det} \leftB \begin{array}{ccc}
1 & a_1 & a_1^2 \\
1 & a_2 & a_2^2 \\
1 & a_3 & a_3^2 
\end{array}
\rightB = \func{det} \leftB \begin{array}{ccc}
1 & a_1 & a_1^2 \\
0 & a_2-a_1 & a_2^2-a_1^2 \\
0 & a_3-a_1 & a_3^2-a_1^2
\end{array}
\rightB
= \func{det} \leftB \begin{array}{cc}
a_2-a_1 & a_2^2-a_1^2 \\
a_3-a_1 & a_3^2-a_1^2  
\end{array}
\rightB
\end{equation*}
Now $(a_2 - a_1)$ and $(a_3 - a_1)$ are common factors in rows 1 and 2, respectively, so
\begin{align*}
\func{det} \leftB \begin{array}{ccc}
1 & a_1 & a_1^2 \\
1 & a_2 & a_2^2 \\
1 & a_3 & a_3^2 
\end{array}
\rightB &= (a_2-a_1)(a_3-a_1)\func{det} \leftB \begin{array}{cc}
1& a_2+a_1 \\
1 & a_3+a_1
\end{array} \rightB \\
&= (a_2-a_1)(a_3-a_1)(a_3-a_2)
\end{align*}
\end{solution}
\end{example}

\noindent The matrix in Example~\ref{exa:007851} is called a Vandermonde matrix, and the formula for its determinant can be generalized to the $n \times n$ case (see Theorem \ref{thm:008552}).\index{determinants!Vandermonde matrix}\index{matrix!Vandermonde matrix}\index{Vandermonde matrix}


If $A$ is an $n \times n$ matrix, forming $uA$ means multiplying \textit{every} row of $A$ by $u$. Applying property 3 of Theorem~\ref{thm:007779}, we can take the common factor $u$ out of each row and so obtain the following useful result.


\begin{theorem}{}{007870}
If A is an $ n \times n$ matrix, then $\func{det}(uA) = u^n \func{det } A$ for any number $u$.
\end{theorem}

The next example displays a type of matrix whose determinant is easy to compute.

\begin{example}{}{007875}
Evaluate $\func{det } A$ if 
$A = \leftB \begin{array}{rrrr}
a & 0 & 0 & 0 \\
u & b & 0 & 0 \\
v & w & c & 0 \\
x & y & z & d 
\end{array} \rightB$.


\begin{solution}
Expand along row 1 to get $\func{det } A = a \left| \begin{array}{rrr}
b & 0 & 0 \\
w & c & 0 \\
y & z & d 
\end{array} \right|$. Now expand this along the top row to get  $\func{det } A = ab \left| \begin{array}{cc}
c & 0 \\
z & d 
 \end{array} \right| = abcd$, the product of the main diagonal entries.
\end{solution}
\end{example}

A square matrix is called a \textbf{lower triangular matrix}\index{lower triangular matrix}\index{matrix!lower triangular matrix}\index{square matrix ($n \times n$ matrix)!lower triangular matrix} if all entries above the main diagonal are zero (as in Example~\ref{exa:007875}). Similarly, an \textbf{upper triangular matrix}\index{upper triangular matrix}\index{matrix!upper triangular matrix}\index{square matrix ($n \times n$ matrix)!upper triangular matrix} is one for which all entries below the main diagonal are zero. A \textbf{triangular matrix}\index{triangular matrices}\index{matrix!triangular matrices}\index{square matrix ($n \times n$ matrix)!triangular matrix} is one that is either upper or lower triangular. Theorem~\ref{thm:007885} gives an easy rule for calculating the determinant of any triangular matrix. The proof is like the solution to Example~\ref{exa:007875}.


\begin{theorem}{}{007885}
If A is a square triangular matrix, then det A is the product of the entries on the main diagonal.\index{determinants!triangular matrix}
\end{theorem}

\noindent Theorem~\ref{thm:007885} is useful in computer calculations because it is a routine matter to carry a matrix to triangular form using row operations.

Block matrices such as those in the 
next theorem arise frequently in practice, and the theorem gives an easy
 method for computing their determinants. This dovetails with Example \ref{exa:004627}.\index{block matrix}

\begin{theorem}{}{007890}
Consider matrices 
$\leftB \begin{array}{cc}
A & X \\
0 & B 
\end{array} \rightB$
and $\leftB \begin{array}{cc}
A & 0 \\
Y & B 
\end{array} \rightB$ in block form, where $A$ and $B$ are square matrices. Then
\begin{equation*}
\func{det} \leftB \begin{array}{cc}
A & X \\
0 & B 
\end{array}\rightB = \func{det } A \func{det } B \mbox{ and }
\func{det} \leftB \begin{array}{cc}
A & 0 \\
Y & B 
\end{array}\rightB = \func{det } A \func{det } B
\end{equation*}\index{determinants!block matrix}
\end{theorem}

\begin{proof}
Write $ T = \func{det} \leftB \begin{array}{cc}
A & X \\
0 & B 
\end{array}\rightB$
 and proceed by induction on $k$ where $A$ is $k \times k$. If $k = 1$, it is the cofactor expansion along column 1. In general let $S_i(T)$ denote the matrix obtained from $T$ by deleting row $i$ and column 1. Then the cofactor expansion of $\func{det } T$ along the first column is
\begin{equation}\label{eq:cofexpdeterminant}
\func{det } T = a_{11}\func{det}(S_1(T))-a_{21}\func{det}(S_2(T)) + \cdots \pm a_{k1}\func{det}(S_k(T))
\end{equation}
where $a_{11}, a_{21}, \cdots, a_{k1}$  are the entries in the first column of $A$. But $S_i(T) = \leftB \begin{array}{cc}
S_i(A) & X_i \\
0 & B 
\end{array} \rightB$ 
 for each $i = 1, 2, \cdots, k$, so $\func{det}(S_i(T)) = \func{det}(S_i(A)) \cdot \func{det } B$ by induction. Hence, Equation \ref{eq:cofexpdeterminant} becomes
\begin{align*}
\func{det } T &= \left\{ a_{11}\func{det}(S_1(T))-a_{21}\func{det}(S_2(T)) + \cdots \pm a_{k1}\func{det}(S_k(T))\right\} \func{det } B \\
&= \left\{ \func{det } A \right\} \func{det } B
\end{align*}
as required. The lower triangular case is similar.
\end{proof}

\begin{example}{}{007910}
\begin{equation*}
\func{det} \leftB \begin{array}{rrrr}
2 & 3 & 1 & 3 \\
1 & -2 & -1 & 1 \\
0 & 1 & 0 & 1 \\
0 & 4 & 0 & 1
\end{array} \rightB
= 
- \left| \begin{array}{rrrr}
2 & 1 & 3 & 3 \\
1 & -1 & -2 & 1 \\
0 & 0 & 1 & 1 \\
0 & 0 & 4 & 1
\end{array} \right|
=
-\left| \begin{array}{rr}
2 & 1 \\
1 & -1 
\end{array} \right|
\left| \begin{array}{rr}
1 & 1 \\
4 & 1 
\end{array} \right|
= - (-3)(-3) = -9
\end{equation*}
\end{example}

The next result shows that $\func{det } A$ is a linear transformation when regarded as a function of a fixed column of $A$. The proof is Exercise \ref{ex:3_1_21}.

\begin{theorem}{}{007914}
Given columns $\vect{c}_{1}, \cdots , \vect{c}_{j-1}, \vect{c}_{j+1}, \cdots , \vect{c}_{n}$ in $\RR^n$, define $T: \RR^n \to \RR$ by 
\begin{equation*}
T(\vect{x}) = \func{det} \leftB 
\begin{array}{ccccccc}
\vect{c}_1 & \cdots & \vect{c}_{j-1} & \vect{x} & \vect{c}_{j+1} & \cdots & \vect{c}_n
\end{array} \rightB
\mbox{ for all } \vect{x} \mbox{ in } \RR^n
\end{equation*}
Then, for all $\vect{x}$ and $\vect{y}$ in $\RR^n$ and all $a$ in $\RR$,
\begin{equation*}
T(\vect{x}+\vect{y})= T(\vect{x}) + T(\vect{y}) \quad \mbox{ and } \quad T(a\vect{x}) = aT(\vect{x})
\end{equation*}
\end{theorem}

\section*{Exercises for \ref{sec:3_1}}

\begin{Filesave}{solutions}
\solsection{Section~\ref{sec:3_1}}
\end{Filesave}

\begin{multicols}{2}
\begin{ex}
Compute the determinants of the following matrices.
\begin{exenumerate}
\exitem $\leftB \begin{array}{rr}
2 & -1 \\
3 & 2
\end{array} \rightB$
\exitem $\leftB \begin{array}{rr}
6 & 9 \\
8 & 12
\end{array} \rightB$
\exitem $\leftB \begin{array}{rr}
a^2 & ab \\
ab & b^2
\end{array} \rightB$
\exitem $\leftB \begin{array}{cc}
a+1 & a \\
a & a-1
\end{array} \rightB$
\exitem $\leftB \begin{array}{rr}
\cos \theta & -\sin \theta \\
\sin \theta & \cos \theta
\end{array} \rightB$
\exitem $\leftB \begin{array}{rrr}
2 & 0 & -3 \\
1 & 2 & 5 \\
0 & 3 & 0
\end{array} \rightB$
\exitem $\leftB \begin{array}{rrr}
1 & 2 & 3 \\
4 & 5 & 6 \\
7 & 8 & 9
\end{array} \rightB$
\exitem $\leftB \begin{array}{rrr}
0 & a & 0 \\
b & c & d \\
0 & e & 0
\end{array} \rightB$
\exitem $\leftB \begin{array}{rrr}
1 & b & c \\
b & c & 1 \\
c & 1 & b
\end{array} \rightB$
\exitem $\leftB \begin{array}{rrr}
0 & a & b \\
a & 0 & c \\
b & c & 0
\end{array} \rightB$
\exitem $\leftB \begin{array}{rrrr}
0 & 1 & -1 & 0 \\
3 & 0 & 0 & 2 \\
0 & 1 & 2 & 1 \\
5 & 0 & 0 & 7
\end{array} \rightB$
\exitem $\leftB \begin{array}{rrrr}
1 & 0 & 3 & 1 \\
2 & 2 & 6 & 0 \\
-1 & 0 & -3 & 1 \\
4 & 1 & 12 & 0
\end{array} \rightB$
\exitem $\leftB \begin{array}{rrrr}
3 & 1 & -5 & 2 \\
1 & 3 & 0 & 1 \\
1 & 0 & 5 & 2 \\
1 & 1 & 2 & -1
\end{array} \rightB$
\exitem $\leftB \begin{array}{rrrr}
4 & -1 & 3 & -1 \\
3 & 1 & 0 & 2 \\
0 & 1 & 2 & 2 \\
1 & 2 & -1 & 1
\end{array} \rightB$
\exitem $\leftB \begin{array}{rrrr}
1 & -1 & 5 & 5 \\
3 & 1 & 2 & 4 \\
-1 & -3 & 8 & 0 \\
1 & 1 & 2 & -1
\end{array} \rightB$
\exitem $\leftB \begin{array}{rrrr}
0 & 0 & 0 & a \\
0 & 0 & b & p \\
0 & c & q & k \\
d & s & t & u
\end{array} \rightB$
\end{exenumerate}

\begin{sol}
\begin{enumerate}[label={\alph*.}]
\setcounter{enumi}{1}
\item $0$
\setcounter{enumi}{3}
\item $-1$
\setcounter{enumi}{5}
\item $-39$
\setcounter{enumi}{7}
\item $0$
\setcounter{enumi}{9}
\item $2abc$
\setcounter{enumi}{11}
\item $0$
\setcounter{enumi}{13}
\item $-56$
\setcounter{enumi}{15}
\item $abcd$
\setcounter{enumi}{17}
\end{enumerate}
\end{sol}
\end{ex}

\begin{ex}
Show that $\func{det } A = 0$ if $A$ has a row or column consisting of zeros.
\end{ex}

\begin{ex}
Show that the sign of the position in the last row and the last column of $A$ is always $+1$.
\end{ex}

\begin{ex}
Show that $\func{det } I = 1$ for any identity matrix $I$.
\end{ex}

\begin{ex}
Evaluate the determinant of each matrix by reducing it to upper triangular form.

\begin{exenumerate}
\exitem $\leftB \begin{array}{rrr}
1 & -1 & 2 \\
3 & 1 & 1 \\
2 & -1 & 3
\end{array} \rightB$
\exitem $\leftB \begin{array}{rrr}
-1 & 3 & 1 \\
2 & 5 & 3  \\
1 & -2 & 1
\end{array} \rightB$
\exitem $\leftB \begin{array}{rrrr}
-1 & -1 & 1 & 0 \\
2 & 1 & 1 & 3 \\
0 & 1 & 1 & 2 \\
1 & 3 & -1 & 2
\end{array} \rightB$
\exitem $\leftB \begin{array}{rrrr}
2 & 3 & 1 & 1 \\
0 & 2 & -1 & 3 \\
0 & 5 & 1 & 1 \\
1 & 1 & 2 & 5
\end{array} \rightB$
\end{exenumerate}
\begin{sol}
\begin{enumerate}[label={\alph*.}]
\setcounter{enumi}{1}
\item  $-17$

\setcounter{enumi}{3}
\item  $106$

\end{enumerate}
\end{sol}
\end{ex}

\begin{ex}
Evaluate by cursory inspection:


\begin{enumerate}[label={\alph*.}]
\item $\func{det} \leftB \begin{array}{ccc}
a & b & c  \\
a+1 & b+1 & c+1  \\
a-1 & b-1 & c-1 
\end{array} \rightB$

\item $\func{det} \leftB \begin{array}{ccc}
a & b & c  \\
a+b & 2b & c+b  \\
2 & 2 & 2 
\end{array} \rightB$

\end{enumerate}
\begin{sol}
\begin{enumerate}[label={\alph*.}]
\setcounter{enumi}{1}
\item  $0$

\end{enumerate}
\end{sol}
\end{ex}

\begin{ex}
If $\func{det} \leftB \begin{array}{rrr}
a & b & c  \\
p & q & r  \\
x & y & z 
\end{array} \rightB = -1$
compute:

\begin{enumerate}[label={\alph*.}]
\item $\func{det} \leftB \begin{array}{ccc}
-x & -y & -z  \\
3p+a & 3q+b & 3r+c  \\
2p & 2q & 2r 
\end{array} \rightB$


\item $\func{det} \leftB \begin{array}{ccc}
-2a & -2b & -2c  \\
2p+x & 2q+y & 2r+z  \\
3x & 3y & 3z 
\end{array} \rightB$


\end{enumerate}
\begin{sol}
\begin{enumerate}[label={\alph*.}]
\setcounter{enumi}{1}
\item  $12$

\end{enumerate}
\end{sol}
\end{ex}

\begin{ex}
Show that:


\begin{enumerate}[leftmargin=1em,label={\alph*.}]
\item $\func{det} \leftB \begin{array}{rrr}
p+x & q+y & r+z  \\
a+x & b+y & c+z  \\
a+p & b+q & c+r 
\end{array} \rightB = 2\func{det} \leftB \begin{array}{rrr}
a & b & c  \\
p & q & r  \\
x & y & z 
\end{array} \rightB$


\item  $\func{det} \leftB \begin{array}{rrr}
2a+p & 2b+q & 2c+r  \\
2p+x & 2q+y & 2r+z  \\
2x+a & 2y+b & 2z+c 
\end{array} \rightB = 9\func{det} \leftB \begin{array}{rrr}
a & b & c  \\
p & q & r  \\
x & y & z 
\end{array} \rightB$



\end{enumerate}
\begin{sol}
\begin{enumerate}[label={\alph*.}]
\setcounter{enumi}{1}
\item $\func{det} \leftB \begin{array}{rrr}
2a+p & 2b+q & 2c+r  \\
2p+x & 2q+y & 2r+z  \\
2x+a & 2y+b & 2z+c 
\end{array} \rightB$ \\ 
$= 3 \func{det} \leftB \begin{array}{rrr}
a+p+x & b+q+y & c+r+z  \\
2p+x & 2q+y & 2r+z  \\
2x+a & 2y+b & 2z+c 
\end{array} \rightB$ \\
$= 3 \func{det} \leftB \begin{array}{rrr}
a+p+x & b+q+y & c+r+z  \\
p-a & q-b & r-c  \\
x-p & y-q & z-r 
\end{array} \rightB$ \\
$= 3 \func{det} \leftB \begin{array}{rrr}
3x & 3y & 3z  \\
p-a & q-b & r-c  \\
x-p & y-q & z-r 
\end{array} \rightB \cdots$

\end{enumerate}
\end{sol}
\end{ex}

\begin{ex}
In each case either prove the statement or give an example showing that it is false:


\begin{enumerate}[label={\alph*.}]
\item $\func{det}(A + B) = \func{det } A + \func{det } B.$

\item If $\func{det } A = 0$, then $A$ has two equal rows.

\item If $A$ is $2 \times 2$, then $\func{det}(A^T) = \func{det } A$.

\item If $R$ is the reduced row-echelon form of $A$, then $\func{det } A = \func{det } R$.

\item If $A$ is $2 \times 2$, then $\func{det}(7A) = 49 \func{det } A$.

\item $\func{det}(A^T) = - \func{det } A$.

\item $\func{det}(-A) = - \func{det } A$.

\item If $\func{det } A = \func{det } B$ where $A$ and $B$ are the same size, then $A$ = $B$.

\end{enumerate}
\begin{sol}
\begin{enumerate}[label={\alph*.}]
\setcounter{enumi}{1}
\item  False. $A = \leftB \begin{array}{rr}
1 & 1 \\
2 & 2 
\end{array} \rightB$ 


\setcounter{enumi}{3}
\item  False. $A = \leftB \begin{array}{rr}
2 & 0 \\
0 & 1 
\end{array} \rightB \rightarrow
R = \leftB \begin{array}{rr}
1 & 0 \\
0 & 1 
\end{array} \rightB
$ 


\setcounter{enumi}{5}
\item  False. $A = \leftB \begin{array}{rr}
1 & 1 \\
0 & 1 
\end{array} \rightB$ 


\setcounter{enumi}{7}
\item  False. $A = \leftB \begin{array}{rr}
1 & 1 \\
0 & 1 
\end{array} \rightB$ 
 and $B = \leftB \begin{array}{rr}
1 & 0 \\
1 & 1 
\end{array} \rightB$ 


\end{enumerate}
\end{sol}
\end{ex}

\columnbreak
\begin{ex}
Compute the determinant of each matrix, using Theorem~\ref{thm:007890}.


\begin{enumerate}[label={\alph*.}]
\item $\leftB \begin{array}{rrrrr}
1 & -1 & 2 & 0 & -2 \\
0 & 1 & 0 & 4 & 1 \\
1 & 1 & 5 & 0 & 0 \\
0 & 0 & 0 & 3 & -1 \\
0 & 0 & 0 & 1 & 1 
\end{array}
\rightB$

\item $\leftB \begin{array}{rrrrr}
1 & 2 & 0 & 3 & 0 \\
-1 & 3 & 1 & 4 & 0 \\
0 & 0 & 2 & 1 & 1 \\
0 & 0 & -1 & 0 & 2 \\
0 & 0 & 3 & 0 & 1 
\end{array}
\rightB$


\end{enumerate}
\begin{sol}
\begin{enumerate}[label={\alph*.}]
\setcounter{enumi}{1}
\item  $35$

\end{enumerate}
\end{sol}
\end{ex}

\begin{ex}
If $\func{det } A = 2, \func{det } B = -1$, and $\func{det } C = 3$, find:

\begin{exenumerate}
\exitem $\func{det} \leftB \begin{array}{ccc}
A & X & Y \\
0 & B & Z \\
0 & 0 & C
\end{array}\rightB$
\exitem $\func{det} \leftB \begin{array}{ccc}
A & 0 & 0 \\
X & B & 0 \\
Y & Z & C
\end{array}\rightB$
\exitem $\func{det} \leftB \begin{array}{ccc}
A & X & Y \\
0 & B & 0 \\
0 & Z & C
\end{array}\rightB$
\exitem $\func{det} \leftB \begin{array}{ccc}
A & X & 0 \\
0 & B & 0 \\
Y & Z & C
\end{array}\rightB$
\end{exenumerate}
\begin{sol}
\begin{enumerate}[label={\alph*.}]
\setcounter{enumi}{1}
\item  $-6$

\setcounter{enumi}{3}
\item  $-6$

\end{enumerate}
\end{sol}
\end{ex}

\begin{ex}
If $A$ has three columns with only the top two entries nonzero, show that $\func{det } A = 0$.
\end{ex}

\begin{ex}
\begin{enumerate}[label={\alph*.}]
\item Find $\func{det } A$ if $A$ is $3 \times 3$ and $\func{det}(2A) = 6$.

\item Under what conditions is $\func{det}(-A) = \func{det } A$?

\end{enumerate}
\end{ex}

\begin{ex}
Evaluate by first adding all other rows to the first row.


\begin{enumerate}[label={\alph*.}]
\item $\func{det } \leftB \begin{array}{ccc}
x-1 & 2 & 3 \\
2 & -3 & x-2 \\
-2 & x & -2
\end{array}\rightB$


\item $\func{det } \leftB \begin{array}{ccc}
x-1 & -3 & 1 \\
2 & -1 & x-1 \\
-3 & x+2 & -2
\end{array}\rightB$


\end{enumerate}
\begin{sol}
\begin{enumerate}[label={\alph*.}]
\setcounter{enumi}{1}
\item  $-(x-2)(x^2 + 2x-12)$

\end{enumerate}
\end{sol}
\end{ex}

\begin{ex}
\begin{enumerate}[label={\alph*.}]
\item Find $b$ if 
$\func{det} \leftB \begin{array}{rrr}
5 & -1 & x \\
2 & 6 & y \\
-5 & 4 & z
\end{array}\rightB = ax + by + cz$.

\item Find $c$ if 
$\func{det} \leftB \begin{array}{rrr}
2 & x & -1 \\
1 & y & 3 \\
-3 & z & 4
\end{array}\rightB = ax + by + cz$.

\end{enumerate}
\begin{sol}
\begin{enumerate}[label={\alph*.}]
\setcounter{enumi}{1}
\item  $-7$

\end{enumerate}
\end{sol}
\end{ex}

\begin{ex}
Find the real numbers $x$ and $y$ such that $\func{det } A = 0$ if:


\begin{exenumerate}
\exitem $A = \leftB \begin{array}{rrr}
0 & x & y \\
y & 0 & x \\
x & y & 0 
\end{array}\rightB$
\exitem $A = \leftB \begin{array}{rrr}
1 & x & x \\
-x & -2 & x \\
-x & -x & -3 
\end{array}\rightB$
\exitem* $A = \leftB \begin{array}{rrrr}
1 & x & x^2 & x^3  \\
x & x^2 & x^3 & 1 \\
x^2 & x^3 & 1 & x \\
x^3 & 1 & x & x^2 
\end{array}\rightB$
\exitem* $A = \leftB \begin{array}{rrrr}
x & y & 0 & 0  \\
0 & x & y & 0 \\
0 & 0 & x & y \\
y & 0 & 0 & x 
\end{array}\rightB$
\end{exenumerate}
\begin{sol}
\begin{enumerate}[label={\alph*.}]
\setcounter{enumi}{1}
\item  $ \pm \frac{\sqrt{6}}{2}$


\setcounter{enumi}{3}
\item  $x = \pm y$

\end{enumerate}
\end{sol}
\end{ex}

\begin{ex}
Show that \\ $\func{det} \leftB \begin{array}{rrrr}
0 & 1 & 1 & 1  \\
1 & 0 & x & x \\
1 & x & 0 & x \\
1 & x & x & 0 
\end{array}\rightB = -3x^2$
\end{ex}

\begin{ex}
Show that \\$\func{det} \leftB \begin{array}{rrrr}
1 & x & x^2 & x^3  \\
a & 1 & x & x^2 \\
p & b & 1 & x \\
q & r & c & 1 
\end{array}\rightB = (1-ax)(1-bx)(1-cx).$
\end{ex}

\begin{ex} \label{ex:3.1.19}
\newline Given the polynomial $p(x) = a + bx + cx^2 + dx^3 + x^4$, the matrix $C = \leftB \begin{array}{rrrr}
0 & 1 & 0 & 0 \\
0 & 0 & 1 & 0 \\
0 & 0 & 0 & 1 \\
-a & -b & -c & -d 
\end{array}\rightB$
 is called the \textbf{companion matrix}\index{matrix!companion matrix}\index{companion matrix}\index{polynomials!companion matrix} of $p(x)$. Show that $\func{det}(xI - C) = p(x)$.
\end{ex}

\begin{ex}
Show that \\ $\func{det} \leftB \begin{array}{rrr}
a+x & b+x & c+x   \\
b+x & c+x & a+x  \\
c+x & a+x & b+x  
\end{array}\rightB \\ = (a+b+c+3x)[(ab+ac+bc)-(a^2+b^2+c^2)]$
\end{ex}

\begin{ex}\label{ex:3_1_21}.
Prove Theorem~\ref{thm:007914}. [\textit{Hint}: Expand the determinant along column \textit{j}.] 

\begin{sol}
Let $\vect{x} = \leftB \begin{array}{c}
x_1 \\
x_2 \\
\vdots \\
x_n
\end{array} \rightB$,  $\vect{y} = \leftB \begin{array}{c}
y_1 \\
y_2 \\
\vdots \\
y_n
\end{array} \rightB$ 
 and $A = \leftB \begin{array}{ccccc} \vect{c}_1 & \cdots & \vect{x} + \vect{y}& \cdots & \vect{c}_n \end{array}\rightB$ where $\vect{x} + \vect{y}$ is in column $j$. Expanding $\func{det } A$ along column $j$ (the one containing $\vect{x} + \vect{y}$):

\begin{align*}
T(\vect{x} + \vect{y}) = \func{det } A &= \sum_{i=1}^{n} (x_i + y_i)c_{ij}(A) \\
&= \sum_{i=1}^{n} x_ic_{ij}(A) + \sum_{i=1}^{n} y_ic_{ij}(A)\\
&= T(\vect{x}) + T(\vect{y})
\end{align*}

Similarly for $T(a\vect{x}) = aT(\vect{x})$.
\end{sol}
\end{ex}
\columnbreak
\begin{ex}
Show that
\begin{equation*}
\func{det} \leftB \begin{array}{ccccc}
0 & 0 & \cdots & 0 & a_1 \\
0 & 0 & \cdots & a_2 & * \\
\vdots & \vdots & & \vdots & \vdots \\
0 & a_{n-1} & \cdots & * & * \\
a_n & * & \cdots & * & *
\end{array}
\rightB
= 
(-1)^k a_1a_2 \cdots a_n
\end{equation*}
where either $n = 2k$ or $n = 2k + 1$, and $*$-entries are arbitrary.
\end{ex}

\begin{ex}
By expanding along the first column, show that:
\begin{equation*}
\func{det} \leftB \begin{array}{ccccccc}
1 & 1 & 0 & 0 & \cdots & 0 & 0 \\
0 & 1 & 1 & 0 & \cdots & 0 & 0 \\
0 & 0 & 1 & 1 & \cdots & 0 & 0 \\
\vdots & \vdots & \vdots & \vdots & & \vdots & \vdots \\
0 & 0 & 0 & 0 & \cdots & 1 & 1 \\
1 & 0 & 0 & 0 & \cdots & 0 & 1 \\
\end{array}
\rightB
= 1 + (-1)^{n+1}
 \end{equation*}

if the matrix is $n \times n, n \geq 2$.
\end{ex}

\begin{ex}
Form matrix $B$ from a matrix $A$ by writing the columns of $A$ in reverse order. Express $\func{det } B$ in terms of $\func{det } A$.

\begin{sol}
If $A$ is $n \times n$, then $\func{det } B = (-1)^k \func{det } A$ where $n = 2k$ or $n = 2k + 1$.
\end{sol}
\end{ex}

\begin{ex}
Prove property 3 of Theorem~\ref{thm:007779} by expanding along the row (or column) in question.
\end{ex}

\begin{ex}
Show that the line through two distinct points $(x_{1}, y_{1})$ and $(x_{2}, y_{2})$ in the plane has equation
\begin{equation*}
\func{det} 
\leftB \begin{array}{ccc}
x & y & 1 \\
x_1 & y_1 & 1 \\
x_2 & y_2 & 1 
\end{array}
\rightB = 0
\end{equation*}
\end{ex}

\begin{ex}
Let $A$ be an $n \times n$ matrix. Given a polynomial $p(x) = a_0 + a_1x + \cdots  + a_mx^m$, we write \\$p(A) = a_{0}I + a_1A + \cdots  + a_mA^m$. 

For example, if $p(x) = 2-3x+5x^2$, then \\ $p(A) = 2I -3A +5A^2$. The \textit{characteristic polynomial} of $A$ is defined to be $c_A(x) = \func{det} [xI - A]$, and the Cayley-Hamilton theorem asserts that $c_A(A) = 0$ for any matrix $A$.


\begin{enumerate}[label={\alph*.}]
\item Verify the theorem for
\begin{multicols}{2}
\begin{enumerate}[label={\roman*.}]
\item 
$A = \leftB \begin{array}{rr}
3 & 2 \\
1 & -1 
\end{array}\rightB$
\item
$A = \leftB \begin{array}{rrr}
1 & -1 & 1 \\
0 & 1 & 0 \\
8 & 2 & 2 
\end{array}\rightB
$
\end{enumerate}
\end{multicols}
\item Prove the theorem for $
A = \leftB \begin{array}{rr}
a & b \\
c & d 
\end{array}\rightB
$
\end{enumerate}
\end{ex}


\end{multicols}

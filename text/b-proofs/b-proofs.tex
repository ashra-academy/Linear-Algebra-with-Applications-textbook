
Logic plays a basic role in human 
affairs. Scientists use logic to draw conclusions from experiments, 
judges use it to deduce consequences of the law, and mathematicians use 
it to prove theorems. Logic arises in ordinary speech with assertions 
such as ``If John studies hard, he will pass the course,'' or ``If an 
integer $n$ is divisible by $6$, then $n$ is divisible by $3$.''\footnote{By an \textit{integer}\index{integers} we mean a ``whole number''\index{whole number}; that is, a number in the set $0, \pm 1, \pm 2, \pm 3, \dots$ }
 In each case, the aim is to assert that if a certain statement is true, then another statement must also be true. In fact, if $p$ and $q$ denote statements, most theorems take the form of an \textbf{implication}\index{implication}: ``If $p$ is true, then $q$ is true.'' We write this in symbols as
\begin{equation*}
p \Rightarrow q 
\end{equation*}
and read it as ``$p$ implies $q$.'' Here $p$ is the \textbf{hypothesis}\index{hypothesis} and $q$ the \textbf{conclusion}\index{conclusion} of the implication. The verification that $p \Rightarrow  q$ is valid is called the \textbf{proof}\index{proof!defined} of the implication. In this section we examine the most common methods of proof\footnote{For a more detailed look at proof techniques see D. Solow, How to Read and 
Do Proofs, 2nd ed. (New York: Wiley, 1990); or J. F. Lucas. \textit{Introduction to Abstract Mathematics}, Chapter 2 (Belmont, CA: Wadsworth, 1986).\index{\textit{Introduction to Abstract Mathematics} (Lucas)}\index{\textit{How to Read and Do Proofs} (Solow)}}
 and illustrate each technique with some examples.


\subsection*{Method of Direct Proof}

To prove that $p \Rightarrow q$, demonstrate directly that $q$ is true whenever $p$ is true.\index{direct proof}\index{proof!direct proof}

\begin{example}{}{034634}
If $n$ is an odd integer, show that $n^{2}$ is odd.


\begin{solution}
  If $n$ is odd, it has the form $n = 2k + 1$ for some integer $k$. Then $n^{2} = 4k^{2} + 4k + 1 = 2(2k^{2} + 2k) + 1$ also is odd because $2k^{2} + 2k$ is an integer.
\end{solution}
\end{example}

Note that the computation $n^{2} = 4k^{2} + 4k + 1$ in Example~\ref{exa:034634}
 involves some simple properties of arithmetic that we did not prove. 
These properties, in turn, can be proved from certain more basic 
properties of numbers (called axioms)\index{axioms}---more about that later. Actually, a
 whole body of mathematical information lies behind nearly every proof 
of any complexity, although this fact usually is not stated explicitly. 
Here is a geometrical example.


\newpage
\begin{example}{}{034648}
In a right triangle, show that the sum of the two acute angles is 90 degrees.


\begin{solution}
\begin{wrapfigure}{l}{5cm} 
\centering

\begin{tikzpicture}[scale=0.85]
%triangle
\draw[dkbluevect,thick] (0,0)--(0,2)--(3,0)--cycle;
\draw[dkbluevect,thick] (0.25,0)--(0.25,0.25)--(0,0.25);
\node[dkbluevect,font=\scriptsize] at (0.15,1.65){$\beta$};
\node[dkbluevect,font=\scriptsize] at (2.4,0.15){$\alpha$};

%rectangle
\draw[dkbluevect,thick] (0,-3)--(0,-1)--(3,-3)--cycle;
\draw[dkbluevect,thick] (0.25,-3)--(0.25,-2.75)--(0,-2.75);
\node[dkbluevect,font=\scriptsize] at (0.15,-1.35){$\beta$};
\node[dkbluevect,font=\scriptsize] at (2.4,-2.85){$\alpha$};
\draw[dkbluevect,thick] (0,-1)--(3,-1)--(3,-3)--cycle;
\draw[dkbluevect,thick] (2.75,-1)--(2.75,-1.25)--(3,-1.25);
\node[dkbluevect,font=\scriptsize] at (2.8,-2.6){$\beta$};
\node[dkbluevect,font=\scriptsize] at (0.6,-1.2){$\alpha$};


\end{tikzpicture}


%\caption{\label{fig:034654}}
\end{wrapfigure}

\setlength{\rightskip}{0pt plus 200pt}
 The right triangle is shown in the diagram. Construct a rectangle with 
sides of the same length as the short sides of the original triangle, 
and draw a diagonal as shown. The original triangle appears on the 
bottom of the rectangle, and the top triangle is identical to the 
original (but rotated). Now it is clear that $\alpha + \beta$ is a right angle.
\vspace{5em}
\end{solution}
\end{example}

Geometry was one of the first subjects in which formal proofs\index{formal proofs}\index{proof!formal proofs} were used---Euclid's \textit{Elements}\index{Euclid}\index{\textit{Elements} (Euclid)} was published about 300 B.C. The \textit{Elements}
 is the most successful textbook ever written, and contains many of the 
basic geometrical theorems that are taught in school today. In 
particular, Euclid included a proof of an earlier theorem (about 500 
B.C.) due to Pythagoras. Recall that, in a right triangle, the side 
opposite the right angle is called the \textit{hypotenuse}\index{triangle!hypotenuse}\index{hypotenuse} of the triangle.


\begin{example}{Pythagoras' Theorem}{034656}
\begin{wrapfigure}{l}{5cm} 
\vspace*{-1em}
\centering

\begin{tikzpicture}[scale=0.85]
%triangle
\draw[dkbluevect,thick] (0.5,0)--(0.5,1)--(2.5,0)--cycle;
\draw[dkbluevect,thick] (0.75,0)--(0.75,0.25)--(0.5,0.25);
\node[dkbluevect,left,font=\footnotesize] at (0.5,0.5){$a$};
\node[dkbluevect,below,font=\footnotesize] at (1.5,0){$b$};
\node[dkbluevect,right,font=\scriptsize] at (0.45,0.65){$\beta$};
\node[dkbluevect,left,font=\scriptsize] at (2,0.15){$\alpha$};
\node[dkbluevect,above,font=\footnotesize] at (1.5,0.5){$c$};

%top square
\draw[dkbluevect,thick] (0,-1) rectangle (3,-4);
\filldraw[fill=ltgreenvect!60!white,draw=dkbluevect,thick] (1,-1) rectangle (3,-2);
\filldraw[fill=ltgreenvect!60!white,draw=dkbluevect,thick] (0,-2) rectangle (1,-4);
\draw[dkbluevect,thick] (1,-1)--(3,-2);
\draw[dkbluevect,thick] (0,-2)--(1,-4);
\node[dkbluevect,left,font=\footnotesize] at (0,-1.5){$a$};
\node[dkbluevect,above,font=\footnotesize] at (0.5,-1){$a$};
\node[dkbluevect,font=\footnotesize] at (0.5,-1.5){$a^2$};
\node[dkbluevect,right,font=\footnotesize] at (3,-3){$b$};
\node[dkbluevect,below,font=\footnotesize] at (2,-3.9){$b$};
\node[dkbluevect,font=\footnotesize] at (2,-3){$b^2$};

%bottom square
\filldraw[fill=ltgreenvect!60!white,draw=dkbluevect,thick] (0,-5) rectangle (3,-8);
\filldraw[fill=white,draw=dkbluevect,thick](2,-5)--(3,-7)--(1,-8)--(0,-6)--cycle;
\node[dkbluevect,font=\footnotesize] at (1.5,-6.5){$c^2$};
\node[dkbluevect,above,font=\footnotesize] at (1,-5){$b$};
\node[dkbluevect,above,font=\footnotesize] at (2.5,-5){$a$};
\node[dkbluevect,right,font=\footnotesize] at (3,-6){$b$};
\node[dkbluevect,right,font=\footnotesize] at (3,-7.5){$a$};
\node[dkbluevect,below,font=\footnotesize] at (2,-8){$b$};
\node[dkbluevect,below,font=\footnotesize] at (0.5,-8){$a$};
\node[dkbluevect,left,font=\footnotesize] at (0,-5.5){$a$};
\node[dkbluevect,left,font=\footnotesize] at (0,-7){$b$};
\node[dkbluevect,left,font=\scriptsize] at (1.65,-5.15){$\alpha$};
\node[dkbluevect,font=\scriptsize] at (2.4,-5.2){$\beta$};
\node[dkbluevect,font=\scriptsize] at (2.85,-6.35){$\alpha$};
\node[dkbluevect,font=\scriptsize] at (2.8,-7.35){$\beta$};
\node[dkbluevect,font=\scriptsize] at (1.6,-7.85){$\alpha$};
\node[dkbluevect,font=\scriptsize] at (0.6,-7.75){$\beta$};
\node[dkbluevect,font=\scriptsize] at (0.15,-6.65){$\alpha$};
\node[dkbluevect,font=\scriptsize] at (0.2,-5.7){$\beta$};

\end{tikzpicture}


%\caption{\label{fig:034667}}
\end{wrapfigure}

\setlength{\rightskip}{0pt plus 200pt}
In a right-angled triangle, show that 
the square of the length of the hypotenuse equals the sum of the squares
 of the lengths of the other two sides.\index{Pythagoras}\index{Pythagoras' theorem}

\begin{solution}
  Let the sides of the right triangle have lengths $a$, $b$, and $c$ as shown. Consider two squares with sides of length $a + b$,
 and place four copies of the triangle in these squares as in the 
diagram. The central rectangle in the second square shown is itself a 
square because the angles $\alpha$ and $\beta$ add to $90$ degrees (using Example~\ref{exa:034648}), so its area is $c^{2}$ as shown. Comparing areas shows that both $a^{2} + b^{2}$ and $c^{2}$ each equal the area of the large square minus four times the area of the original triangle, and hence are equal.
\vspace*{6em}
\end{solution}
\end{example}

Sometimes it is convenient (or even 
necessary) to break a proof into parts, and deal with each case 
separately. We formulate the general method as follows:


\subsection*{Method of Reduction to Cases}

To prove that $p \Rightarrow q$, show that $p$ implies at least one of a list $p_{1}, p_{2}, \dots , p_{n}$ of statements (the cases) and then show that $p_{i} \Rightarrow q$ for each $i$.\index{proof!reduction to cases}\index{reduction to cases}


\begin{example}{}{034677}
Show that $n^{2} \geq 0$ for every integer $n$.


\begin{solution}
  This statement can be expressed as an implication: If $n$ is an integer, then $n^{2} \geq 0$. To prove it, consider the following three cases:
\begin{equation*}
(1) \ n > 0;  \quad (2) \ n = 0; \quad (3) \ n < 0.
\end{equation*}
Then $n^{2} > 0$ in Cases (1) and (3) because the product of two positive (or two negative) integers is positive. In Case (2) $n^{2} = 0^{2} = 0$, so $n^{2} \geq 0$ in every case.
\end{solution}
\end{example}

\begin{example}{}{034691}
If $n$ is an integer, show that $n^{2} - n$ is even.


\begin{solution}
  We consider two cases:
\begin{equation*}
(1) \ n \mbox{ is even; } \quad (2) \ n \mbox{ is odd.}
\end{equation*}
We have $n^{2} - n = n(n - 1)$, so this is even in Case (1) because any multiple of an even number is again even. Similarly, $n - 1$ is even in Case (2) so $n(n - 1)$ is again even for the same reason. Hence $n^{2} - n$ is even in any case.
\end{solution}
\end{example}

The statements used in mathematics are 
required to be either true or false. This leads to a proof technique 
which causes consternation in many beginning students. The method is a 
formal version of a debating strategy whereby the debater assumes the 
truth of an opponent's position and shows that it leads to an absurd 
conclusion.


\subsection*{Method of Proof by Contradiction}

To prove that $p \Rightarrow q$, show that the assumption that both $p$ is true and $q$ is false leads to a contradiction. In other words, if $p$ is true, then $q$ must be true; that is, $p \Rightarrow q$.\index{contradiction, proof by}\index{proof!by contradiction}


\begin{example}{}{034707}
If $r$ is a rational number (fraction), show that $r^{2} \neq 2$.


\begin{solution}
  To argue by contradiction, we assume that $r$ is a rational number and that $r^{2} = 2$, and show that this assumption leads to a contradiction. Let $m$ and $n$ be integers such that $r = \frac{m}{n}$ is in lowest terms (so, in particular, $m$ and $n$ are not both even). Then $r^{2} = 2$ gives $m^{2} = 2n^{2}$, so $m^{2}$ is even. This means $m$ is even (Example~\ref{exa:034634}), say $m = 2k$. But then $2n^{2} = m^{2} = 4k^{2}$, so $n^{2} = 2k^{2}$ is even, and hence $n$ is even. This shows that $n$ and $m$ are both even, contrary to the choice of these numbers.
\end{solution}
\end{example}

\begin{example}{Pigeonhole Principle}{034724}
If $n + 1$ pigeons are placed in $n$ holes, then some hole contains at least $2$ pigeons.

\begin{solution}
  Assume the conclusion is false. Then each hole contains at most one pigeon and so, since there are $n$ holes, there must be at most $n$ pigeons, contrary to assumption.
\end{solution}\index{pigeonhole principle}
\end{example}

The next example involves the notion of a \textit{prime}\index{prime}
 number, that is an integer that is greater than 1 which cannot be 
factored as the product of two smaller positive integers both greater 
than $1$. The first few primes are $2, 3, 5, 7, 11, \dots$.


\begin{example}{}{034732}
If $2^n - 1$ is a prime number, show that $n$ is a prime number.


\begin{solution}
  We must show that $p \Rightarrow q$ where $p$ is the statement ``$2^n - 1$ is a prime'', and $q$ is the statement ``$n$ is a prime.'' Suppose that $p$ is true but $q$ is false so that $n$ is not a prime, say $n = ab$ where $a \geq 2$ and $b \geq 2$ are integers. If we write $2^a = x$, then $2^n = 2^{ab} = (2^a)^b = x^b$. Hence $2^n - 1$ factors:
\begin{equation*}
2^n - 1 = x^b - 1 = (x-1)(x^{b-1} +x^{b-2} + \cdots + x^2 + x + 1 )
\end{equation*}
As $x \geq 4$, this expression is a factorization of $2^n - 1$ into smaller positive integers, contradicting the assumption that $2^n - 1$ is prime.
\end{solution}
\end{example}

The next example exhibits one way to show that an implication is \textit{not} valid.


\begin{example}{}{034752}
Show that the implication ``$n$ is a prime $\Rightarrow 2^n - 1$ is a prime'' is false.


\begin{solution}
  The first four primes are $2$, $3$, $5$, and $7$, and the corresponding values for $2^n - 1$ are $3$, $7$, $31$, $127$ (when $n = 2$, $3$, $5$, $7$). These are all prime as the reader can verify. This 
result seems to be evidence that the implication is true. However, the 
next prime is $11$ and $2^{11} - 1 = 2047 = 23 \cdot 89$, which is clearly \textit{not} a prime.
\end{solution}
\end{example}

\noindent We say that $n = 11$ is a \textbf{counterexample}\index{counterexample} to the (proposed) implication in Example~\ref{exa:034752}.
 Note that, if you can find even one example for which an implication is
 not valid, the implication is false. Thus disproving implications is in
 a sense easier than proving them.


The implications in Example~\ref{exa:034732} and Example~\ref{exa:034752} are closely related: They have the form $p \Rightarrow q$ and $q \Rightarrow p$, where $p$ and $q$ are statements. Each is called the \textbf{converse}\index{converse} of the other and, as these examples show, an implication can be valid even though its converse is not valid. If \textit{both} $p \Rightarrow q$ and $q \Rightarrow p$ are valid, the statements $p$ and $q$ are called \textbf{logically equivalent}\index{logically equivalent}. This is written in symbols as
\begin{equation*}
p \Leftrightarrow q
\end{equation*}
and is read ``$p$ if and only if $q$''. Many of the most 
satisfying theorems make the assertion that two statements, ostensibly 
quite different, are in fact logically equivalent.


\begin{example}{}{034764}
If $n$ is an integer, show that ``$n$ is odd $\Leftrightarrow n^{2}$ is odd.''


\begin{solution}
  In Example~\ref{exa:034634} we proved the implication ``$n$ is odd $\Rightarrow n^{2}$ is odd.'' Here we prove the converse by contradiction. If $n^{2}$ is odd, we assume that $n$ is not odd. Then $n$ is even, say $n = 2k$, so $n^{2} = 4k^{2}$, which is also even, a contradiction.
\end{solution}
\end{example}

Many more examples of proofs can be 
found in this book and, although they are often more complex, most are 
based on one of these methods. In fact, linear algebra is one of the 
best topics on which the reader can sharpen his or her skill at 
constructing proofs. Part of the reason for this is that much of linear 
algebra is developed using the \textbf{axiomatic method}\index{axiomatic method}. That is, in the 
course of studying various examples it is observed that they all have 
certain properties in common. Then a general, abstract system is studied
 in which these basic properties are \textit{assumed} to hold (and are called \textbf{axioms}). In this system, statements (called \textbf{theorems}\index{theorems}) are deduced from the axioms using the methods presented in this appendix. These theorems will then be true in \textit{all}
 the concrete examples, because the axioms hold in each case. But this 
procedure is more than just an efficient method for finding theorems in 
the examples. By reducing the proof to its essentials, we gain a better 
understanding of why the theorem is true and how it relates to analogous
 theorems in other abstract systems.


The
 axiomatic method is not new. Euclid\index{Euclid} first used it in about 300 B.C. to 
derive all the propositions of (euclidean) geometry from a list of 10 
axioms. The method lends itself well to linear algebra. The axioms are 
simple and easy to understand, and there are only a few of them. For 
example, the theory of vector spaces contains a large number of theorems
 derived from only ten simple axioms.

\vspace*{-1em}
\section*{Exercises for \ref{chap:appbproofs}}

\begin{Filesave}{solutions}
\solsection{Appendix~\ref{chap:appbproofs}}
\end{Filesave}

\begin{multicols}{2}
\begin{ex}
In each case prove the result and either prove the converse or give a counterexample.


\begin{enumerate}[label={\alph*.}]
\item If $n$ is an even integer, then $n^{2}$ is a multiple of $4$.

\item If $m$ is an even integer and $n$ is an odd integer, then $m + n$ is odd.

\item If $x = 2$ or $x = 3$, then $x^{3} - 6x^{2} + 11x - 6 = 0$.

\item If $x^{2} - 5x + 6 = 0$, then $x = 2$ or $x = 3$.

\end{enumerate}
\begin{sol}
\begin{enumerate}[label={\alph*.}]
\setcounter{enumi}{1}
\item  If $m = 2p$ and $n = 2q + 1$ where $p$ and $q$ are integers, then $m + n = 2(p + q) + 1$ is odd. The converse is false: $m = 1$ and $n = 2$ is a counterexample.

\setcounter{enumi}{3}
\item  $x^{2} - 5x + 6 = (x - 2)(x - 3)$ so, if this is zero, then $x = 2$ or $x = 3$. The converse is true: each of $2$ and $3$ satisfies $x^{2} - 5x + 6 = 0$.

\end{enumerate}
\end{sol}
\end{ex}

\begin{ex}
In each case either prove the result by splitting into cases, or give a counterexample.

\begin{enumerate}[label={\alph*.}]
\item If $n$ is any integer, then $n^{2} = 4k + 1$ for some integer $k$.

\item If $n$ is any odd integer, then $n^{2} = 8k + 1$ for some integer $k$.

\item If $n$ is any integer, $n^{3} - n = 3k$ for some integer $k$. [\textit{Hint}: Use the fact that each integer has one of the forms $3k$, $3k + 1$, or $3k + 2$, where $k$ is an integer.]

\end{enumerate}
\begin{sol}
\begin{enumerate}[label={\alph*.}]
\setcounter{enumi}{1}
\item  This implication is true. If $n = 2t + 1$ where $t$ is an integer, then $n^{2} = 4t^{2} + 4t + 1 = 4t(t + 1) + 1$. Now $t$ is either even or odd, say $t = 2m$ or $t = 2m + 1$. If $t = 2m$, then $n^{2} = 8m(2m + 1) + 1$; if $t = 2m + 1$, then $n^{2} = 8(2m + 1)(m + 1) + 1$. Either way, $n^{2}$ has the form $n^{2} = 8k + 1$ for some integer $k$.

\end{enumerate}
\end{sol}
\end{ex}

\begin{ex}
In each case prove the result by contradiction and either prove the converse or give a counterexample.


\begin{enumerate}[label={\alph*.}]
\item If $n > 2$ is a prime integer, then $n$ is odd.

\item If $n + m = 25$ where $n$ and $m$ are integers, then one of $n$ and $m$ is greater than $12$.

\item If $a$ and $b$ are positive numbers and $a \leq b$, then $\sqrt{a} \leq \sqrt{b}$.

\item If $m$ and $n$ are integers and $mn$ is even, then $m$ is even or $n$ is even.

\end{enumerate}
\begin{sol}
\begin{enumerate}[label={\alph*.}]
\setcounter{enumi}{1}
\item  Assume that the statement ``one of $m$ and $n$ is greater than $12$'' is false. Then both $n \leq 12$ and $m \leq 12$, so $n + m \leq 24$, contradicting the hypothesis that $n + m = 25$. This proves the implication. The converse is false: $n = 13$ and $m = 13$ is a counterexample.

\setcounter{enumi}{3}
\item  Assume that the statement ``$m$ is even or $n$ is even'' is false. Then both $m$ and $n$ are odd, so $mn$ is odd, contradicting the hypothesis. The converse is true: If $m$ or $n$ is even, then $mn$ is even.

\end{enumerate}
\end{sol}
\end{ex}

\begin{ex}
Prove each implication by contradiction.


\begin{enumerate}[label={\alph*.}]
\item If $x$ and $y$ are positive numbers, then \newline $\sqrt{x + y} \neq \sqrt{x} + \sqrt{y}$.

\item If $x$ is irrational and $y$ is rational, then $x + y$ is irrational.

\item If $13$ people are selected, at least $2$ have birthdays in the same month.

\end{enumerate}
\begin{sol}
\begin{enumerate}[label={\alph*.}]
\setcounter{enumi}{1}
\item  If $x$ is irrational and $y$ is rational, assume that $x + y$ is rational. Then $x = (x + y) - y$ is the difference of two rationals, and so is rational, contrary to the hypothesis.

\end{enumerate}
\end{sol}
\end{ex}

\begin{ex}
Disprove each statement by giving a counterexample.


\begin{enumerate}[label={\alph*.}]
\item $n^{2} + n + 11$ is a prime for all positive integers $n$.

\item $n^{3} \geq 2^n$ for all integers $n \geq 2$.

\item If $n \geq 2$ points are arranged on a circle in such a way that no three of the 
lines joining them have a common point, then these lines divide the 
circle into $2^{n-1}$ regions. [The cases $n = 2$, $3$, and $4$ are shown in the diagram.]
\begin{figure}[H]
\centering

\begin{tikzpicture}[scale=0.5]
%n=2
\draw (0,0) [dkgreenvect,thick] circle (2);
\draw[dkgreenvect,thick] (-1.8,-0.87)--(0.87,1.8);
\fill[dkgreenvect,thick] (-1.8,-0.87) circle (3pt);
\fill[dkgreenvect,thick] (0.87,1.8) circle (3pt);
\node[below,font=\footnotesize] at (0,-2){$n=2$};
%n=3 add 5 to all x coord for shift
\draw (5,0) [dkgreenvect,thick] circle (2);
\draw[dkgreenvect,thick] (4,-1.73)--(6.5,1.32);
\draw[dkgreenvect,thick] (4,-1.73)--(4.5,1.93);
\draw[dkgreenvect,thick] (6.5,1.32)--(4.5,1.93);
\fill[dkgreenvect,thick] (4,-1.73) circle (3pt);
\fill[dkgreenvect,thick] (6.5,1.32) circle (3pt);
\fill[dkgreenvect,thick] (4.5,1.93) circle (3pt);
\node[below,font=\footnotesize] at (5,-2){$n=3$};
%n=4 add 10 to all x coord for shift
\draw (10,0) [dkgreenvect,thick] circle (2);
\draw[dkgreenvect,thick] (9.8,-1.99)--(11.75,-0.97);
\draw[dkgreenvect,thick] (9.8,-1.99)--(10.6,1.91);
\draw[dkgreenvect,thick] (11.75,-0.97)--(10.6,1.91);
\draw[dkgreenvect,thick] (9.8,-1.99)--(8.5,1.32);
\draw[dkgreenvect,thick] (8.5,1.32)--(10.6,1.91);
\draw[dkgreenvect,thick] (11.75,-0.97)--(8.5,1.32);
\fill[dkgreenvect,thick] (9.8,-1.99) circle (3pt);
\fill[dkgreenvect,thick] (11.75,-0.97) circle (3pt);
\fill[dkgreenvect,thick] (10.6,1.91) circle (3pt);
\fill[dkgreenvect,thick] (8.5,1.32) circle (3pt);
\node[below,font=\footnotesize] at (10,-2){$n=4$};

\end{tikzpicture}


%\caption{\label{fig:034841}}
\end{figure}
\end{enumerate}
\begin{sol}
\begin{enumerate}[label={\alph*.}]
\setcounter{enumi}{1}
\item  $n = 10$ is a counterexample because $10^3 = 1000$ while $2^{10} = 1024$, so the statement $n^{3} \geq 2^n$ is false if $n = 10$. Note that $n^{3} \geq 2^n$ \textit{does} hold for $2 \leq n \leq 9$.

\end{enumerate}
\end{sol}
\end{ex}

\begin{ex}
The number $e$ from calculus has a series expansion
\begin{equation*}
e = 1 + \frac{1}{1!} + \frac{1}{2!} + \frac{1}{3!} + \cdots
\end{equation*}
where $n! = n(n - 1) \cdots  3 \cdot 2 \cdot 1$ for each integer $n \geq 1$. Prove that $e$ is irrational by contradiction. [\textit{Hint}: If $e = m/n$, consider
\begin{equation*}
k = n! \left(e- 1 - \frac{1}{1!} - \frac{1}{2!} - \frac{1}{3!} - \cdots - \frac{1}{n!} \right).
\end{equation*}
Show that $k$ is a positive integer and that
\begin{equation*}
k = \frac{1}{n+1} + \frac{1}{(n+1)(n+2)} + \cdots < \frac{1}{n}. ]
\end{equation*}
\end{ex}

\end{multicols}


